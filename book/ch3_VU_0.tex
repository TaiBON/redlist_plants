\footnotesize\selectfont
%\begin{table}[!h]
        {\def\arraystretch{1.5}\tabcolsep=2pt
        \begin{longtable}{L{2.5cm}L{2cm}L{5cm}L{2.5cm}L{3cm}}
        \multicolumn{2}{l}{\large{Lycophytes 石松類植物}} & & \\
        & & & &\\
        \toprule
          \color{red}{\textbf{科名}} & \color{red}{\textbf{科中名}} & \color{red}{\textbf{分類群學名}} & \color{red}{\textbf{分類群中名}} & \color{red}{\textbf{評估標準}} \\
        \midrule 
        \endfirsthead

        \multicolumn{5}{l}{\large\color{red}{\Kai{國家易危 (NVU) 類別維管束植物名錄(續)}}} \\
        \toprule
        \color{red}{\textbf{科名}} & \color{red}{\textbf{科中名}} & \color{red}{\textbf{分類群學名}} & \color{red}{\textbf{分類群中名}} & \color{red}{\textbf{評估標準}} \\
        \midrule
        \endhead
                Lycopodiaceae & 石松科 & \href{http://www.theplantlist.org/tpl1.1/search?q=Huperzia+selago}{\textit{Huperzia selago} } & 小杉葉石松 & D1 \index{Huperzia@\textit{Huperzia}!selago@\textbf{\textit{selago}}}  \index{小杉葉石松@{\Song{小杉葉石松}}} \\
    Lycopodiaceae & 石松科 & \href{http://www.theplantlist.org/tpl1.1/search?q=Lycopodium+annotinum}{\textit{Lycopodium annotinum} } & 杉葉蔓石松 & D1 \index{Lycopodium@\textit{Lycopodium}!annotinum@\textbf{\textit{annotinum}}}  \index{杉葉蔓石松@{\Song{杉葉蔓石松}}} \\
    \bottomrule
        \end{longtable}
    %%\end{table}
        }
    