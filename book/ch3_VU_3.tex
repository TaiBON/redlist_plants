\noindent\normalfont\selectfont Angiosperms 被子植物
\footnotesize\selectfont
%\begin{table}[!h]
        {\def\arraystretch{1.5}\tabcolsep=2pt
        \begin{longtable}{p{2.5cm}p{2.5cm}p{4.5cm}p{2.5cm}p{3cm}}
        \toprule
          科名 & 科中名 & 分類群學名 & 分類群中名 & 評估標準 \\
        \midrule 
        \endfirsthead

        {{\bfseries 續前頁 }} \\
        科名 & 科中名 & 分類群學名 & 分類群中名 & 評估標準 \\
        \midrule
        \endhead
                Acanthaceae & 爵床科 & \textit{Asystasiella neesiana}  & 尼氏擬馬偕花 & B2ac(iv); D2 \index{Asystasiella@\textit{Asystasiella}!neesiana@\textit{neesiana}}  \index{尼氏擬馬偕花} \\
    Acanthaceae & 爵床科 & \textit{Hemigraphis cumingiana}  & 直立半插花 & D2 \index{Hemigraphis@\textit{Hemigraphis}!cumingiana@\textit{cumingiana}}  \index{直立半插花} \\
    Acanthaceae & 爵床科 & \textit{Hemigraphis primulifolia}  & 恆春半插花 & D2 \index{Hemigraphis@\textit{Hemigraphis}!primulifolia@\textit{primulifolia}}  \index{恆春半插花} \\
    Acanthaceae & 爵床科 & \textit{Hemigraphis reptans}  & 匍匐半插花 & D2 \index{Hemigraphis@\textit{Hemigraphis}!reptans@\textit{reptans}}  \index{匍匐半插花} \\
    Acanthaceae & 爵床科 & \textit{Justicia procumbens} var. \textit{hayatae}  & 早田氏爵床 & D2 \index{Justicia@\textit{Justicia}!procumbens@\textit{procumbens}!var. hayatae@var. \textit{hayatae}}  \index{早田氏爵床} \\
    Acanthaceae & 爵床科 & \textit{Justicia procumbens} var. \textit{hirsuta}  & 密毛爵床 & D2 \index{Justicia@\textit{Justicia}!procumbens@\textit{procumbens}!var. hirsuta@var. \textit{hirsuta}}  \index{密毛爵床} \\
    Acanthaceae & 爵床科 & \textit{Justicia procumbens} var. \textit{linearifolia}  & 狹葉爵床 & D2 \index{Justicia@\textit{Justicia}!procumbens@\textit{procumbens}!var. linearifolia@var. \textit{linearifolia}}  \index{狹葉爵床} \\
    Acanthaceae & 爵床科 & \textit{Justicia quadrifaria}  & 花蓮爵床 & D2 \index{Justicia@\textit{Justicia}!quadrifaria@\textit{quadrifaria}}  \index{花蓮爵床} \\
    Acanthaceae & 爵床科 & \textit{Kudoacanthus albonervosa}  & 銀脈爵床 & D2 \index{Kudoacanthus@\textit{Kudoacanthus}!albonervosa@\textit{albonervosa}}  \index{銀脈爵床} \\
    Acanthaceae & 爵床科 & \textit{Staurogyne debilis}  & 菲律賓哈哼花 & D1 \index{Staurogyne@\textit{Staurogyne}!debilis@\textit{debilis}}  \index{菲律賓哈哼花} \\
    Actinidiaceae & 獼猴桃科 & \textit{Actinidia arguta}  & 軟棗獼猴桃 & B1ab(v); C2a(i); D1 \index{Actinidia@\textit{Actinidia}!arguta@\textit{arguta}}  \index{軟棗獼猴桃} \\
    Alismataceae & 澤瀉科 & \textit{Alisma canaliculatum}  & 澤瀉 & D1 \index{Alisma@\textit{Alisma}!canaliculatum@\textit{canaliculatum}}  \index{澤瀉} \\
    Amaranthaceae & 莧科 & \textit{Philoxerus wrightii}  & 安旱草 & D1+2 \index{Philoxerus@\textit{Philoxerus}!wrightii@\textit{wrightii}}  \index{安旱草} \\
    Anacardiaceae & 漆樹科 & \textit{Rhus hypoleuca}  & 裏白漆 & D1 \index{Rhus@\textit{Rhus}!hypoleuca@\textit{hypoleuca}}  \index{裏白漆} \\
    Apiaceae & 繖形科 & \textit{Angelica dahurica} var. \textit{formosana}  & 野當歸 & D1 \index{Angelica@\textit{Angelica}!dahurica@\textit{dahurica}!var. formosana@var. \textit{formosana}}  \index{野當歸} \\
    Apiaceae & 繖形科 & \textit{Angelica hirsutiflora}  & 濱當歸 & D1 \index{Angelica@\textit{Angelica}!hirsutiflora@\textit{hirsutiflora}}  \index{濱當歸} \\
    Apiaceae & 繖形科 & \textit{Chaerophyllum taiwanianum}  & 臺灣山薰香 & D2 \index{Chaerophyllum@\textit{Chaerophyllum}!taiwanianum@\textit{taiwanianum}}  \index{臺灣山薰香} \\
    Apocynaceae & 夾竹桃科 & \textit{Jasminanthes mucronata}  & 舌瓣花 & D1+2 \index{Jasminanthes@\textit{Jasminanthes}!mucronata@\textit{mucronata}}  \index{舌瓣花} \\
    Aquifoliaceae & 冬青科 & \textit{Ilex integra}  & 全緣葉冬青 & D1 \index{Ilex@\textit{Ilex}!integra@\textit{integra}}  \index{全緣葉冬青} \\
    Araceae & 天南星科 & \textit{Arisaema ilanense}  & 宜蘭天南星 & D1 \index{Arisaema@\textit{Arisaema}!ilanense@\textit{ilanense}}  \index{宜蘭天南星} \\
    Araceae & 天南星科 & \textit{Arisaema matsudae}  & 線花天南星 & D1 \index{Arisaema@\textit{Arisaema}!matsudae@\textit{matsudae}}  \index{線花天南星} \\
    Araceae & 天南星科 & \textit{Arisaema nanjenense}  & 南仁山天南星 & D1 \index{Arisaema@\textit{Arisaema}!nanjenense@\textit{nanjenense}}  \index{南仁山天南星} \\
    Araceae & 天南星科 & \textit{Arisaema taiwanense} var. \textit{brevipedunculatum}  & 短梗天南星 & D1 \index{Arisaema@\textit{Arisaema}!taiwanense@\textit{taiwanense}!var. brevipedunculatum@var. \textit{brevipedunculatum}}  \index{短梗天南星} \\
    Araceae & 天南星科 & \textit{Remusatia vivipara}  & 臺灣目賊芋 & D1 \index{Remusatia@\textit{Remusatia}!vivipara@\textit{vivipara}}  \index{臺灣目賊芋} \\
    Araceae & 天南星科 & \textit{Remusatia yunnanensis}  & 雲南岩芋 & B2;C2a;D1 \index{Remusatia@\textit{Remusatia}!yunnanensis@\textit{yunnanensis}}  \index{雲南岩芋} \\
    Araceae & 天南星科 & \textit{Rhaphidophora liukiuensis}  & 針房藤 & D2 \index{Rhaphidophora@\textit{Rhaphidophora}!liukiuensis@\textit{liukiuensis}}  \index{針房藤} \\
    Araceae & 天南星科 & \textit{Schismatoglottis kotoensis}  & 蘭嶼芋 & D2 \index{Schismatoglottis@\textit{Schismatoglottis}!kotoensis@\textit{kotoensis}}  \index{蘭嶼芋} \\
    Araliaceae & 五加科 & \textit{Schefflera odorata}  & 鵝掌藤 & A2; D1 \index{Schefflera@\textit{Schefflera}!odorata@\textit{odorata}}  \index{鵝掌藤} \\
    Araliaceae & 五加科 & \textit{Sinopanax formosana}  & 華參 & C2a(i) \index{Sinopanax@\textit{Sinopanax}!formosana@\textit{formosana}}  \index{華參} \\
    Arecaceae & 棕櫚科 & \textit{Livistona chinensis} var. \textit{subglobosa}  & 蒲葵 & D1+2 \index{Livistona@\textit{Livistona}!chinensis@\textit{chinensis}!var. subglobosa@var. \textit{subglobosa}}  \index{蒲葵} \\
    Aristolochiaceae & 馬兜鈴科 & \textit{Aristolochia cucurbitifolia}  & 瓜葉馬兜鈴 & D1+2 \index{Aristolochia@\textit{Aristolochia}!cucurbitifolia@\textit{cucurbitifolia}}  \index{瓜葉馬兜鈴} \\
    Aristolochiaceae & 馬兜鈴科 & \textit{Asarum crassusepalum}  & 鴛鴦湖細辛 & D1+2 \index{Asarum@\textit{Asarum}!crassusepalum@\textit{crassusepalum}}  \index{鴛鴦湖細辛} \\
    Aristolochiaceae & 馬兜鈴科 & \textit{Asarum epigynum}  & 上花細辛 & D1+2 \index{Asarum@\textit{Asarum}!epigynum@\textit{epigynum}}  \index{上花細辛} \\
    Aristolochiaceae & 馬兜鈴科 & \textit{Asarum hypogynum}  & 下花細辛 & D1 \index{Asarum@\textit{Asarum}!hypogynum@\textit{hypogynum}}  \index{下花細辛} \\
    Aristolochiaceae & 馬兜鈴科 & \textit{Asarum taipingshanianum}  & 太平山細辛 & D1+2 \index{Asarum@\textit{Asarum}!taipingshanianum@\textit{taipingshanianum}}  \index{太平山細辛} \\
    Asparagaceae & 天門冬科 & \textit{Aspidistra mushaensis}  & 霧社蜘蛛抱蛋 & D1 \index{Aspidistra@\textit{Aspidistra}!mushaensis@\textit{mushaensis}}  \index{霧社蜘蛛抱蛋} \\
    Asparagaceae & 天門冬科 & \textit{Barnardia japonica}  & 綿棗兒 & D1+2 \index{Barnardia@\textit{Barnardia}!japonica@\textit{japonica}}  \index{綿棗兒} \\
    Asparagaceae & 天門冬科 & \textit{Dracaena angustifolia}  & 番仔林投 & D2 \index{Dracaena@\textit{Dracaena}!angustifolia@\textit{angustifolia}}  \index{番仔林投} \\
    Asparagaceae & 天門冬科 & \textit{Rohdea japonica} var. \textit{watanabei}  & 萬年青 & D1 \index{Rohdea@\textit{Rohdea}!japonica@\textit{japonica}!var. watanabei@var. \textit{watanabei}}  \index{萬年青} \\
    Asteraceae & 菊科 & \textit{Acmella paniculata}  & 金鈕釦 & D1 \index{Acmella@\textit{Acmella}!paniculata@\textit{paniculata}}  \index{金鈕釦} \\
    Asteraceae & 菊科 & \textit{Ainsliaea secundiflora}  & 中原氏鬼督郵 & B2ab(ii,iii,v); D1+2 \index{Ainsliaea@\textit{Ainsliaea}!secundiflora@\textit{secundiflora}}  \index{中原氏鬼督郵} \\
    Asteraceae & 菊科 & \textit{Aster shimadai}  & 島田氏雞兒腸 & D1+2 \index{Aster@\textit{Aster}!shimadai@\textit{shimadai}}  \index{島田氏雞兒腸} \\
    Asteraceae & 菊科 & \textit{Blumea conspicua}  & 大花艾納香 & D1+2 \index{Blumea@\textit{Blumea}!conspicua@\textit{conspicua}}  \index{大花艾納香} \\
    Asteraceae & 菊科 & \textit{Blumea linearis}  & 狹葉艾納香 & D1+2 \index{Blumea@\textit{Blumea}!linearis@\textit{linearis}}  \index{狹葉艾納香} \\
    Asteraceae & 菊科 & \textit{Crossostephium chinense}  & 蘄艾 & B2ab(v) \index{Crossostephium@\textit{Crossostephium}!chinense@\textit{chinense}}  \index{蘄艾} \\
    Asteraceae & 菊科 & \textit{Eupatorium clematideum} var. \textit{gracillimum}  & 高士佛澤蘭 & D1+2 \index{Eupatorium@\textit{Eupatorium}!clematideum@\textit{clematideum}!var. gracillimum@var. \textit{gracillimum}}  \index{高士佛澤蘭} \\
    Asteraceae & 菊科 & \textit{Glossocardia bidens}  & 香茹 & C2b \index{Glossocardia@\textit{Glossocardia}!bidens@\textit{bidens}}  \index{香茹} \\
    Asteraceae & 菊科 & \textit{Gynura elliptica}  & 蘭嶼木耳菜 & D2 \index{Gynura@\textit{Gynura}!elliptica@\textit{elliptica}}  \index{蘭嶼木耳菜} \\
    Asteraceae & 菊科 & \textit{Ligularia japonica}  & 大吳風草 & D1+2 \index{Ligularia@\textit{Ligularia}!japonica@\textit{japonica}}  \index{大吳風草} \\
    Asteraceae & 菊科 & \textit{Ligularia kojimae}  & 高山橐吾 & C2a(i) \index{Ligularia@\textit{Ligularia}!kojimae@\textit{kojimae}}  \index{高山橐吾} \\
    Asteraceae & 菊科 & \textit{Nemosenecio formosanus}  & 臺灣劉寄奴 & D1 \index{Nemosenecio@\textit{Nemosenecio}!formosanus@\textit{formosanus}}  \index{臺灣劉寄奴} \\
    Asteraceae & 菊科 & \textit{Parasenecio hwangshanica}  & 黃山蟹甲草 & D1 \index{Parasenecio@\textit{Parasenecio}!hwangshanica@\textit{hwangshanica}}  \index{黃山蟹甲草} \\
    Asteraceae & 菊科 & \textit{Parasenecio monantha}  & 玉山蟹甲草 & D1 \index{Parasenecio@\textit{Parasenecio}!monantha@\textit{monantha}}  \index{玉山蟹甲草} \\
    Asteraceae & 菊科 & \textit{Pluchea pteropoda}  & 光梗闊苞菊 & D1 \index{Pluchea@\textit{Pluchea}!pteropoda@\textit{pteropoda}}  \index{光梗闊苞菊} \\
    Asteraceae & 菊科 & \textit{Saussurea glandulosa}  & 高山青木香 & D2 \index{Saussurea@\textit{Saussurea}!glandulosa@\textit{glandulosa}}  \index{高山青木香} \\
    Asteraceae & 菊科 & \textit{Saussurea kanzanensis}  & 關山青木香 & B; D2 \index{Saussurea@\textit{Saussurea}!kanzanensis@\textit{kanzanensis}}  \index{關山青木香} \\
    Asteraceae & 菊科 & \textit{Saussurea kiraisiensis}  & 奇萊青木香 & B; D2 \index{Saussurea@\textit{Saussurea}!kiraisiensis@\textit{kiraisiensis}}  \index{奇萊青木香} \\
    Asteraceae & 菊科 & \textit{Senecio crataegifolius}  & 小蔓黃菀 & B2ab(i, ii) \index{Senecio@\textit{Senecio}!crataegifolius@\textit{crataegifolius}}  \index{小蔓黃菀} \\
    Asteraceae & 菊科 & \textit{Vernonia maritima}  & 濱斑鳩菊 & D2 \index{Vernonia@\textit{Vernonia}!maritima@\textit{maritima}}  \index{濱斑鳩菊} \\
    Balanophoraceae & 蛇菰科 & \textit{Balanophora fungosa}  & 粗穗蛇菰 & D2 \index{Balanophora@\textit{Balanophora}!fungosa@\textit{fungosa}}  \index{粗穗蛇菰} \\
    Balanophoraceae & 蛇菰科 & \textit{Balanophora wrightii}  & 海桐生蛇菰 & D2 \index{Balanophora@\textit{Balanophora}!wrightii@\textit{wrightii}}  \index{海桐生蛇菰} \\
    Balsaminaceae & 鳳仙花科 & \textit{Impatiens devolii}  & 棣慕華鳳仙花 & D2 \index{Impatiens@\textit{Impatiens}!devolii@\textit{devolii}}  \index{棣慕華鳳仙花} \\
    Balsaminaceae & 鳳仙花科 & \textit{Impatiens tayemonii}  & 黃花鳳仙花 & B2b(ii,iii)c(iv) \index{Impatiens@\textit{Impatiens}!tayemonii@\textit{tayemonii}}  \index{黃花鳳仙花} \\
    Begoniaceae & 秋海棠科 & \textit{Begonia austrotaiwanensis}  & 南臺灣秋海棠 & D2 \index{Begonia@\textit{Begonia}!austrotaiwanensis@\textit{austrotaiwanensis}}  \index{南臺灣秋海棠} \\
    Begoniaceae & 秋海棠科 & \textit{Begonia chuyunshanensis}  & 出雲山秋海棠 & D1 \index{Begonia@\textit{Begonia}!chuyunshanensis@\textit{chuyunshanensis}}  \index{出雲山秋海棠} \\
    Begoniaceae & 秋海棠科 & \textit{Begonia lukuana}  & 鹿谷秋海棠 & D2 \index{Begonia@\textit{Begonia}!lukuana@\textit{lukuana}}  \index{鹿谷秋海棠} \\
    Begoniaceae & 秋海棠科 & \textit{Begonia tengchiana}  & 藤枝秋海棠 & D1+2 \index{Begonia@\textit{Begonia}!tengchiana@\textit{tengchiana}}  \index{藤枝秋海棠} \\
    Begoniaceae & 秋海棠科 & \textit{Begonia wutaiana}  & 霧臺秋海棠 & D1+2 \index{Begonia@\textit{Begonia}!wutaiana@\textit{wutaiana}}  \index{霧臺秋海棠} \\
    Berberidaceae & 小檗科 & \textit{Berberis nantoensis}  & 南投小檗 & B1b(iii)c(iii) \index{Berberis@\textit{Berberis}!nantoensis@\textit{nantoensis}}  \index{南投小檗} \\
    Berberidaceae & 小檗科 & \textit{Mahonia japonica}  & 十大功勞 & B2ab(i); C1 \index{Mahonia@\textit{Mahonia}!japonica@\textit{japonica}}  \index{十大功勞} \\
    Berberidaceae & 小檗科 & \textit{Mahonia oiwakensis}  & 阿里山十大功勞 & B2ac(iii) \index{Mahonia@\textit{Mahonia}!oiwakensis@\textit{oiwakensis}}  \index{阿里山十大功勞} \\
    Boraginaceae & 紫草科 & \textit{Lithospermum zollingeri}  & 梓木草 & D2 \index{Lithospermum@\textit{Lithospermum}!zollingeri@\textit{zollingeri}}  \index{梓木草} \\
    Burmanniaceae & 水玉簪科 & \textit{Burmannia cryptopetala}  & 透明水玉簪 & D1 \index{Burmannia@\textit{Burmannia}!cryptopetala@\textit{cryptopetala}}  \index{透明水玉簪} \\
    Burmanniaceae & 水玉簪科 & \textit{Burmannia liukiuensis}  & 琉球水玉簪 & D1 \index{Burmannia@\textit{Burmannia}!liukiuensis@\textit{liukiuensis}}  \index{琉球水玉簪} \\
    Buxaceae & 黃楊科 & \textit{Buxus microphylla} subsp. \textit{sinica} var. \textit{tarokoensis}  & 太魯閣黃楊 & D1 \index{Buxus@\textit{Buxus}!microphylla@\textit{microphylla}!subsp. sinica@subsp. \textit{sinica}!var. tarokoensis@var. \textit{tarokoensis}}  \index{太魯閣黃楊} \\
    Cannabaceae & 大麻科 & \textit{Celtis philippensis}  & 菲律賓朴樹 & D1+2 \index{Celtis@\textit{Celtis}!philippensis@\textit{philippensis}}  \index{菲律賓朴樹} \\
    Capparaceae & 山柑科 & \textit{Capparis acutifolia}  & 銳葉山柑 & D1 \index{Capparis@\textit{Capparis}!acutifolia@\textit{acutifolia}}  \index{銳葉山柑} \\
    Capparaceae & 山柑科 & \textit{Capparis floribunda}  & 多花山柑 & D1 \index{Capparis@\textit{Capparis}!floribunda@\textit{floribunda}}  \index{多花山柑} \\
    Caprifoliaceae & 忍冬科 & \textit{Abelia chinensis} var. \textit{ionandra}  & 臺灣糯米條 & A3, D1 \index{Abelia@\textit{Abelia}!chinensis@\textit{chinensis}!var. ionandra@var. \textit{ionandra}}  \index{臺灣糯米條} \\
    Caprifoliaceae & 忍冬科 & \textit{Lonicera kawakamii}  & 川上氏忍冬 & D2 \index{Lonicera@\textit{Lonicera}!kawakamii@\textit{kawakamii}}  \index{川上氏忍冬} \\
    Caprifoliaceae & 忍冬科 & \textit{Lonicera macrantha}  & 大花忍冬 & D2 \index{Lonicera@\textit{Lonicera}!macrantha@\textit{macrantha}}  \index{大花忍冬} \\
    Caprifoliaceae & 忍冬科 & \textit{Lonicera oiwakensis}  & 追分忍冬 & D1 \index{Lonicera@\textit{Lonicera}!oiwakensis@\textit{oiwakensis}}  \index{追分忍冬} \\
    Caryophyllaceae & 石竹科 & \textit{Dianthus palinensis}  & 巴陵石竹 & D1+2 \index{Dianthus@\textit{Dianthus}!palinensis@\textit{palinensis}}  \index{巴陵石竹} \\
    Caryophyllaceae & 石竹科 & \textit{Dianthus pygmaeus}  & 玉山石竹 & B2ab(i,v) \index{Dianthus@\textit{Dianthus}!pygmaeus@\textit{pygmaeus}}  \index{玉山石竹} \\
    Caryophyllaceae & 石竹科 & \textit{Silene glabella}  & 南湖大山蠅子草 & D1 \index{Silene@\textit{Silene}!glabella@\textit{glabella}}  \index{南湖大山蠅子草} \\
    Caryophyllaceae & 石竹科 & \textit{Silene morrisonmontana}  & 玉山蠅子草 & D1 \index{Silene@\textit{Silene}!morrisonmontana@\textit{morrisonmontana}}  \index{玉山蠅子草} \\
    Commelinaceae & 鴨跖草科 & \textit{Cyanotis axillaris}  & 鞘苞花 & D1 \index{Cyanotis@\textit{Cyanotis}!axillaris@\textit{axillaris}}  \index{鞘苞花} \\
    Commelinaceae & 鴨跖草科 & \textit{Murdannia spirata}  & 矮水竹葉 & D1 \index{Murdannia@\textit{Murdannia}!spirata@\textit{spirata}}  \index{矮水竹葉} \\
    Convolvulaceae & 旋花科 & \textit{Argyreia formosana}  & 鈍葉朝顏 & D1 \index{Argyreia@\textit{Argyreia}!formosana@\textit{formosana}}  \index{鈍葉朝顏} \\
    Convolvulaceae & 旋花科 & \textit{Ipomoea imperati}  & 厚葉牽牛 & D1 \index{Ipomoea@\textit{Ipomoea}!imperati@\textit{imperati}}  \index{厚葉牽牛} \\
    Convolvulaceae & 旋花科 & \textit{Merremia hirta}  & 變葉姬旋花 & B2ac(iv) \index{Merremia@\textit{Merremia}!hirta@\textit{hirta}}  \index{變葉姬旋花} \\
    Crassulaceae & 景天科 & \textit{Kalanchoe garambiensis}  & 鵝鑾鼻燈籠草 & B2ab(ii, v); D2 \index{Kalanchoe@\textit{Kalanchoe}!garambiensis@\textit{garambiensis}}  \index{鵝鑾鼻燈籠草} \\
    Crassulaceae & 景天科 & \textit{Sedum microsepalum}  & 小萼佛甲草 & D1 \index{Sedum@\textit{Sedum}!microsepalum@\textit{microsepalum}}  \index{小萼佛甲草} \\
    Crassulaceae & 景天科 & \textit{Sedum sekiteiense}  & 石碇佛甲草 & D1 \index{Sedum@\textit{Sedum}!sekiteiense@\textit{sekiteiense}}  \index{石碇佛甲草} \\
    Cucurbitaceae & 葫蘆科 & \textit{Siraitia taiwaniana}  & 臺灣羅漢果 & D2 \index{Siraitia@\textit{Siraitia}!taiwaniana@\textit{taiwaniana}}  \index{臺灣羅漢果} \\
    Cucurbitaceae & 葫蘆科 & \textit{Trichosanthes quinquangulata}  & 蘭嶼括樓 & D2 \index{Trichosanthes@\textit{Trichosanthes}!quinquangulata@\textit{quinquangulata}}  \index{蘭嶼括樓} \\
    Cyperaceae & 莎草科 & \textit{Carex echinata}  & 刺苞薹 & D1 \index{Carex@\textit{Carex}!echinata@\textit{echinata}}  \index{刺苞薹} \\
    Cyperaceae & 莎草科 & \textit{Carex grallatoria} var. \textit{heteroclita}  & 異型菱果薹 & D1 \index{Carex@\textit{Carex}!grallatoria@\textit{grallatoria}!var. heteroclita@var. \textit{heteroclita}}  \index{異型菱果薹} \\
    Cyperaceae & 莎草科 & \textit{Carex jisaburo-ohwiana}  & 大井氏扁果薹 & D1 \index{Carex@\textit{Carex}!jisaburo-ohwiana@\textit{jisaburo-ohwiana}}  \index{大井氏扁果薹} \\
    Cyperaceae & 莎草科 & \textit{Carex nemostachys}  & 毛囊果薹 & D2 \index{Carex@\textit{Carex}!nemostachys@\textit{nemostachys}}  \index{毛囊果薹} \\
    Cyperaceae & 莎草科 & \textit{Carex purpureotincta}  & 太魯閣薹 & D1+2 \index{Carex@\textit{Carex}!purpureotincta@\textit{purpureotincta}}  \index{太魯閣薹} \\
    Cyperaceae & 莎草科 & \textit{Cyperus platystylis}  & 寬柱莎草 & B2ab(iii,iv) \index{Cyperus@\textit{Cyperus}!platystylis@\textit{platystylis}}  \index{寬柱莎草} \\
    Cyperaceae & 莎草科 & \textit{Diplacrum caricinum}  & 裂穎茅 & D1+2 \index{Diplacrum@\textit{Diplacrum}!caricinum@\textit{caricinum}}  \index{裂穎茅} \\
    Cyperaceae & 莎草科 & \textit{Eleocharis acutangula}  & 桃園藺 & A1(a) \index{Eleocharis@\textit{Eleocharis}!acutangula@\textit{acutangula}}  \index{桃園藺} \\
    Cyperaceae & 莎草科 & \textit{Fimbristylis eragrostis}  & 紫穗飄拂草 & D2 \index{Fimbristylis@\textit{Fimbristylis}!eragrostis@\textit{eragrostis}}  \index{紫穗飄拂草} \\
    Cyperaceae & 莎草科 & \textit{Fimbristylis griffithii}  & 葛氏飄拂草 & D2 \index{Fimbristylis@\textit{Fimbristylis}!griffithii@\textit{griffithii}}  \index{葛氏飄拂草} \\
    Dioscoreaceae & 薯蕷科 & \textit{Dioscorea cumingii}  & 蘭嶼田薯 & D1+2 \index{Dioscorea@\textit{Dioscorea}!cumingii@\textit{cumingii}}  \index{蘭嶼田薯} \\
    Droseraceae & 茅膏菜科 & \textit{Drosera burmannii}  & 金錢草 & B2ac(ii) \index{Drosera@\textit{Drosera}!burmannii@\textit{burmannii}}  \index{金錢草} \\
    Ebenaceae & 柿樹科 & \textit{Diospyros ferrea}  & 象牙樹 & A4d, C1, D1 \index{Diospyros@\textit{Diospyros}!ferrea@\textit{ferrea}}  \index{象牙樹} \\
    Ebenaceae & 柿樹科 & \textit{Diospyros rhombifolia}  & 菱葉柿 & B2a, D1 \index{Diospyros@\textit{Diospyros}!rhombifolia@\textit{rhombifolia}}  \index{菱葉柿} \\
    Elaeagnaceae & 胡頹子科 & \textit{Elaeagnus tarokoensis}  & 太魯閣胡頹子 & D1 \index{Elaeagnus@\textit{Elaeagnus}!tarokoensis@\textit{tarokoensis}}  \index{太魯閣胡頹子} \\
    Elaeocarpaceae & 杜英科 & \textit{Elaeocarpus multiflorus}  & 繁花薯豆 & B2ab(ii,v); D1 \index{Elaeocarpus@\textit{Elaeocarpus}!multiflorus@\textit{multiflorus}}  \index{繁花薯豆} \\
    Elaeocarpaceae & 杜英科 & \textit{Elaeocarpus sylvestris} var. \textit{lanyuensis}  & 蘭嶼杜英 & D1 \index{Elaeocarpus@\textit{Elaeocarpus}!sylvestris@\textit{sylvestris}!var. lanyuensis@var. \textit{lanyuensis}}  \index{蘭嶼杜英} \\
    Ericaceae & 杜鵑花科 & \textit{Enkianthus perulatus}  & 臺灣吊鐘花 & A2, D1 \index{Enkianthus@\textit{Enkianthus}!perulatus@\textit{perulatus}}  \index{臺灣吊鐘花} \\
    Ericaceae & 杜鵑花科 & \textit{Monotropa hypopithys}  & 錫杖花 & D2 \index{Monotropa@\textit{Monotropa}!hypopithys@\textit{hypopithys}}  \index{錫杖花} \\
    Ericaceae & 杜鵑花科 & \textit{Rhododendron chilanshanense}  & 棲蘭山杜鵑 & A2a, D2 \index{Rhododendron@\textit{Rhododendron}!chilanshanense@\textit{chilanshanense}}  \index{棲蘭山杜鵑} \\
    Euphorbiaceae & 大戟科 & \textit{Alchornea trewioides} var. \textit{formosae}  & 臺灣山麻桿 & B2ab(iii,v); D1 \index{Alchornea@\textit{Alchornea}!trewioides@\textit{trewioides}!var. formosae@var. \textit{formosae}}  \index{臺灣山麻桿} \\
    Euphorbiaceae & 大戟科 & \textit{Chamaesyce garanbiensis}  & 鵝鑾鼻大戟 & D2 \index{Chamaesyce@\textit{Chamaesyce}!garanbiensis@\textit{garanbiensis}}  \index{鵝鑾鼻大戟} \\
    Euphorbiaceae & 大戟科 & \textit{Euphorbia jolkini}  & 岩大戟 & D1 \index{Euphorbia@\textit{Euphorbia}!jolkini@\textit{jolkini}}  \index{岩大戟} \\
    Euphorbiaceae & 大戟科 & \textit{Excoecaria kawakamii}  & 蘭嶼土沉香 & D1 \index{Excoecaria@\textit{Excoecaria}!kawakamii@\textit{kawakamii}}  \index{蘭嶼土沉香} \\
    Euphorbiaceae & 大戟科 & \textit{Gelonium aequoreum}  & 白樹仔 & B2ab(ii,v) \index{Gelonium@\textit{Gelonium}!aequoreum@\textit{aequoreum}}  \index{白樹仔} \\
    Fabaceae & 豆科 & \textit{Astragalus nokoensis}  & 能高大山紫雲英 & D1 \index{Astragalus@\textit{Astragalus}!nokoensis@\textit{nokoensis}}  \index{能高大山紫雲英} \\
    Fabaceae & 豆科 & \textit{Caesalpinia bonduc}  & 老虎心 & D1+2 \index{Caesalpinia@\textit{Caesalpinia}!bonduc@\textit{bonduc}}  \index{老虎心} \\
    Fabaceae & 豆科 & \textit{Cassia sophora} var. \textit{penghuana}  & 澎湖決明 & D2 \index{Cassia@\textit{Cassia}!sophora@\textit{sophora}!var. penghuana@var. \textit{penghuana}}  \index{澎湖決明} \\
    Fabaceae & 豆科 & \textit{Chamaecrista garambiensis}  & 鵝鑾鼻決明 & D2 \index{Chamaecrista@\textit{Chamaecrista}!garambiensis@\textit{garambiensis}}  \index{鵝鑾鼻決明} \\
    Fabaceae & 豆科 & \textit{Desmodium gracillimum}  & 細葉山螞蝗 & D2 \index{Desmodium@\textit{Desmodium}!gracillimum@\textit{gracillimum}}  \index{細葉山螞蝗} \\
    Fabaceae & 豆科 & \textit{Dolichos trilobus} var. \textit{kosyunensis}  & 三裂葉扁豆 & D2 \index{Dolichos@\textit{Dolichos}!trilobus@\textit{trilobus}!var. kosyunensis@var. \textit{kosyunensis}}  \index{三裂葉扁豆} \\
    Fabaceae & 豆科 & \textit{Dumasia miaoliensis}  & 苗栗野豇豆 & D2 \index{Dumasia@\textit{Dumasia}!miaoliensis@\textit{miaoliensis}}  \index{苗栗野豇豆} \\
    Fabaceae & 豆科 & \textit{Eriosema chinense}  & 豬仔笠 & D2 \index{Eriosema@\textit{Eriosema}!chinense@\textit{chinense}}  \index{豬仔笠} \\
    Fabaceae & 豆科 & \textit{Gleditsia rolfei}  & 恆春皂莢 & D1 \index{Gleditsia@\textit{Gleditsia}!rolfei@\textit{rolfei}}  \index{恆春皂莢} \\
    Fabaceae & 豆科 & \textit{Glycine dolichocarpa}  & 扁豆莢大豆 & D2 \index{Glycine@\textit{Glycine}!dolichocarpa@\textit{dolichocarpa}}  \index{扁豆莢大豆} \\
    Fabaceae & 豆科 & \textit{Glycine max} subsp. \textit{formosana}  & 臺灣大豆 & D2 \index{Glycine@\textit{Glycine}!max@\textit{max}!subsp. formosana@subsp. \textit{formosana}}  \index{臺灣大豆} \\
    Fabaceae & 豆科 & \textit{Glycine tabacina}  & 澎湖大豆 & D2 \index{Glycine@\textit{Glycine}!tabacina@\textit{tabacina}}  \index{澎湖大豆} \\
    Fabaceae & 豆科 & \textit{Indigofera zollingeriana}  & 蘭嶼木藍 & D2 \index{Indigofera@\textit{Indigofera}!zollingeriana@\textit{zollingeriana}}  \index{蘭嶼木藍} \\
    Fabaceae & 豆科 & \textit{Lotus pacifica}  & 蘭嶼百脈根 & D2 \index{Lotus@\textit{Lotus}!pacifica@\textit{pacifica}}  \index{蘭嶼百脈根} \\
    Fabaceae & 豆科 & \textit{Maackia taiwanensis}  & 臺灣馬鞍樹 & D2 \index{Maackia@\textit{Maackia}!taiwanensis@\textit{taiwanensis}}  \index{臺灣馬鞍樹} \\
    Fabaceae & 豆科 & \textit{Mucuna membranacea}  & 蘭嶼血藤 & D2 \index{Mucuna@\textit{Mucuna}!membranacea@\textit{membranacea}}  \index{蘭嶼血藤} \\
    Fabaceae & 豆科 & \textit{Ormocarpum cochinchinensis}  & 濱槐 & D2 \index{Ormocarpum@\textit{Ormocarpum}!cochinchinensis@\textit{cochinchinensis}}  \index{濱槐} \\
    Fabaceae & 豆科 & \textit{Ormosia formosana}  & 臺灣紅豆樹 & D2 \index{Ormosia@\textit{Ormosia}!formosana@\textit{formosana}}  \index{臺灣紅豆樹} \\
    Fabaceae & 豆科 & \textit{Ormosia hengchuniana}  & 恆春紅豆樹 & D2 \index{Ormosia@\textit{Ormosia}!hengchuniana@\textit{hengchuniana}}  \index{恆春紅豆樹} \\
    Fabaceae & 豆科 & \textit{Zornia intecta}  & 臺東葵草 & D2 \index{Zornia@\textit{Zornia}!intecta@\textit{intecta}}  \index{臺東葵草} \\
    Fagaceae & 殼斗科 & \textit{Castanopsis chinensis}  & 桂林栲 & D2 \index{Castanopsis@\textit{Castanopsis}!chinensis@\textit{chinensis}}  \index{桂林栲} \\
    Fagaceae & 殼斗科 & \textit{Lithocarpus dodonaeifolius}  & 柳葉石櫟 & D2 \index{Lithocarpus@\textit{Lithocarpus}!dodonaeifolius@\textit{dodonaeifolius}}  \index{柳葉石櫟} \\
    Fagaceae & 殼斗科 & \textit{Lithocarpus nantoensis}  & 南投石櫟 & D1 \index{Lithocarpus@\textit{Lithocarpus}!nantoensis@\textit{nantoensis}}  \index{南投石櫟} \\
    Fagaceae & 殼斗科 & \textit{Quercus repandifolia}  & 波葉櫟 & D1 \index{Quercus@\textit{Quercus}!repandifolia@\textit{repandifolia}}  \index{波葉櫟} \\
    Gentianaceae & 龍膽科 & \textit{Centaurium pulchellum} var. \textit{altaicum}  & 百金花 & B2ac(); D2 \index{Centaurium@\textit{Centaurium}!pulchellum@\textit{pulchellum}!var. altaicum@var. \textit{altaicum}}  \index{百金花} \\
    Gentianaceae & 龍膽科 & \textit{Fagraea ceilanica}  & 灰莉 & D1 \index{Fagraea@\textit{Fagraea}!ceilanica@\textit{ceilanica}}  \index{灰莉} \\
    Gentianaceae & 龍膽科 & \textit{Gentiana bambuseti}  & 竹林龍膽 & B1+2?; C2b \index{Gentiana@\textit{Gentiana}!bambuseti@\textit{bambuseti}}  \index{竹林龍膽} \\
    Gentianaceae & 龍膽科 & \textit{Gentiana tatakensis}  & 塔塔加龍膽 & D2 \index{Gentiana@\textit{Gentiana}!tatakensis@\textit{tatakensis}}  \index{塔塔加龍膽} \\
    Gentianaceae & 龍膽科 & \textit{Pterygocalyx volubilis}  & 翼萼蔓 & B2ac(iii) \index{Pterygocalyx@\textit{Pterygocalyx}!volubilis@\textit{volubilis}}  \index{翼萼蔓} \\
    Gentianaceae & 龍膽科 & \textit{Swertia arisanensis}  & 阿里山當藥 & B2ac(iv) \index{Swertia@\textit{Swertia}!arisanensis@\textit{arisanensis}}  \index{阿里山當藥} \\
    Gentianaceae & 龍膽科 & \textit{Swertia tozanensis}  & 高山當藥 & B2ac(iv) \index{Swertia@\textit{Swertia}!tozanensis@\textit{tozanensis}}  \index{高山當藥} \\
    Gentianaceae & 龍膽科 & \textit{Tripterospermum cordifolium}  & 高山肺形草 & B2ac(iii) \index{Tripterospermum@\textit{Tripterospermum}!cordifolium@\textit{cordifolium}}  \index{高山肺形草} \\
    Gentianaceae & 龍膽科 & \textit{Tripterospermum hualienense}  & 東臺肺形草 & B1 \index{Tripterospermum@\textit{Tripterospermum}!hualienense@\textit{hualienense}}  \index{東臺肺形草} \\
    Gesneriaceae & 苦苣苔科 & \textit{Chirita anachoreta}  & 雙心皮草 & D2 \index{Chirita@\textit{Chirita}!anachoreta@\textit{anachoreta}}  \index{雙心皮草} \\
    Gesneriaceae & 苦苣苔科 & \textit{Cyrtandra umbellifera}  & 雄胞囊草 & D1+2 \index{Cyrtandra@\textit{Cyrtandra}!umbellifera@\textit{umbellifera}}  \index{雄胞囊草} \\
    Gesneriaceae & 苦苣苔科 & \textit{Epithema taiwanense} var. \textit{fasciculatum}  & 密花苣苔 & D2 \index{Epithema@\textit{Epithema}!taiwanense@\textit{taiwanense}!var. fasciculatum@var. \textit{fasciculatum}}  \index{密花苣苔} \\
    Hamamelidaceae & 金縷梅科 & \textit{Distylium gracile}  & 細葉蚊母樹 & A4; D1 \index{Distylium@\textit{Distylium}!gracile@\textit{gracile}}  \index{細葉蚊母樹} \\
    Hernandiaceae & 蓮葉桐科 & \textit{Hernandia nymphiifolia}  & 蓮葉桐 & D1 \index{Hernandia@\textit{Hernandia}!nymphiifolia@\textit{nymphiifolia}}  \index{蓮葉桐} \\
    Hydrocharitaceae & 水鱉科 & \textit{Najas browniana}  & 高雄茨藻 & D1 \index{Najas@\textit{Najas}!browniana@\textit{browniana}}  \index{高雄茨藻} \\
    Hydroleaceae & 田基麻科 & \textit{Hydrolea zeylanica}  & 探芹草 & B2ab(iii), D2 \index{Hydrolea@\textit{Hydrolea}!zeylanica@\textit{zeylanica}}  \index{探芹草} \\
    Hypericaceae & 金絲桃科 & \textit{Hypericum nakamurai}  & 清水金絲桃 & D1 \index{Hypericum@\textit{Hypericum}!nakamurai@\textit{nakamurai}}  \index{清水金絲桃} \\
    Hypericaceae & 金絲桃科 & \textit{Hypericum subalatum}  & 方莖金絲桃 & A4; D1 \index{Hypericum@\textit{Hypericum}!subalatum@\textit{subalatum}}  \index{方莖金絲桃} \\
    Juncaceae & 燈心草科 & \textit{Juncus wallichianus}  & 小葉燈心草 & B2ac(ii,iii) \index{Juncus@\textit{Juncus}!wallichianus@\textit{wallichianus}}  \index{小葉燈心草} \\
    Lamiaceae & 唇形科 & \textit{Ajuga dictyocarpa}  & 網果筋骨草 & B2ab(iii) \index{Ajuga@\textit{Ajuga}!dictyocarpa@\textit{dictyocarpa}}  \index{網果筋骨草} \\
    Lamiaceae & 唇形科 & \textit{Ajuga nipponensis}  & 日本筋骨草 & D1+2 \index{Ajuga@\textit{Ajuga}!nipponensis@\textit{nipponensis}}  \index{日本筋骨草} \\
    Lamiaceae & 唇形科 & \textit{Ajuga pygmaea}  & 矮筋骨草 & D1+2 \index{Ajuga@\textit{Ajuga}!pygmaea@\textit{pygmaea}}  \index{矮筋骨草} \\
    Lamiaceae & 唇形科 & \textit{Basilicum polystachyon}  & 小冠薰 & D1+2 \index{Basilicum@\textit{Basilicum}!polystachyon@\textit{polystachyon}}  \index{小冠薰} \\
    Lamiaceae & 唇形科 & \textit{Callicarpa formosana} var. \textit{longifolia}  & 長葉杜虹花 & D1 \index{Callicarpa@\textit{Callicarpa}!formosana@\textit{formosana}!var. longifolia@var. \textit{longifolia}}  \index{長葉杜虹花} \\
    Lamiaceae & 唇形科 & \textit{Callicarpa hypoleucophylla}  & 灰背葉紫珠 & D1 \index{Callicarpa@\textit{Callicarpa}!hypoleucophylla@\textit{hypoleucophylla}}  \index{灰背葉紫珠} \\
    Lamiaceae & 唇形科 & \textit{Collinsonia macrobracteata}  & 大苞偏穗花 & D1+2 \index{Collinsonia@\textit{Collinsonia}!macrobracteata@\textit{macrobracteata}}  \index{大苞偏穗花} \\
    Lamiaceae & 唇形科 & \textit{Comanthosphace formosana}  & 臺灣白木草 & D1+2 \index{Comanthosphace@\textit{Comanthosphace}!formosana@\textit{formosana}}  \index{臺灣白木草} \\
    Lamiaceae & 唇形科 & \textit{Prunella vulgaris} subsp. \textit{asiatica} var. \textit{nanhutashanensis}  & 高山夏枯草 & D1+2 \index{Prunella@\textit{Prunella}!vulgaris@\textit{vulgaris}!subsp. asiatica@subsp. \textit{asiatica}!var. nanhutashanensis@var. \textit{nanhutashanensis}}  \index{高山夏枯草} \\
    Lamiaceae & 唇形科 & \textit{Salvia hayatana}  & 早田氏鼠尾草 & D2  \index{Salvia@\textit{Salvia}!hayatana@\textit{hayatana}}  \index{早田氏鼠尾草} \\
    Lamiaceae & 唇形科 & \textit{Scutellaria austrotaiwanensis}  & 南臺灣黃芩 & D1+2 \index{Scutellaria@\textit{Scutellaria}!austrotaiwanensis@\textit{austrotaiwanensis}}  \index{南臺灣黃芩} \\
    Lamiaceae & 唇形科 & \textit{Scutellaria playfairii}  & 布烈氏黃芩 & D1 \index{Scutellaria@\textit{Scutellaria}!playfairii@\textit{playfairii}}  \index{布烈氏黃芩} \\
    Lamiaceae & 唇形科 & \textit{Scutellaria taiwanensis}  & 臺灣黃芩 & D1+2 \index{Scutellaria@\textit{Scutellaria}!taiwanensis@\textit{taiwanensis}}  \index{臺灣黃芩} \\
    Lamiaceae & 唇形科 & \textit{Scutellaria tashiroi}  & 田代氏黃芩 & D1 \index{Scutellaria@\textit{Scutellaria}!tashiroi@\textit{tashiroi}}  \index{田代氏黃芩} \\
    Lamiaceae & 唇形科 & \textit{Suzukia luchuensis}  & 琉球鈴木草 & D1+2 \index{Suzukia@\textit{Suzukia}!luchuensis@\textit{luchuensis}}  \index{琉球鈴木草} \\
    Lamiaceae & 唇形科 & \textit{Teucrium taiwanianum}  & 臺灣香科科 & D1 \index{Teucrium@\textit{Teucrium}!taiwanianum@\textit{taiwanianum}}  \index{臺灣香科科} \\
    Lamiaceae & 唇形科 & \textit{Vitex trifolia}  & 三葉埔姜 & D1 \index{Vitex@\textit{Vitex}!trifolia@\textit{trifolia}}  \index{三葉埔姜} \\
    Lauraceae & 樟科 & \textit{Cinnamomum brevipedunculatum}  & 小葉樟 & C2a(i) \index{Cinnamomum@\textit{Cinnamomum}!brevipedunculatum@\textit{brevipedunculatum}}  \index{小葉樟} \\
    Lauraceae & 樟科 & \textit{Neolitsea hiiranensis}  & 南仁山新木薑子 & D1 \index{Neolitsea@\textit{Neolitsea}!hiiranensis@\textit{hiiranensis}}  \index{南仁山新木薑子} \\
    Lecythidaceae & 玉蕊科 & \textit{Barringtonia asiatica}  & 棋盤腳樹 & D1 \index{Barringtonia@\textit{Barringtonia}!asiatica@\textit{asiatica}}  \index{棋盤腳樹} \\
    Lecythidaceae & 玉蕊科 & \textit{Barringtonia racemosa}  & 水茄苳 & B2a; D1 \index{Barringtonia@\textit{Barringtonia}!racemosa@\textit{racemosa}}  \index{水茄苳} \\
    Lentibulariaceae & 狸藻科 & \textit{Utricularia gibba}  & 絲葉狸藻 & B2b(iii)c(ii) \index{Utricularia@\textit{Utricularia}!gibba@\textit{gibba}}  \index{絲葉狸藻} \\
    Liliaceae & 百合科 & \textit{Tricyrtis suzukii}  & 鈴木氏油點草 & D1+2 \index{Tricyrtis@\textit{Tricyrtis}!suzukii@\textit{suzukii}}  \index{鈴木氏油點草} \\
    Linderniaceae & 母草科 & \textit{Legazpia polygonoides}  & 三翅萼 & D1+2 \index{Legazpia@\textit{Legazpia}!polygonoides@\textit{polygonoides}}  \index{三翅萼} \\
    Linderniaceae & 母草科 & \textit{Lindernia tenuifolia}  & 薄葉見風紅 & B2ab(iii); D1 \index{Lindernia@\textit{Lindernia}!tenuifolia@\textit{tenuifolia}}  \index{薄葉見風紅} \\
    Linderniaceae & 母草科 & \textit{Torenia benthamiana}  & 毛葉蝴蝶草 & D2 \index{Torenia@\textit{Torenia}!benthamiana@\textit{benthamiana}}  \index{毛葉蝴蝶草} \\
    Loganiaceae & 馬錢科 & \textit{Geniostema rupestre}  & 偽木荔枝 & D1 \index{Geniostema@\textit{Geniostema}!rupestre@\textit{rupestre}}  \index{偽木荔枝} \\
    Loranthaceae & 桑寄生科 & \textit{Taxillus limprichtii} var. \textit{longiflorus}  & 亮葉木蘭寄生 & D1 \index{Taxillus@\textit{Taxillus}!limprichtii@\textit{limprichtii}!var. longiflorus@var. \textit{longiflorus}}  \index{亮葉木蘭寄生} \\
    Lythraceae & 千屈菜科 & \textit{Trapa incisa}  & 小果菱 & B1ab(i,iii)+2ab(i,iii) \index{Trapa@\textit{Trapa}!incisa@\textit{incisa}}  \index{小果菱} \\
    Magnoliaceae & 木蘭科 & \textit{Magnolia kachirachirai}  & 烏心石舅 & B1ab(i,ii,v); D1 \index{Magnolia@\textit{Magnolia}!kachirachirai@\textit{kachirachirai}}  \index{烏心石舅} \\
    Malvaceae & 錦葵科 & \textit{Grewia eriocarpa}  & 大葉捕魚木 & A4, D1 \index{Grewia@\textit{Grewia}!eriocarpa@\textit{eriocarpa}}  \index{大葉捕魚木} \\
    Malvaceae & 錦葵科 & \textit{Pterospermum niveum}  & 翅子樹 & D1 \index{Pterospermum@\textit{Pterospermum}!niveum@\textit{niveum}}  \index{翅子樹} \\
    Marantaceae & 竹芋科 & \textit{Donax canniformis}  & 蘭嶼竹芋 & D1 \index{Donax@\textit{Donax}!canniformis@\textit{canniformis}}  \index{蘭嶼竹芋} \\
    Melastomataceae & 野牡丹科 & \textit{Bredia dulanica}  & 都蘭山金石榴 & D1 \index{Bredia@\textit{Bredia}!dulanica@\textit{dulanica}}  \index{都蘭山金石榴} \\
    Melastomataceae & 野牡丹科 & \textit{Medinilla formosana}  & 臺灣野牡丹藤 & B1ab(iv); C2a(i) \index{Medinilla@\textit{Medinilla}!formosana@\textit{formosana}}  \index{臺灣野牡丹藤} \\
    Melastomataceae & 野牡丹科 & \textit{Melastoma intermedia}  & 水社野牡丹 & D1 \index{Melastoma@\textit{Melastoma}!intermedia@\textit{intermedia}}  \index{水社野牡丹} \\
    Melastomataceae & 野牡丹科 & \textit{Memecylon lanceolatum}  & 革葉羊角扭 & C2a(i) \index{Memecylon@\textit{Memecylon}!lanceolatum@\textit{lanceolatum}}  \index{革葉羊角扭} \\
    Meliaceae & 楝科 & \textit{Aglaia chittagonga}  & 蘭嶼樹蘭 & D1 \index{Aglaia@\textit{Aglaia}!chittagonga@\textit{chittagonga}}  \index{蘭嶼樹蘭} \\
    Meliaceae & 楝科 & \textit{Aphanamixis polystachya}  & 穗花樹蘭 & D1 \index{Aphanamixis@\textit{Aphanamixis}!polystachya@\textit{polystachya}}  \index{穗花樹蘭} \\
    Meliaceae & 楝科 & \textit{Chisocheton patens}  & 蘭嶼擬樫木 & D1+2 \index{Chisocheton@\textit{Chisocheton}!patens@\textit{patens}}  \index{蘭嶼擬樫木} \\
    Meliaceae & 楝科 & \textit{Dysoxylum arborescens}  & 小葉樫木 & D1+2 \index{Dysoxylum@\textit{Dysoxylum}!arborescens@\textit{arborescens}}  \index{小葉樫木} \\
    Menispermaceae & 防己科 & \textit{Cocculus laurifolius}  & 樟葉木防己 & D1 \index{Cocculus@\textit{Cocculus}!laurifolius@\textit{laurifolius}}  \index{樟葉木防己} \\
    Menispermaceae & 防己科 & \textit{Stephania tetrandra}  & 石蟾蜍 & A3, D1 \index{Stephania@\textit{Stephania}!tetrandra@\textit{tetrandra}}  \index{石蟾蜍} \\
    Menyanthaceae & 睡菜科 & \textit{Nymphoides coreana}  & 小莕菜 & B2ab(iii,iv)c(ii,iii) \index{Nymphoides@\textit{Nymphoides}!coreana@\textit{coreana}}  \index{小莕菜} \\
    Mitrastemonaceae & 奴草科 & \textit{Mitrastemon yamamotoi} var. \textit{yamamotoi}  & 臺灣奴草 & B2ac(iii) \index{Mitrastemon@\textit{Mitrastemon}!yamamotoi@\textit{yamamotoi}!var. yamamotoi@var. \textit{yamamotoi}}  \index{臺灣奴草} \\
    Moraceae & 桑科 & \textit{Artocarpus xanthocarpus}  & 蘭嶼麵包樹 & D1+2 \index{Artocarpus@\textit{Artocarpus}!xanthocarpus@\textit{xanthocarpus}}  \index{蘭嶼麵包樹} \\
    Moraceae & 桑科 & \textit{Ficus esquiroliana}  & 黃毛榕 & D1 \index{Ficus@\textit{Ficus}!esquiroliana@\textit{esquiroliana}}  \index{黃毛榕} \\
    Moraceae & 桑科 & \textit{Ficus heteropleura}  & 尖尾長葉榕 & D1+2 \index{Ficus@\textit{Ficus}!heteropleura@\textit{heteropleura}}  \index{尖尾長葉榕} \\
    Moraceae & 桑科 & \textit{Ficus pedunculosa}  & 蔓榕 & D1 \index{Ficus@\textit{Ficus}!pedunculosa@\textit{pedunculosa}}  \index{蔓榕} \\
    Moraceae & 桑科 & \textit{Ficus pedunculosa} var. \textit{mearnsii}  & 鵝鑾鼻蔓榕 & D1 \index{Ficus@\textit{Ficus}!pedunculosa@\textit{pedunculosa}!var. mearnsii@var. \textit{mearnsii}}  \index{鵝鑾鼻蔓榕} \\
    Myristicaceae & 肉豆蔻科 & \textit{Myristica ceylanica} var. \textit{cagayanensis}  & 蘭嶼肉荳蔻 & D1 \index{Myristica@\textit{Myristica}!ceylanica@\textit{ceylanica}!var. cagayanensis@var. \textit{cagayanensis}}  \index{蘭嶼肉荳蔻} \\
    Myrtaceae & 桃金孃科 & \textit{Syzygium paucivenium}  & 疏脈赤楠 & D1+2 \index{Syzygium@\textit{Syzygium}!paucivenium@\textit{paucivenium}}  \index{疏脈赤楠} \\
    Myrtaceae & 桃金孃科 & \textit{Syzygium taiwanicum}  & 臺灣棒花蒲桃 & D1+2 \index{Syzygium@\textit{Syzygium}!taiwanicum@\textit{taiwanicum}}  \index{臺灣棒花蒲桃} \\
    Onagraceae & 柳葉菜科 & \textit{Epilobium nankotaizanense}  & 南湖柳葉菜 & D2 \index{Epilobium@\textit{Epilobium}!nankotaizanense@\textit{nankotaizanense}}  \index{南湖柳葉菜} \\
    Onagraceae & 柳葉菜科 & \textit{Epilobium pengii}  & 彭氏柳葉菜 & D2 \index{Epilobium@\textit{Epilobium}!pengii@\textit{pengii}}  \index{彭氏柳葉菜} \\
    Onagraceae & 柳葉菜科 & \textit{Epilobium taiwanianum}  & 臺灣柳葉菜 & D2 \index{Epilobium@\textit{Epilobium}!taiwanianum@\textit{taiwanianum}}  \index{臺灣柳葉菜} \\
    Onagraceae & 柳葉菜科 & \textit{Ludwigia ovalis}  & 卵葉水丁香 & D2 \index{Ludwigia@\textit{Ludwigia}!ovalis@\textit{ovalis}}  \index{卵葉水丁香} \\
    Onagraceae & 柳葉菜科 & \textit{Ludwigia perennis}  & 小花水丁香 & D2 \index{Ludwigia@\textit{Ludwigia}!perennis@\textit{perennis}}  \index{小花水丁香} \\
    Orchidaceae & 蘭科 & \textit{Androcorys pusillus}  & 小兜蕊蘭 & D1 \index{Androcorys@\textit{Androcorys}!pusillus@\textit{pusillus}}  \index{小兜蕊蘭} \\
    Orchidaceae & 蘭科 & \textit{Appendicula kotoensis}  & 蘭嶼竹節蘭 & D1 \index{Appendicula@\textit{Appendicula}!kotoensis@\textit{kotoensis}}  \index{蘭嶼竹節蘭} \\
    Orchidaceae & 蘭科 & \textit{Bulbophyllum albociliatum} var. \textit{albociliatum}  & 白毛捲瓣蘭 & D1 \index{Bulbophyllum@\textit{Bulbophyllum}!albociliatum@\textit{albociliatum}!var. albociliatum@var. \textit{albociliatum}}  \index{白毛捲瓣蘭} \\
    Orchidaceae & 蘭科 & \textit{Bulbophyllum brevipedunculatum}  & 短梗豆蘭 & D1 \index{Bulbophyllum@\textit{Bulbophyllum}!brevipedunculatum@\textit{brevipedunculatum}}  \index{短梗豆蘭} \\
    Orchidaceae & 蘭科 & \textit{Bulbophyllum confragosum}  & 斷尾捲瓣蘭 & D2 \index{Bulbophyllum@\textit{Bulbophyllum}!confragosum@\textit{confragosum}}  \index{斷尾捲瓣蘭} \\
    Orchidaceae & 蘭科 & \textit{Bulbophyllum sui}  & 長軸捲瓣蘭 & D2 \index{Bulbophyllum@\textit{Bulbophyllum}!sui@\textit{sui}}  \index{長軸捲瓣蘭} \\
    Orchidaceae & 蘭科 & \textit{Calanthe angustifolia}  & 矮根節蘭 & D1 \index{Calanthe@\textit{Calanthe}!angustifolia@\textit{angustifolia}}  \index{矮根節蘭} \\
    Orchidaceae & 蘭科 & \textit{Cymbidium faberi}  & 九華蘭 & A1acd \index{Cymbidium@\textit{Cymbidium}!faberi@\textit{faberi}}  \index{九華蘭} \\
    Orchidaceae & 蘭科 & \textit{Cymbidium kanran}  & 寒蘭 & A1acd \index{Cymbidium@\textit{Cymbidium}!kanran@\textit{kanran}}  \index{寒蘭} \\
    Orchidaceae & 蘭科 & \textit{Dendrobium catenatum}  & 黃花石斛 & D1 \index{Dendrobium@\textit{Dendrobium}!catenatum@\textit{catenatum}}  \index{黃花石斛} \\
    Orchidaceae & 蘭科 & \textit{Dendrobium goldschmidtianum}  & 紅花石斛 & A2ad \index{Dendrobium@\textit{Dendrobium}!goldschmidtianum@\textit{goldschmidtianum}}  \index{紅花石斛} \\
    Orchidaceae & 蘭科 & \textit{Dendrobium somai}  & 小雙花石斛 & B2ab(iii,v) \index{Dendrobium@\textit{Dendrobium}!somai@\textit{somai}}  \index{小雙花石斛} \\
    Orchidaceae & 蘭科 & \textit{Dendrochilum uncatum}  & 黃穗蘭 & D1 \index{Dendrochilum@\textit{Dendrochilum}!uncatum@\textit{uncatum}}  \index{黃穗蘭} \\
    Orchidaceae & 蘭科 & \textit{Didymoplexiella siamensis}  & 錨柱蘭 & D1 \index{Didymoplexiella@\textit{Didymoplexiella}!siamensis@\textit{siamensis}}  \index{錨柱蘭} \\
    Orchidaceae & 蘭科 & \textit{Erythrorchis altissima}  & 蔓莖山珊瑚 & D1 \index{Erythrorchis@\textit{Erythrorchis}!altissima@\textit{altissima}}  \index{蔓莖山珊瑚} \\
    Orchidaceae & 蘭科 & \textit{Eucosia hengchunensis}  & 恆春歌綠懷蘭 & D2 \index{Eucosia@\textit{Eucosia}!hengchunensis@\textit{hengchunensis}}  \index{恆春歌綠懷蘭} \\
    Orchidaceae & 蘭科 & \textit{Gastrochilus rantabunensis}  & 合歡松蘭 & D1 \index{Gastrochilus@\textit{Gastrochilus}!rantabunensis@\textit{rantabunensis}}  \index{合歡松蘭} \\
    Orchidaceae & 蘭科 & \textit{Gastrodia flabilabella}  & 夏赤箭 & D1 \index{Gastrodia@\textit{Gastrodia}!flabilabella@\textit{flabilabella}}  \index{夏赤箭} \\
    Orchidaceae & 蘭科 & \textit{Gastrodia fontinalis} var. \textit{suburceolata}  & 壺花赤箭 & D1+2 \index{Gastrodia@\textit{Gastrodia}!fontinalis@\textit{fontinalis}!var. suburceolata@var. \textit{suburceolata}}  \index{壺花赤箭} \\
    Orchidaceae & 蘭科 & \textit{Gastrodia pubilabiata}  & 冬赤箭 & D1 \index{Gastrodia@\textit{Gastrodia}!pubilabiata@\textit{pubilabiata}}  \index{冬赤箭} \\
    Orchidaceae & 蘭科 & \textit{Gastrodia shimizuana}  & 清水氏赤箭 & D1 \index{Gastrodia@\textit{Gastrodia}!shimizuana@\textit{shimizuana}}  \index{清水氏赤箭} \\
    Orchidaceae & 蘭科 & \textit{Goodyera bomiensis}  & 波密斑葉蘭 & D2 \index{Goodyera@\textit{Goodyera}!bomiensis@\textit{bomiensis}}  \index{波密斑葉蘭} \\
    Orchidaceae & 蘭科 & \textit{Goodyera pendula}  & 垂葉斑葉蘭 & D1 \index{Goodyera@\textit{Goodyera}!pendula@\textit{pendula}}  \index{垂葉斑葉蘭} \\
    Orchidaceae & 蘭科 & \textit{Goodyera viridiflora}  & 綠花斑葉蘭 & D1 \index{Goodyera@\textit{Goodyera}!viridiflora@\textit{viridiflora}}  \index{綠花斑葉蘭} \\
    Orchidaceae & 蘭科 & \textit{Goodyera yamiana}  & 蘭嶼金銀草 & D1 \index{Goodyera@\textit{Goodyera}!yamiana@\textit{yamiana}}  \index{蘭嶼金銀草} \\
    Orchidaceae & 蘭科 & \textit{Haraella retrocalla}  & 香蘭 & A1acd \index{Haraella@\textit{Haraella}!retrocalla@\textit{retrocalla}}  \index{香蘭} \\
    Orchidaceae & 蘭科 & \textit{Hemipilia cordifolia}  & 玉山一葉蘭 & D1 \index{Hemipilia@\textit{Hemipilia}!cordifolia@\textit{cordifolia}}  \index{玉山一葉蘭} \\
    Orchidaceae & 蘭科 & \textit{Kuhlhasseltia integra}  & 綠葉旗唇蘭 & D1 \index{Kuhlhasseltia@\textit{Kuhlhasseltia}!integra@\textit{integra}}  \index{綠葉旗唇蘭} \\
    Orchidaceae & 蘭科 & \textit{Lecanorchis amethystea}  & 紫晶皿蘭 & D1 \index{Lecanorchis@\textit{Lecanorchis}!amethystea@\textit{amethystea}}  \index{紫晶皿蘭} \\
    Orchidaceae & 蘭科 & \textit{Liparis auriculata}  & 雙葉羊耳蒜 & D1 \index{Liparis@\textit{Liparis}!auriculata@\textit{auriculata}}  \index{雙葉羊耳蒜} \\
    Orchidaceae & 蘭科 & \textit{Liparis krameri} var. \textit{sasakii}  & 尾唇羊耳蒜 & D1 \index{Liparis@\textit{Liparis}!krameri@\textit{krameri}!var. sasakii@var. \textit{sasakii}}  \index{尾唇羊耳蒜} \\
    Orchidaceae & 蘭科 & \textit{Malaxis microtatantha}  & 小軟葉蘭 & D1+2 \index{Malaxis@\textit{Malaxis}!microtatantha@\textit{microtatantha}}  \index{小軟葉蘭} \\
    Orchidaceae & 蘭科 & \textit{Malaxis purpurea}  & 紫花軟葉蘭 & D1 \index{Malaxis@\textit{Malaxis}!purpurea@\textit{purpurea}}  \index{紫花軟葉蘭} \\
    Orchidaceae & 蘭科 & \textit{Malaxis ramosii}  & 圓唇軟葉蘭 & D1 \index{Malaxis@\textit{Malaxis}!ramosii@\textit{ramosii}}  \index{圓唇軟葉蘭} \\
    Orchidaceae & 蘭科 & \textit{Neottia breviscapa}  & 短葶雙葉蘭 & D2 \index{Neottia@\textit{Neottia}!breviscapa@\textit{breviscapa}}  \index{短葶雙葉蘭} \\
    Orchidaceae & 蘭科 & \textit{Neottia microauriculata}  & 微耳雙葉蘭 & D1 \index{Neottia@\textit{Neottia}!microauriculata@\textit{microauriculata}}  \index{微耳雙葉蘭} \\
    Orchidaceae & 蘭科 & \textit{Neottia nankomontana}  & 南湖雙葉蘭 & D1+2 \index{Neottia@\textit{Neottia}!nankomontana@\textit{nankomontana}}  \index{南湖雙葉蘭} \\
    Orchidaceae & 蘭科 & \textit{Neottia taizanensis}  & 大山雙葉蘭 & D1 \index{Neottia@\textit{Neottia}!taizanensis@\textit{taizanensis}}  \index{大山雙葉蘭} \\
    Orchidaceae & 蘭科 & \textit{Neottia tatakaensis}  & 塔塔加雙葉蘭 & D1 \index{Neottia@\textit{Neottia}!tatakaensis@\textit{tatakaensis}}  \index{塔塔加雙葉蘭} \\
    Orchidaceae & 蘭科 & \textit{Nervilia crociformis}  & 四重溪脈葉蘭 & D1 \index{Nervilia@\textit{Nervilia}!crociformis@\textit{crociformis}}  \index{四重溪脈葉蘭} \\
    Orchidaceae & 蘭科 & \textit{Nervilia lanyuensis}  & 蘭嶼脈葉蘭 & D1+2 \index{Nervilia@\textit{Nervilia}!lanyuensis@\textit{lanyuensis}}  \index{蘭嶼脈葉蘭} \\
    Orchidaceae & 蘭科 & \textit{Nervilia plicata}  & 紫花脈葉蘭 & C1 \index{Nervilia@\textit{Nervilia}!plicata@\textit{plicata}}  \index{紫花脈葉蘭} \\
    Orchidaceae & 蘭科 & \textit{Oberonia pumila} var. \textit{pumila}  & 小騎士蘭 & D1 \index{Oberonia@\textit{Oberonia}!pumila@\textit{pumila}!var. pumila@var. \textit{pumila}}  \index{小騎士蘭} \\
    Orchidaceae & 蘭科 & \textit{Oreorchis bilamellata}  & 雙板山蘭 & D1 \index{Oreorchis@\textit{Oreorchis}!bilamellata@\textit{bilamellata}}  \index{雙板山蘭} \\
    Orchidaceae & 蘭科 & \textit{Oreorchis fargesii}  & 密花山蘭 & D1 \index{Oreorchis@\textit{Oreorchis}!fargesii@\textit{fargesii}}  \index{密花山蘭} \\
    Orchidaceae & 蘭科 & \textit{Oreorchis micrantha}  & 南湖山蘭 & D1 \index{Oreorchis@\textit{Oreorchis}!micrantha@\textit{micrantha}}  \index{南湖山蘭} \\
    Orchidaceae & 蘭科 & \textit{Phaius tankervilleae}  & 紅鶴頂蘭 & C1; D1 \index{Phaius@\textit{Phaius}!tankervilleae@\textit{tankervilleae}}  \index{紅鶴頂蘭} \\
    Orchidaceae & 蘭科 & \textit{Platanthera longicalcarata}  & 長距粉蝶蘭 & D1 \index{Platanthera@\textit{Platanthera}!longicalcarata@\textit{longicalcarata}}  \index{長距粉蝶蘭} \\
    Orchidaceae & 蘭科 & \textit{Pleione bulbocodioides}  & 臺灣一葉蘭 & D1 \index{Pleione@\textit{Pleione}!bulbocodioides@\textit{bulbocodioides}}  \index{臺灣一葉蘭} \\
    Orchidaceae & 蘭科 & \textit{Ponerorchis takasago-montana}  & 高山小蝶蘭 & D1 \index{Ponerorchis@\textit{Ponerorchis}!takasago-montana@\textit{takasago-montana}}  \index{高山小蝶蘭} \\
    Orchidaceae & 蘭科 & \textit{Ponerorchis tominagai}  & 紅斑蘭 & D1 \index{Ponerorchis@\textit{Ponerorchis}!tominagai@\textit{tominagai}}  \index{紅斑蘭} \\
    Orchidaceae & 蘭科 & \textit{Schoenorchis vanoverberghii}  & 蘆蘭 & B2a; C2a(i) \index{Schoenorchis@\textit{Schoenorchis}!vanoverberghii@\textit{vanoverberghii}}  \index{蘆蘭} \\
    Orchidaceae & 蘭科 & \textit{Stigmatodactylus shikokiana}  & 絲柱蘭 & D1 \index{Stigmatodactylus@\textit{Stigmatodactylus}!shikokiana@\textit{shikokiana}}  \index{絲柱蘭} \\
    Orchidaceae & 蘭科 & \textit{Tainia latifolia}  & 闊葉杜鵑蘭 & C2a(i) \index{Tainia@\textit{Tainia}!latifolia@\textit{latifolia}}  \index{闊葉杜鵑蘭} \\
    Orchidaceae & 蘭科 & \textit{Thrixspermum pensile}  & 倒垂風蘭 & D1 \index{Thrixspermum@\textit{Thrixspermum}!pensile@\textit{pensile}}  \index{倒垂風蘭} \\
    Orchidaceae & 蘭科 & \textit{Thrixspermum subulatum}  & 厚葉風蘭 & D1 \index{Thrixspermum@\textit{Thrixspermum}!subulatum@\textit{subulatum}}  \index{厚葉風蘭} \\
    Orchidaceae & 蘭科 & \textit{Tropidia nanhuae}  & 南化摺唇蘭 & D1 \index{Tropidia@\textit{Tropidia}!nanhuae@\textit{nanhuae}}  \index{南化摺唇蘭} \\
    Orchidaceae & 蘭科 & \textit{Tuberolabium kotoense}  & 管唇蘭 & A1acd \index{Tuberolabium@\textit{Tuberolabium}!kotoense@\textit{kotoense}}  \index{管唇蘭} \\
    Orchidaceae & 蘭科 & \textit{Zeuxine integrilabella}  & 全唇線柱蘭 & D1 \index{Zeuxine@\textit{Zeuxine}!integrilabella@\textit{integrilabella}}  \index{全唇線柱蘭} \\
    Orchidaceae & 蘭科 & \textit{Zeuxine yehii}  & 蘭嶼線柱蘭 & D2 \index{Zeuxine@\textit{Zeuxine}!yehii@\textit{yehii}}  \index{蘭嶼線柱蘭} \\
    Orobanchaceae & 列當科 & \textit{Christisonia hookeri}  & 假野菰 & D2 \index{Christisonia@\textit{Christisonia}!hookeri@\textit{hookeri}}  \index{假野菰} \\
    Orobanchaceae & 列當科 & \textit{Euphrasia tarokoana}  & 太魯閣小米草 & D2 \index{Euphrasia@\textit{Euphrasia}!tarokoana@\textit{tarokoana}}  \index{太魯閣小米草} \\
    Oxalidaceae & 酢醬草科 & \textit{Oxalis acetosella} subsp. \textit{taemoni}  & 大霸尖山酢漿草 & D2 \index{Oxalis@\textit{Oxalis}!acetosella@\textit{acetosella}!subsp. taemoni@subsp. \textit{taemoni}}  \index{大霸尖山酢漿草} \\
    Paulowniaceae & 泡桐科 & \textit{Paulownia kawakamii}  & 白桐 & C2a(i) \index{Paulownia@\textit{Paulownia}!kawakamii@\textit{kawakamii}}  \index{白桐} \\
    Pentaphylacaceae & 五列木科 & \textit{Anneslea lanceolata}  & 細葉茶梨 & C2a(i) \index{Anneslea@\textit{Anneslea}!lanceolata@\textit{lanceolata}}  \index{細葉茶梨} \\
    Phyllanthaceae & 葉下珠科 & \textit{Antidesma pleuricum}  & 蘭嶼枯里珍 & B2ab(ii,v); D1 \index{Antidesma@\textit{Antidesma}!pleuricum@\textit{pleuricum}}  \index{蘭嶼枯里珍} \\
    Phyllanthaceae & 葉下珠科 & \textit{Glochidion puber}  & 紅毛饅頭果 & B2ab(i,ii,iii); D \index{Glochidion@\textit{Glochidion}!puber@\textit{puber}}  \index{紅毛饅頭果} \\
    Phyllanthaceae & 葉下珠科 & \textit{Margaritaria indica}  & 紫黃 & B2ab(i,ii) \index{Margaritaria@\textit{Margaritaria}!indica@\textit{indica}}  \index{紫黃} \\
    Pittosporaceae & 海桐科 & \textit{Pittosporum illicioides} var. \textit{angustifolium}  & 細葉疏果海桐 & A4; D1 \index{Pittosporum@\textit{Pittosporum}!illicioides@\textit{illicioides}!var. angustifolium@var. \textit{angustifolium}}  \index{細葉疏果海桐} \\
    Plantaginaceae & 車前科 & \textit{Deinostema adenocaulon}  & 毛澤番椒 & D2 \index{Deinostema@\textit{Deinostema}!adenocaulon@\textit{adenocaulon}}  \index{毛澤番椒} \\
    Plantaginaceae & 車前科 & \textit{Limnophila aromatica}  & 紫蘇草 & B2ab(iii); D2 \index{Limnophila@\textit{Limnophila}!aromatica@\textit{aromatica}}  \index{紫蘇草} \\
    Plantaginaceae & 車前科 & \textit{Limnophila rugosa}  & 大葉石龍尾 & B2ab(iii) \index{Limnophila@\textit{Limnophila}!rugosa@\textit{rugosa}}  \index{大葉石龍尾} \\
    Plantaginaceae & 車前科 & \textit{Veronicastrum axillare} var. \textit{simadai}  & 新竹腹水草 & D1 \index{Veronicastrum@\textit{Veronicastrum}!axillare@\textit{axillare}!var. simadai@var. \textit{simadai}}  \index{新竹腹水草} \\
    Poaceae & 禾本科 & \textit{Sporobolus hancei}  & 韓氏鼠尾粟 & D1+2 \index{Sporobolus@\textit{Sporobolus}!hancei@\textit{hancei}}  \index{韓氏鼠尾粟} \\
    Poaceae & 禾本科 & \textit{Thaumastochloa chenii}  & 其昌假蛇尾草 & B2b(ii,iii)c(ii); D1 \index{Thaumastochloa@\textit{Thaumastochloa}!chenii@\textit{chenii}}  \index{其昌假蛇尾草} \\
    Poaceae & 禾本科 & \textit{Trisetum bifidum}  & 三毛草 & B2ab(ii)c(iv); D \index{Trisetum@\textit{Trisetum}!bifidum@\textit{bifidum}}  \index{三毛草} \\
    Polygalaceae & 遠志科 & \textit{Epirixanthes elongata}  & 寄生鱗葉草 & D1+2 \index{Epirixanthes@\textit{Epirixanthes}!elongata@\textit{elongata}}  \index{寄生鱗葉草} \\
    Polygalaceae & 遠志科 & \textit{Polygala arcuata}  & 巨葉花遠志 & D1+2 \index{Polygala@\textit{Polygala}!arcuata@\textit{arcuata}}  \index{巨葉花遠志} \\
    Polygalaceae & 遠志科 & \textit{Polygala chinensis}  & 華南遠志 & C1 \index{Polygala@\textit{Polygala}!chinensis@\textit{chinensis}}  \index{華南遠志} \\
    Polygonaceae & 蓼科 & \textit{Persicaria hastatosagittata}  & 長箭葉蓼 & B2ab(iii) \index{Persicaria@\textit{Persicaria}!hastatosagittata@\textit{hastatosagittata}}  \index{長箭葉蓼} \\
    Polygonaceae & 蓼科 & \textit{Persicaria pulchra}  & 絨毛蓼 & D2 \index{Persicaria@\textit{Persicaria}!pulchra@\textit{pulchra}}  \index{絨毛蓼} \\
    Polygonaceae & 蓼科 & \textit{Persicaria sagittata}  & 箭葉蓼 & D2 \index{Persicaria@\textit{Persicaria}!sagittata@\textit{sagittata}}  \index{箭葉蓼} \\
    Potamogetonaceae & 眼子菜科 & \textit{Potamogeton distinctus}  & 異匙葉藻 & A2cd \index{Potamogeton@\textit{Potamogeton}!distinctus@\textit{distinctus}}  \index{異匙葉藻} \\
    Potamogetonaceae & 眼子菜科 & \textit{Potamogeton oxyphyllus}  & 線葉藻 & D2 \index{Potamogeton@\textit{Potamogeton}!oxyphyllus@\textit{oxyphyllus}}  \index{線葉藻} \\
    Potamogetonaceae & 眼子菜科 & \textit{Potamogeton pusillus}  & 柳絲藻 & A2cd \index{Potamogeton@\textit{Potamogeton}!pusillus@\textit{pusillus}}  \index{柳絲藻} \\
    Potamogetonaceae & 眼子菜科 & \textit{Zannichellia palustris}  & 角果藻 & D1 \index{Zannichellia@\textit{Zannichellia}!palustris@\textit{palustris}}  \index{角果藻} \\
    Primulaceae & 報春花科 & \textit{Ardisia kusukuensis}  & 高士佛紫金牛 & D1+2 \index{Ardisia@\textit{Ardisia}!kusukuensis@\textit{kusukuensis}}  \index{高士佛紫金牛} \\
    Primulaceae & 報春花科 & \textit{Ardisia maclurei}  & 麥氏紫金牛 & D1+2 \index{Ardisia@\textit{Ardisia}!maclurei@\textit{maclurei}}  \index{麥氏紫金牛} \\
    Primulaceae & 報春花科 & \textit{Ardisia villosa}  & 雪下紅 & D1 \index{Ardisia@\textit{Ardisia}!villosa@\textit{villosa}}  \index{雪下紅} \\
    Putranjivaceae & 非洲核果木科 & \textit{Drypetes littoralis}  & 鐵色 & D1 \index{Drypetes@\textit{Drypetes}!littoralis@\textit{littoralis}}  \index{鐵色} \\
    Ranunculaceae & 毛茛科 & \textit{Aconitum fukutomei} var. \textit{formosanum}  & 蔓烏頭 & D1 \index{Aconitum@\textit{Aconitum}!fukutomei@\textit{fukutomei}!var. formosanum@var. \textit{formosanum}}  \index{蔓烏頭} \\
    Ranunculaceae & 毛茛科 & \textit{Anemone stolonifera}  & 匍枝銀蓮花 & D1 \index{Anemone@\textit{Anemone}!stolonifera@\textit{stolonifera}}  \index{匍枝銀蓮花} \\
    Ranunculaceae & 毛茛科 & \textit{Clematis psilandra}  & 臺灣牡丹藤 & D1 \index{Clematis@\textit{Clematis}!psilandra@\textit{psilandra}}  \index{臺灣牡丹藤} \\
    Ranunculaceae & 毛茛科 & \textit{Clematis tashiroi} var. \textit{huangii}  & 黃氏鐵線蓮 & D1 \index{Clematis@\textit{Clematis}!tashiroi@\textit{tashiroi}!var. huangii@var. \textit{huangii}}  \index{黃氏鐵線蓮} \\
    Ranunculaceae & 毛茛科 & \textit{Clematis terniflora} var. \textit{garanbiensis}  & 鵝鑾鼻鐵線蓮 & D2 \index{Clematis@\textit{Clematis}!terniflora@\textit{terniflora}!var. garanbiensis@var. \textit{garanbiensis}}  \index{鵝鑾鼻鐵線蓮} \\
    Ranunculaceae & 毛茛科 & \textit{Clematis tsugetorum}  & 高山鐵線蓮 & D1 \index{Clematis@\textit{Clematis}!tsugetorum@\textit{tsugetorum}}  \index{高山鐵線蓮} \\
    Ranunculaceae & 毛茛科 & \textit{Ranunculus chinensis}  & 茴茴蒜 & D2 \index{Ranunculus@\textit{Ranunculus}!chinensis@\textit{chinensis}}  \index{茴茴蒜} \\
    Ranunculaceae & 毛茛科 & \textit{Ranunculus ternatus}  & 小毛茛 & D1 \index{Ranunculus@\textit{Ranunculus}!ternatus@\textit{ternatus}}  \index{小毛茛} \\
    Ranunculaceae & 毛茛科 & \textit{Thalictrum myriophyllum}  & 密葉唐松草 & D1 \index{Thalictrum@\textit{Thalictrum}!myriophyllum@\textit{myriophyllum}}  \index{密葉唐松草} \\
    Ranunculaceae & 毛茛科 & \textit{Thalictrum rubescens}  & 南湖唐松草 & D1 \index{Thalictrum@\textit{Thalictrum}!rubescens@\textit{rubescens}}  \index{南湖唐松草} \\
    Ranunculaceae & 毛茛科 & \textit{Thalictrum urbaini} var. \textit{majus}  & 大花傅氏唐松草 & D1 \index{Thalictrum@\textit{Thalictrum}!urbaini@\textit{urbaini}!var. majus@var. \textit{majus}}  \index{大花傅氏唐松草} \\
    Ranunculaceae & 毛茛科 & \textit{Trollius taihasenzanensis}  & 臺灣金蓮花 & D1 \index{Trollius@\textit{Trollius}!taihasenzanensis@\textit{taihasenzanensis}}  \index{臺灣金蓮花} \\
    Rhamnaceae & 鼠李科 & \textit{Berchemia arisanensis}  & 阿里山黃鱔藤 & A4; D1 \index{Berchemia@\textit{Berchemia}!arisanensis@\textit{arisanensis}}  \index{阿里山黃鱔藤} \\
    Rhizophoraceae & 紅樹科 & \textit{Rhizophora stylosa}  & 紅海欖 & C1 \index{Rhizophora@\textit{Rhizophora}!stylosa@\textit{stylosa}}  \index{紅海欖} \\
    Rosaceae & 薔薇科 & \textit{Filipendula kiraishiensis}  & 臺灣蚊子草 & C2a(i) \index{Filipendula@\textit{Filipendula}!kiraishiensis@\textit{kiraishiensis}}  \index{臺灣蚊子草} \\
    Rosaceae & 薔薇科 & \textit{Geum japonicum}  & 日本水楊梅 & D1+2 \index{Geum@\textit{Geum}!japonicum@\textit{japonicum}}  \index{日本水楊梅} \\
    Rosaceae & 薔薇科 & \textit{Prunus pogonostyla}  & 庭梅 & A4a \index{Prunus@\textit{Prunus}!pogonostyla@\textit{pogonostyla}}  \index{庭梅} \\
    Rosaceae & 薔薇科 & \textit{Pyracantha koidzumii}  & 臺灣火刺木 & A1d; D1 \index{Pyracantha@\textit{Pyracantha}!koidzumii@\textit{koidzumii}}  \index{臺灣火刺木} \\
    Rosaceae & 薔薇科 & \textit{Rosa bracteata} var. \textit{bracteata}  & 琉球野薔薇 & B1ab(i)+2ab(iii) \index{Rosa@\textit{Rosa}!bracteata@\textit{bracteata}!var. bracteata@var. \textit{bracteata}}  \index{琉球野薔薇} \\
    Rosaceae & 薔薇科 & \textit{Rosa cymosa}  & 小果薔薇 & B2ab(ii); D2 \index{Rosa@\textit{Rosa}!cymosa@\textit{cymosa}}  \index{小果薔薇} \\
    Rosaceae & 薔薇科 & \textit{Rubus nagasawanus} var. \textit{arachnoideus}  & 灰葉懸鉤子 & B2ab(i) \index{Rubus@\textit{Rubus}!nagasawanus@\textit{nagasawanus}!var. arachnoideus@var. \textit{arachnoideus}}  \index{灰葉懸鉤子} \\
    Rosaceae & 薔薇科 & \textit{Rubus nagasawanus} var. \textit{nagasawanus}  & 粗毛懸鉤子 & A3; B1ab(iii); D1+2 \index{Rubus@\textit{Rubus}!nagasawanus@\textit{nagasawanus}!var. nagasawanus@var. \textit{nagasawanus}}  \index{粗毛懸鉤子} \\
    Rosaceae & 薔薇科 & \textit{Spiraea tarokoensis}  & 太魯閣繡線菊 & D1 \index{Spiraea@\textit{Spiraea}!tarokoensis@\textit{tarokoensis}}  \index{太魯閣繡線菊} \\
    Rosaceae & 薔薇科 & \textit{Spiraea tatakaensis}  & 塔塔加繡線菊 & A4a; B1ab(ii); D2; E \index{Spiraea@\textit{Spiraea}!tatakaensis@\textit{tatakaensis}}  \index{塔塔加繡線菊} \\
    Rosaceae & 薔薇科 & \textit{Stephanandra incisa}  & 冠蕊木 & D1 \index{Stephanandra@\textit{Stephanandra}!incisa@\textit{incisa}}  \index{冠蕊木} \\
    Rubiaceae & 茜草科 & \textit{Argostemma solaniflorum}  & 水冠草 & D2 \index{Argostemma@\textit{Argostemma}!solaniflorum@\textit{solaniflorum}}  \index{水冠草} \\
    Rubiaceae & 茜草科 & \textit{Galium fukuyamai}  & 福山氏豬殃殃 & D1 \index{Galium@\textit{Galium}!fukuyamai@\textit{fukuyamai}}  \index{福山氏豬殃殃} \\
    Rubiaceae & 茜草科 & \textit{Galium morii}  & 森氏豬殃殃 & D1 \index{Galium@\textit{Galium}!morii@\textit{morii}}  \index{森氏豬殃殃} \\
    Rubiaceae & 茜草科 & \textit{Hedyotis chrysotricha}  & 金毛耳草 & D2 \index{Hedyotis@\textit{Hedyotis}!chrysotricha@\textit{chrysotricha}}  \index{金毛耳草} \\
    Rubiaceae & 茜草科 & \textit{Lasianthus hiiranensis}  & 棲蘭山雞屎樹 & D2 \index{Lasianthus@\textit{Lasianthus}!hiiranensis@\textit{hiiranensis}}  \index{棲蘭山雞屎樹} \\
    Rubiaceae & 茜草科 & \textit{Lasianthus tsangii}  & 長苞雞屎樹 & D1 \index{Lasianthus@\textit{Lasianthus}!tsangii@\textit{tsangii}}  \index{長苞雞屎樹} \\
    Rubiaceae & 茜草科 & \textit{Mitchella undulata}  & 蔓虎刺 & D1+2 \index{Mitchella@\textit{Mitchella}!undulata@\textit{undulata}}  \index{蔓虎刺} \\
    Rubiaceae & 茜草科 & \textit{Mussaenda albiflora}  & 水社玉葉金花 & D1 \index{Mussaenda@\textit{Mussaenda}!albiflora@\textit{albiflora}}  \index{水社玉葉金花} \\
    Rubiaceae & 茜草科 & \textit{Mussaenda taiwanensis}  & 臺灣玉葉金花 & D1 \index{Mussaenda@\textit{Mussaenda}!taiwanensis@\textit{taiwanensis}}  \index{臺灣玉葉金花} \\
    Rubiaceae & 茜草科 & \textit{Ophiorrhiza mitchelloides}  & 玉蘭草 & D2 \index{Ophiorrhiza@\textit{Ophiorrhiza}!mitchelloides@\textit{mitchelloides}}  \index{玉蘭草} \\
    Rubiaceae & 茜草科 & \textit{Randia canthioides}  & 臺北茜草樹 & D2 \index{Randia@\textit{Randia}!canthioides@\textit{canthioides}}  \index{臺北茜草樹} \\
    Rubiaceae & 茜草科 & \textit{Randia wallichii}  & 大果玉心花 & D2 \index{Randia@\textit{Randia}!wallichii@\textit{wallichii}}  \index{大果玉心花} \\
    Rubiaceae & 茜草科 & \textit{Uncaria rhynchophylla}  & 嘴葉鉤藤 & D2 \index{Uncaria@\textit{Uncaria}!rhynchophylla@\textit{rhynchophylla}}  \index{嘴葉鉤藤} \\
    Rutaceae & 芸香科 & \textit{Citrus depressa}  & 臺灣香檬 & D1 \index{Citrus@\textit{Citrus}!depressa@\textit{depressa}}  \index{臺灣香檬} \\
    Rutaceae & 芸香科 & \textit{Clausena anisum-olens}  & 短柱黃皮 & D1 \index{Clausena@\textit{Clausena}!anisum-olens@\textit{anisum-olens}}  \index{短柱黃皮} \\
    Rutaceae & 芸香科 & \textit{Zanthoxylum acanthopodium}  & 岩花椒 & D1 \index{Zanthoxylum@\textit{Zanthoxylum}!acanthopodium@\textit{acanthopodium}}  \index{岩花椒} \\
    Rutaceae & 芸香科 & \textit{Zanthoxylum avicennae}  & 狗花椒 & B1ab(v)+2ab(v); C2a(i) \index{Zanthoxylum@\textit{Zanthoxylum}!avicennae@\textit{avicennae}}  \index{狗花椒} \\
    Rutaceae & 芸香科 & \textit{Zanthoxylum integrifoliolum}  & 蘭嶼花椒 & B1ab(v)+2ab(v); C2a(i) \index{Zanthoxylum@\textit{Zanthoxylum}!integrifoliolum@\textit{integrifoliolum}}  \index{蘭嶼花椒} \\
    Santalaceae & 檀香科 & \textit{Thesium chinense}  & 百蕊草 & D2 \index{Thesium@\textit{Thesium}!chinense@\textit{chinense}}  \index{百蕊草} \\
    Saxifragaceae & 虎耳草科 & \textit{Chrysosplenium japonicum}  & 日本貓兒眼睛草 & D2 \index{Chrysosplenium@\textit{Chrysosplenium}!japonicum@\textit{japonicum}}  \index{日本貓兒眼睛草} \\
    Saxifragaceae & 虎耳草科 & \textit{Tiarella polyphylla}  & 黃水枝 & D1+2 \index{Tiarella@\textit{Tiarella}!polyphylla@\textit{polyphylla}}  \index{黃水枝} \\
    Scrophulariaceae & 玄參科 & \textit{Buddleja curviflora}  & 彎花醉魚木 & D1 \index{Buddleja@\textit{Buddleja}!curviflora@\textit{curviflora}}  \index{彎花醉魚木} \\
    Simaroubaceae & 苦木科 & \textit{Picrasma quassioides}  & 苦樹 & D1 \index{Picrasma@\textit{Picrasma}!quassioides@\textit{quassioides}}  \index{苦樹} \\
    Smilacaceae & 菝葜科 & \textit{Smilax luei}  & 呂氏菝葜 & D2 \index{Smilax@\textit{Smilax}!luei@\textit{luei}}  \index{呂氏菝葜} \\
    Smilacaceae & 菝葜科 & \textit{Smilax nipponica}  & 日本菝葜 & D2 \index{Smilax@\textit{Smilax}!nipponica@\textit{nipponica}}  \index{日本菝葜} \\
    Stemonuraceae & 金檀木科 & \textit{Gomphandra luzoniensis}  & 呂宋毛蕊木 & C2a(i); D1 \index{Gomphandra@\textit{Gomphandra}!luzoniensis@\textit{luzoniensis}}  \index{呂宋毛蕊木} \\
    Styracaceae & 安息香科 & \textit{Styrax japonica} var. \textit{kotoensis}  & 蘭嶼野茉莉 & D1 \index{Styrax@\textit{Styrax}!japonica@\textit{japonica}!var. kotoensis@var. \textit{kotoensis}}  \index{蘭嶼野茉莉} \\
    Styracaceae & 安息香科 & \textit{Styrax matsumuraei}  & 臺灣野茉莉 & D1 \index{Styrax@\textit{Styrax}!matsumuraei@\textit{matsumuraei}}  \index{臺灣野茉莉} \\
    Symplocaceae & 灰木科 & \textit{Symplocos decora}  & 小泉氏灰木 & D1 \index{Symplocos@\textit{Symplocos}!decora@\textit{decora}}  \index{小泉氏灰木} \\
    Symplocaceae & 灰木科 & \textit{Symplocos grandis}  & 大葉灰木 & D1 \index{Symplocos@\textit{Symplocos}!grandis@\textit{grandis}}  \index{大葉灰木} \\
    Symplocaceae & 灰木科 & \textit{Symplocos nokoensis}  & 能高山灰木 & D1 \index{Symplocos@\textit{Symplocos}!nokoensis@\textit{nokoensis}}  \index{能高山灰木} \\
    Symplocaceae & 灰木科 & \textit{Symplocos sasakii}  & 佐佐木氏灰木 & D1 \index{Symplocos@\textit{Symplocos}!sasakii@\textit{sasakii}}  \index{佐佐木氏灰木} \\
    Symplocaceae & 灰木科 & \textit{Symplocos shilanensis}  & 希蘭灰木 & D1 \index{Symplocos@\textit{Symplocos}!shilanensis@\textit{shilanensis}}  \index{希蘭灰木} \\
    Theaceae & 茶科 & \textit{Camellia furfuracea}  & 垢果山茶 & C2a(i) \index{Camellia@\textit{Camellia}!furfuracea@\textit{furfuracea}}  \index{垢果山茶} \\
    Theaceae & 茶科 & \textit{Camellia hengchunensis}  & 恆春山茶 & D1+D2 \index{Camellia@\textit{Camellia}!hengchunensis@\textit{hengchunensis}}  \index{恆春山茶} \\
    Theaceae & 茶科 & \textit{Camellia japonica}  & 日本山茶 & A4a \index{Camellia@\textit{Camellia}!japonica@\textit{japonica}}  \index{日本山茶} \\
    Theaceae & 茶科 & \textit{Camellia kissi}  & 落瓣油茶 & D1+2 \index{Camellia@\textit{Camellia}!kissi@\textit{kissi}}  \index{落瓣油茶} \\
    Theaceae & 茶科 & \textit{Camellia nokoensis}  & 能高山茶 & D1 \index{Camellia@\textit{Camellia}!nokoensis@\textit{nokoensis}}  \index{能高山茶} \\
    Theaceae & 茶科 & \textit{Camellia trichoclada}  & 毛枝連蕊茶 & D1+2 \index{Camellia@\textit{Camellia}!trichoclada@\textit{trichoclada}}  \index{毛枝連蕊茶} \\
    Thymelaeaceae & 瑞香科 & \textit{Daphne genkwa}  & 芫花 & A4a \index{Daphne@\textit{Daphne}!genkwa@\textit{genkwa}}  \index{芫花} \\
    Thymelaeaceae & 瑞香科 & \textit{Wikstroemia mononectaria}  & 紅蕘花 & A4; D1 \index{Wikstroemia@\textit{Wikstroemia}!mononectaria@\textit{mononectaria}}  \index{紅蕘花} \\
    Typhaceae & 香蒲科 & \textit{Sparganium fallax}  & 東亞黑三稜 & D1 \index{Sparganium@\textit{Sparganium}!fallax@\textit{fallax}}  \index{東亞黑三稜} \\
    Urticaceae & 蕁麻科 & \textit{Boehmeria clidemioides}  & 序葉苧麻 & D2 \index{Boehmeria@\textit{Boehmeria}!clidemioides@\textit{clidemioides}}  \index{序葉苧麻} \\
    Urticaceae & 蕁麻科 & \textit{Boehmeria hwaliensis}  & 花蓮苧麻 & D2 \index{Boehmeria@\textit{Boehmeria}!hwaliensis@\textit{hwaliensis}}  \index{花蓮苧麻} \\
    Urticaceae & 蕁麻科 & \textit{Boehmeria pilushanensis}  & 畢祿山苧麻 & B2ab(ii) \index{Boehmeria@\textit{Boehmeria}!pilushanensis@\textit{pilushanensis}}  \index{畢祿山苧麻} \\
    Urticaceae & 蕁麻科 & \textit{Elatostema acuteserratum}  & 銳齒樓梯草 & D2 \index{Elatostema@\textit{Elatostema}!acuteserratum@\textit{acuteserratum}}  \index{銳齒樓梯草} \\
    Urticaceae & 蕁麻科 & \textit{Elatostema oblongifolium}  & 長圓樓梯草 & D1+2 \index{Elatostema@\textit{Elatostema}!oblongifolium@\textit{oblongifolium}}  \index{長圓樓梯草} \\
    Urticaceae & 蕁麻科 & \textit{Elatostema strigillosum}  & 微粗毛樓梯草 & D2 \index{Elatostema@\textit{Elatostema}!strigillosum@\textit{strigillosum}}  \index{微粗毛樓梯草} \\
    Urticaceae & 蕁麻科 & \textit{Elatostema villosum}  & 柔毛樓梯草 & B2ab(iii) \index{Elatostema@\textit{Elatostema}!villosum@\textit{villosum}}  \index{柔毛樓梯草} \\
    Urticaceae & 蕁麻科 & \textit{Gonostegia pentandra}  & 五蕊石薯 & D2 \index{Gonostegia@\textit{Gonostegia}!pentandra@\textit{pentandra}}  \index{五蕊石薯} \\
    Urticaceae & 蕁麻科 & \textit{Laportea bulbifera}  & 珠芽桑葉麻 & B2ab(ii) \index{Laportea@\textit{Laportea}!bulbifera@\textit{bulbifera}}  \index{珠芽桑葉麻} \\
    Urticaceae & 蕁麻科 & \textit{Pilea elliptifolia}  & 橢圓葉冷水麻 & D2 \index{Pilea@\textit{Pilea}!elliptifolia@\textit{elliptifolia}}  \index{橢圓葉冷水麻} \\
    Urticaceae & 蕁麻科 & \textit{Pilea japonica}  & 日本冷水麻 & B2ab(ii) \index{Pilea@\textit{Pilea}!japonica@\textit{japonica}}  \index{日本冷水麻} \\
    Violaceae & 菫菜科 & \textit{Viola obtusa} var. \textit{tsuifengensis}  & 翠峰菫菜 & D1+2 \index{Viola@\textit{Viola}!obtusa@\textit{obtusa}!var. tsuifengensis@var. \textit{tsuifengensis}}  \index{翠峰菫菜} \\
    Zingiberaceae & 薑科 & \textit{Vanoverberghia sasakiana}  & 蘭嶼法氏薑 & D1 \index{Vanoverberghia@\textit{Vanoverberghia}!sasakiana@\textit{sasakiana}}  \index{蘭嶼法氏薑} \\
    \bottomrule
        \end{longtable}
    %%\end{table}
        }
    