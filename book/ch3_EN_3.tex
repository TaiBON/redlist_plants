\noindent\normalfont\selectfont Angiosperms 被子植物
\footnotesize\selectfont
%\begin{table}[!h]
        {\def\arraystretch{1.5}\tabcolsep=2pt
        \begin{longtable}{p{2.5cm}p{2.5cm}p{4.5cm}p{2.5cm}p{3cm}}
        \toprule
          科名 & 科中名 & 分類群學名 & 分類群中名 & 評估標準 \\
        \midrule 
        \endfirsthead

        {{\bfseries 續前頁 }} \\
        科名 & 科中名 & 分類群學名 & 分類群中名 & 評估標準 \\
        \midrule
        \endhead
                Acanthaceae & 爵床科 & \textit{Hygrophila pogonocalyx}  & 大安水蓑衣 & D \index{Hygrophila@\textit{Hygrophila}!pogonocalyx@\textit{pogonocalyx}}  \index{大安水蓑衣} \\
    Acanthaceae & 爵床科 & \textit{Hygrophila polysperma}  & 小獅子草 & D \index{Hygrophila@\textit{Hygrophila}!polysperma@\textit{polysperma}}  \index{小獅子草} \\
    Acanthaceae & 爵床科 & \textit{Strobilanthes lanyuensis}  & 蘭嶼馬藍 & D \index{Strobilanthes@\textit{Strobilanthes}!lanyuensis@\textit{lanyuensis}}  \index{蘭嶼馬藍} \\
    Alismataceae & 澤瀉科 & \textit{Sagittaria guayanensis} subsp. \textit{lappula}  & 臺灣冠果草 & B2b(iii,iv)c(iii,iv); C1 \index{Sagittaria@\textit{Sagittaria}!guayanensis@\textit{guayanensis}!subsp. lappula@subsp. \textit{lappula}}  \index{臺灣冠果草} \\
    Amaryllidaceae & 石蒜科 & \textit{Lycoris aurea}  & 龍爪花 & A2acd \index{Lycoris@\textit{Lycoris}!aurea@\textit{aurea}}  \index{龍爪花} \\
    Anacardiaceae & 漆樹科 & \textit{Semecarpus cuneiformis}  & 鈍葉大果漆 & B2ab(i) \index{Semecarpus@\textit{Semecarpus}!cuneiformis@\textit{cuneiformis}}  \index{鈍葉大果漆} \\
    Apiaceae & 繖形科 & \textit{Angelica tarokoensis}  & 太魯閣當歸 & D \index{Angelica@\textit{Angelica}!tarokoensis@\textit{tarokoensis}}  \index{太魯閣當歸} \\
    Apiaceae & 繖形科 & \textit{Bupleurum kaoi}  & 高氏柴胡 & D \index{Bupleurum@\textit{Bupleurum}!kaoi@\textit{kaoi}}  \index{高氏柴胡} \\
    Apiaceae & 繖形科 & \textit{Chaerophyllum nanhuense}  & 南湖山薰香 & C2a(ii)b \index{Chaerophyllum@\textit{Chaerophyllum}!nanhuense@\textit{nanhuense}}  \index{南湖山薰香} \\
    Apocynaceae & 夾竹桃科 & \textit{Cynanchum atratum}  & 牛皮消 & C2a(ii) \index{Cynanchum@\textit{Cynanchum}!atratum@\textit{atratum}}  \index{牛皮消} \\
    Araceae & 天南星科 & \textit{Amorphophallus paeoniifolius}  & 疣柄魔芋 & D \index{Amorphophallus@\textit{Amorphophallus}!paeoniifolius@\textit{paeoniifolius}}  \index{疣柄魔芋} \\
    Araliaceae & 五加科 & \textit{Dendropanax trifidus}  & 三菱果樹參 & B2a; D \index{Dendropanax@\textit{Dendropanax}!trifidus@\textit{trifidus}}  \index{三菱果樹參} \\
    Arecaceae & 棕櫚科 & \textit{Pinanga tashiroi}  & 山檳榔 & D \index{Pinanga@\textit{Pinanga}!tashiroi@\textit{tashiroi}}  \index{山檳榔} \\
    Aristolochiaceae & 馬兜鈴科 & \textit{Asarum chatienshanianum}  & 插天山細辛 & B2b(ii, iii) \index{Asarum@\textit{Asarum}!chatienshanianum@\textit{chatienshanianum}}  \index{插天山細辛} \\
    Aristolochiaceae & 馬兜鈴科 & \textit{Asarum satsumense}  & 薩摩細辛 & B2ab(ii,iii) \index{Asarum@\textit{Asarum}!satsumense@\textit{satsumense}}  \index{薩摩細辛} \\
    Aristolochiaceae & 馬兜鈴科 & \textit{Asarum villisepalum}  & 神秘湖細辛 & B2ab(ii,iii) \index{Asarum@\textit{Asarum}!villisepalum@\textit{villisepalum}}  \index{神秘湖細辛} \\
    Aristolochiaceae & 馬兜鈴科 & \textit{Asarum yaeyamense}  & 八重山細辛 & B2ab(ii,iii,iv) \index{Asarum@\textit{Asarum}!yaeyamense@\textit{yaeyamense}}  \index{八重山細辛} \\
    Asteraceae & 菊科 & \textit{Aster altaicus}  & 臺東鐵桿蒿 & C2b; D \index{Aster@\textit{Aster}!altaicus@\textit{altaicus}}  \index{臺東鐵桿蒿} \\
    Asteraceae & 菊科 & \textit{Aster chingshuiensis}  & 清水馬蘭 & B2ab(ii) \index{Aster@\textit{Aster}!chingshuiensis@\textit{chingshuiensis}}  \index{清水馬蘭} \\
    Asteraceae & 菊科 & \textit{Aster morrisonensis}  & 玉山鐵桿蒿 & B2ab(ii) \index{Aster@\textit{Aster}!morrisonensis@\textit{morrisonensis}}  \index{玉山鐵桿蒿} \\
    Asteraceae & 菊科 & \textit{Aster ovalifolius}  & 臺灣紺菊 & B2ab(iii) \index{Aster@\textit{Aster}!ovalifolius@\textit{ovalifolius}}  \index{臺灣紺菊} \\
    Asteraceae & 菊科 & \textit{Aster taoyuenensis}  & 桃園馬蘭 & D \index{Aster@\textit{Aster}!taoyuenensis@\textit{taoyuenensis}}  \index{桃園馬蘭} \\
    Asteraceae & 菊科 & \textit{Cirsium lineare}  & 華蓟 & B2ab(ii, iii)c(iv): D \index{Cirsium@\textit{Cirsium}!lineare@\textit{lineare}}  \index{華蓟} \\
    Asteraceae & 菊科 & \textit{Dendranthema lavandulifolium} var. \textit{tomentellum}  & 新竹油菊 & D \index{Dendranthema@\textit{Dendranthema}!lavandulifolium@\textit{lavandulifolium}!var. tomentellum@var. \textit{tomentellum}}  \index{新竹油菊} \\
    Asteraceae & 菊科 & \textit{Parasenecio nokoensis}  & 能高蟹甲草 & B2ab(ii) \index{Parasenecio@\textit{Parasenecio}!nokoensis@\textit{nokoensis}}  \index{能高蟹甲草} \\
    Asteraceae & 菊科 & \textit{Pertya simozawai}  & 半高野帚 & A1acd; B2ab(iii); C2a(i) \index{Pertya@\textit{Pertya}!simozawai@\textit{simozawai}}  \index{半高野帚} \\
    Asteraceae & 菊科 & \textit{Senecio kuanshanensis}  & 關山千里光 & D \index{Senecio@\textit{Senecio}!kuanshanensis@\textit{kuanshanensis}}  \index{關山千里光} \\
    Asteraceae & 菊科 & \textit{Senecio tarokoensis}  & 太魯閣千里光 & B1ab(ii, iv, v); D \index{Senecio@\textit{Senecio}!tarokoensis@\textit{tarokoensis}}  \index{太魯閣千里光} \\
    Asteraceae & 菊科 & \textit{Taraxacum formosanum}  & 臺灣蒲公英 & A2cd \index{Taraxacum@\textit{Taraxacum}!formosanum@\textit{formosanum}}  \index{臺灣蒲公英} \\
    Balanophoraceae & 蛇菰科 & \textit{Balanophora yakushimensis}  & 屋久島蛇菰 & B2ac; C2a; D1 \index{Balanophora@\textit{Balanophora}!yakushimensis@\textit{yakushimensis}}  \index{屋久島蛇菰} \\
    Begoniaceae & 秋海棠科 & \textit{Begonia bouffordii}  & 九九峰秋海棠 & B2ac(iv) \index{Begonia@\textit{Begonia}!bouffordii@\textit{bouffordii}}  \index{九九峰秋海棠} \\
    Brassicaceae & 十字花科 & \textit{Arabidopsis halleri} var. \textit{gemmifera}  & 葉芽筷子芥 & B2ab(v) \index{Arabidopsis@\textit{Arabidopsis}!halleri@\textit{halleri}!var. gemmifera@var. \textit{gemmifera}}  \index{葉芽筷子芥} \\
    Brassicaceae & 十字花科 & \textit{Barbarea taiwaniana}  & 臺灣山芥菜 & D \index{Barbarea@\textit{Barbarea}!taiwaniana@\textit{taiwaniana}}  \index{臺灣山芥菜} \\
    Brassicaceae & 十字花科 & \textit{Draba sekiyana}  & 臺灣山薺 & D \index{Draba@\textit{Draba}!sekiyana@\textit{sekiyana}}  \index{臺灣山薺} \\
    Burmanniaceae & 水玉簪科 & \textit{Burmannia championii}  & 頭花水玉簪 & D \index{Burmannia@\textit{Burmannia}!championii@\textit{championii}}  \index{頭花水玉簪} \\
    Burmanniaceae & 水玉簪科 & \textit{Gymnosiphon aphyllus}  & 小水玉簪 & D \index{Gymnosiphon@\textit{Gymnosiphon}!aphyllus@\textit{aphyllus}}  \index{小水玉簪} \\
    Calophyllaceae & 胡桐科 & \textit{Calophyllum blancoi}  & 蘭嶼胡桐 & C2a(i) \index{Calophyllum@\textit{Calophyllum}!blancoi@\textit{blancoi}}  \index{蘭嶼胡桐} \\
    Capparaceae & 山柑科 & \textit{Capparis pubiflora}  & 毛花山柑 & D \index{Capparis@\textit{Capparis}!pubiflora@\textit{pubiflora}}  \index{毛花山柑} \\
    Caryophyllaceae & 石竹科 & \textit{Silene fortunei} var. \textit{kiruninsularis}  & 基隆蠅子草 & D \index{Silene@\textit{Silene}!fortunei@\textit{fortunei}!var. kiruninsularis@var. \textit{kiruninsularis}}  \index{基隆蠅子草} \\
    Celastraceae & 衛矛科 & \textit{Euonymus pallidifolia}  & 淡綠葉衛矛 & B2ab(iv) \index{Euonymus@\textit{Euonymus}!pallidifolia@\textit{pallidifolia}}  \index{淡綠葉衛矛} \\
    Clusiaceae & 藤黃科 & \textit{Garcinia subelliptica}  & 菲島福木 & B2ab(ii,v) \index{Garcinia@\textit{Garcinia}!subelliptica@\textit{subelliptica}}  \index{菲島福木} \\
    Coldeniaceae & 生果草科 & \textit{Coldenia procumbens}  & 臥莖同籬生果草 & B2; D \index{Coldenia@\textit{Coldenia}!procumbens@\textit{procumbens}}  \index{臥莖同籬生果草} \\
    Convolvulaceae & 旋花科 & \textit{Ipomoea sumatrana}  & 蘇門答臘牽牛 & C2b \index{Ipomoea@\textit{Ipomoea}!sumatrana@\textit{sumatrana}}  \index{蘇門答臘牽牛} \\
    Cordiaceae & 破布子科 & \textit{Cordia aspera} subsp. \textit{kanehirai}  & 金平氏破布子 & B1ab(iii) \index{Cordia@\textit{Cordia}!aspera@\textit{aspera}!subsp. kanehirai@subsp. \textit{kanehirai}}  \index{金平氏破布子} \\
    Crassulaceae & 景天科 & \textit{Sedum nokoense}  & 能高佛甲草 & B2ab(iii); C2a(i) \index{Sedum@\textit{Sedum}!nokoense@\textit{nokoense}}  \index{能高佛甲草} \\
    Cyperaceae & 莎草科 & \textit{Carex laticeps}  & 彎喙薹 & D1 \index{Carex@\textit{Carex}!laticeps@\textit{laticeps}}  \index{彎喙薹} \\
    Cyperaceae & 莎草科 & \textit{Carex scaposa}  & 花葶薹草 & D \index{Carex@\textit{Carex}!scaposa@\textit{scaposa}}  \index{花葶薹草} \\
    Cyperaceae & 莎草科 & \textit{Cladium jamaicense}  & 克拉莎 & A2d(ii) \index{Cladium@\textit{Cladium}!jamaicense@\textit{jamaicense}}  \index{克拉莎} \\
    Cyperaceae & 莎草科 & \textit{Eleocharis atropurpurea}  & 黑果藺 & B1ac(iii,iv) \index{Eleocharis@\textit{Eleocharis}!atropurpurea@\textit{atropurpurea}}  \index{黑果藺} \\
    Cyperaceae & 莎草科 & \textit{Eleocharis ochrostachys}  & 日月潭藺 & B2ab(iii) \index{Eleocharis@\textit{Eleocharis}!ochrostachys@\textit{ochrostachys}}  \index{日月潭藺} \\
    Cyperaceae & 莎草科 & \textit{Fimbristylis stolonifera}  & 匍匐莖飄拂草 & D1 \index{Fimbristylis@\textit{Fimbristylis}!stolonifera@\textit{stolonifera}}  \index{匍匐莖飄拂草} \\
    Cyperaceae & 莎草科 & \textit{Rhynchospora alba}  & 白穗刺子莞 & D \index{Rhynchospora@\textit{Rhynchospora}!alba@\textit{alba}}  \index{白穗刺子莞} \\
    Droseraceae & 茅膏菜科 & \textit{Drosera indica}  & 長葉茅膏菜 & A2c;B2b(ii)c(iv) \index{Drosera@\textit{Drosera}!indica@\textit{indica}}  \index{長葉茅膏菜} \\
    Ebenaceae & 柿樹科 & \textit{Diospyros kotoensis}  & 蘭嶼柿 & B1ab(iii) \index{Diospyros@\textit{Diospyros}!kotoensis@\textit{kotoensis}}  \index{蘭嶼柿} \\
    Elaeagnaceae & 胡頹子科 & \textit{Elaeagnus ohashii}  & 大橋胡頹子 & B2a, D \index{Elaeagnus@\textit{Elaeagnus}!ohashii@\textit{ohashii}}  \index{大橋胡頹子} \\
    Euphorbiaceae & 大戟科 & \textit{Euphorbia shouanensis}  & 霞山大戟 & D \index{Euphorbia@\textit{Euphorbia}!shouanensis@\textit{shouanensis}}  \index{霞山大戟} \\
    Euphorbiaceae & 大戟科 & \textit{Euphorbia tarokoensis}  & 太魯閣大戟 & B2ab(i,ii) \index{Euphorbia@\textit{Euphorbia}!tarokoensis@\textit{tarokoensis}}  \index{太魯閣大戟} \\
    Fabaceae & 豆科 & \textit{Albizia retusa}  & 蘭嶼合歡 & D \index{Albizia@\textit{Albizia}!retusa@\textit{retusa}}  \index{蘭嶼合歡} \\
    Fabaceae & 豆科 & \textit{Apios taiwanianus}  & 臺灣土圞兒 & D \index{Apios@\textit{Apios}!taiwanianus@\textit{taiwanianus}}  \index{臺灣土圞兒} \\
    Fabaceae & 豆科 & \textit{Astragalus nankotaizanensis}  & 南湖大山紫雲英 & D \index{Astragalus@\textit{Astragalus}!nankotaizanensis@\textit{nankotaizanensis}}  \index{南湖大山紫雲英} \\
    Fabaceae & 豆科 & \textit{Crotalaria similes}  & 鵝鑾鼻野百合 & D \index{Crotalaria@\textit{Crotalaria}!similes@\textit{similes}}  \index{鵝鑾鼻野百合} \\
    Fabaceae & 豆科 & \textit{Desmodium renifolium}  & 腎葉山螞蝗 & B2ab(iii) \index{Desmodium@\textit{Desmodium}!renifolium@\textit{renifolium}}  \index{腎葉山螞蝗} \\
    Fabaceae & 豆科 & \textit{Entada koshunensis}  & 恆春鴨腱藤 & D \index{Entada@\textit{Entada}!koshunensis@\textit{koshunensis}}  \index{恆春鴨腱藤} \\
    Fabaceae & 豆科 & \textit{Indigofera pedicellata}  & 長梗木藍 & B2ab(iii) \index{Indigofera@\textit{Indigofera}!pedicellata@\textit{pedicellata}}  \index{長梗木藍} \\
    Fabaceae & 豆科 & \textit{Indigofera ramulosissima}  & 太魯閣木藍 & D \index{Indigofera@\textit{Indigofera}!ramulosissima@\textit{ramulosissima}}  \index{太魯閣木藍} \\
    Fagaceae & 殼斗科 & \textit{Castanopsis eyrei}  & 反刺苦櫧 & B1ab(i,ii,v) \index{Castanopsis@\textit{Castanopsis}!eyrei@\textit{eyrei}}  \index{反刺苦櫧} \\
    Fagaceae & 殼斗科 & \textit{Lithocarpus chiaratuangensis}  & 加拉段柯 & C2a(i) \index{Lithocarpus@\textit{Lithocarpus}!chiaratuangensis@\textit{chiaratuangensis}}  \index{加拉段柯} \\
    Fagaceae & 殼斗科 & \textit{Lithocarpus shinsuiensis}  & 浸水營石櫟 & C2a(i) \index{Lithocarpus@\textit{Lithocarpus}!shinsuiensis@\textit{shinsuiensis}}  \index{浸水營石櫟} \\
    Fagaceae & 殼斗科 & \textit{Quercus glandulifera} var. \textit{brevipetiolata}  & 思茅櫧櫟 & B1ab(iii,v) \index{Quercus@\textit{Quercus}!glandulifera@\textit{glandulifera}!var. brevipetiolata@var. \textit{brevipetiolata}}  \index{思茅櫧櫟} \\
    Gentianaceae & 龍膽科 & \textit{Gentiana horaimontana}  & 高山龍膽 & B1ac(iii) \index{Gentiana@\textit{Gentiana}!horaimontana@\textit{horaimontana}}  \index{高山龍膽} \\
    Gentianaceae & 龍膽科 & \textit{Swertia changii}  & 大漢山當藥 & B2ac(iv) \index{Swertia@\textit{Swertia}!changii@\textit{changii}}  \index{大漢山當藥} \\
    Gesneriaceae & 苦苣苔科 & \textit{Epithema taiwanense} var. \textit{taiwanense}  & 臺灣苣苔 & B2ab(i, ii) \index{Epithema@\textit{Epithema}!taiwanense@\textit{taiwanense}!var. taiwanense@var. \textit{taiwanense}}  \index{臺灣苣苔} \\
    Gesneriaceae & 苦苣苔科 & \textit{Lysionotus pauciflorus} var. \textit{ikedae}  & 蘭嶼石吊蘭 & D \index{Lysionotus@\textit{Lysionotus}!pauciflorus@\textit{pauciflorus}!var. ikedae@var. \textit{ikedae}}  \index{蘭嶼石吊蘭} \\
    Hamamelidaceae & 金縷梅科 & \textit{Corylopsis stenopetala}  & 臺灣瑞木 & D \index{Corylopsis@\textit{Corylopsis}!stenopetala@\textit{stenopetala}}  \index{臺灣瑞木} \\
    Hamamelidaceae & 金縷梅科 & \textit{Distyliopsis dunnii}  & 尖葉水絲梨 & D \index{Distyliopsis@\textit{Distyliopsis}!dunnii@\textit{dunnii}}  \index{尖葉水絲梨} \\
    Juncaceae & 燈心草科 & \textit{Juncus ohwianus}  & 大井氏燈心草 &  \index{Juncus@\textit{Juncus}!ohwianus@\textit{ohwianus}}  \index{大井氏燈心草} \\
    Lamiaceae & 唇形科 & \textit{Lycopus lucidus}  & 地筍 & A1ac \index{Lycopus@\textit{Lycopus}!lucidus@\textit{lucidus}}  \index{地筍} \\
    Lamiaceae & 唇形科 & \textit{Pogostemon stellatus}  & 水虎尾 & A4c \index{Pogostemon@\textit{Pogostemon}!stellatus@\textit{stellatus}}  \index{水虎尾} \\
    Lamiaceae & 唇形科 & \textit{Salvia tashiroi}  & 田代氏鼠尾草 & B2ab(ii,v) \index{Salvia@\textit{Salvia}!tashiroi@\textit{tashiroi}}  \index{田代氏鼠尾草} \\
    Lauraceae & 樟科 & \textit{Cinnamomum austrosinense}  & 牡丹葉桂皮 & C2a(i) \index{Cinnamomum@\textit{Cinnamomum}!austrosinense@\textit{austrosinense}}  \index{牡丹葉桂皮} \\
    Lauraceae & 樟科 & \textit{Cinnamomum kanehirae}  & 牛樟 & A1acd \index{Cinnamomum@\textit{Cinnamomum}!kanehirae@\textit{kanehirae}}  \index{牛樟} \\
    Lauraceae & 樟科 & \textit{Neolitsea sericea} var. \textit{aurata}  & 金新木薑子 & B1ab(iii,v); C2a(ii) \index{Neolitsea@\textit{Neolitsea}!sericea@\textit{sericea}!var. aurata@var. \textit{aurata}}  \index{金新木薑子} \\
    Lentibulariaceae & 狸藻科 & \textit{Utricularia aurea}  & 黃花狸藻 & B2ab(iii)c(ii) \index{Utricularia@\textit{Utricularia}!aurea@\textit{aurea}}  \index{黃花狸藻} \\
    Lentibulariaceae & 狸藻科 & \textit{Utricularia bifida}  & 挖耳草 & B2b(iii)c(ii) \index{Utricularia@\textit{Utricularia}!bifida@\textit{bifida}}  \index{挖耳草} \\
    Lentibulariaceae & 狸藻科 & \textit{Utricularia heterosepala}  & 異萼挖耳草 & B2ac(ii); D \index{Utricularia@\textit{Utricularia}!heterosepala@\textit{heterosepala}}  \index{異萼挖耳草} \\
    Linderniaceae & 母草科 & \textit{Lindernia scutellariiformis}  & 臺南見風紅 & B2ab(i,iii) \index{Lindernia@\textit{Lindernia}!scutellariiformis@\textit{scutellariiformis}}  \index{臺南見風紅} \\
    Loganiaceae & 馬錢科 & \textit{Gardneria nutans}  & 垂花蓬萊葛 & D \index{Gardneria@\textit{Gardneria}!nutans@\textit{nutans}}  \index{垂花蓬萊葛} \\
    Malpighiaceae & 黃褥花科 & \textit{Tristellateia australasiae}  & 三星果藤 & D \index{Tristellateia@\textit{Tristellateia}!australasiae@\textit{australasiae}}  \index{三星果藤} \\
    Malvaceae & 錦葵科 & \textit{Heritiera littoralis}  & 銀葉樹 & B2a; D \index{Heritiera@\textit{Heritiera}!littoralis@\textit{littoralis}}  \index{銀葉樹} \\
    Malvaceae & 錦葵科 & \textit{Hibiscus surattensis}  & 刺芙蓉 & C2a(i) \index{Hibiscus@\textit{Hibiscus}!surattensis@\textit{surattensis}}  \index{刺芙蓉} \\
    Malvaceae & 錦葵科 & \textit{Thespesia populnea}  & 繖楊 & D \index{Thespesia@\textit{Thespesia}!populnea@\textit{populnea}}  \index{繖楊} \\
    Melastomataceae & 野牡丹科 & \textit{Medinilla hayatana}  & 蘭嶼野牡丹藤 & B2ab(iv) \index{Medinilla@\textit{Medinilla}!hayatana@\textit{hayatana}}  \index{蘭嶼野牡丹藤} \\
    Melastomataceae & 野牡丹科 & \textit{Memecylon pendulum}  & 垂枝羊角扭 & D \index{Memecylon@\textit{Memecylon}!pendulum@\textit{pendulum}}  \index{垂枝羊角扭} \\
    Menispermaceae & 防己科 & \textit{Sinomenium acutum}  & 漢防己 & B2ab(ii,v); D \index{Sinomenium@\textit{Sinomenium}!acutum@\textit{acutum}}  \index{漢防己} \\
    Menispermaceae & 防己科 & \textit{Tinospora dentata}  & 恆春青牛膽 & D \index{Tinospora@\textit{Tinospora}!dentata@\textit{dentata}}  \index{恆春青牛膽} \\
    Menyanthaceae & 睡菜科 & \textit{Nymphoides indica}  & 印度莕菜 & B1b(ii,v)c(ii) \index{Nymphoides@\textit{Nymphoides}!indica@\textit{indica}}  \index{印度莕菜} \\
    Mitrastemonaceae & 奴草科 & \textit{Mitrastemon yamamotoi} var. \textit{kanehirai}  & 菱形奴草 & B2 \index{Mitrastemon@\textit{Mitrastemon}!yamamotoi@\textit{yamamotoi}!var. kanehirai@var. \textit{kanehirai}}  \index{菱形奴草} \\
    Myricaceae & 楊梅科 & \textit{Myrica adenophora}  & 青楊梅 & A4d, D \index{Myrica@\textit{Myrica}!adenophora@\textit{adenophora}}  \index{青楊梅} \\
    Myristicaceae & 肉豆蔻科 & \textit{Myristica elliptica} var. \textit{simiarum}  & 紅頭肉豆蔻 & D \index{Myristica@\textit{Myristica}!elliptica@\textit{elliptica}!var. simiarum@var. \textit{simiarum}}  \index{紅頭肉豆蔻} \\
    Olacaceae & 鐵青樹科 & \textit{Olax imbricata}  & 菲律賓鐵青樹 & D \index{Olax@\textit{Olax}!imbricata@\textit{imbricata}}  \index{菲律賓鐵青樹} \\
    Oleaceae & 木犀科 & \textit{Chionanthus retusus}  & 流蘇樹 & C2a(i) \index{Chionanthus@\textit{Chionanthus}!retusus@\textit{retusus}}  \index{流蘇樹} \\
    Orchidaceae & 蘭科 & \textit{Acanthephippium pictum}  & 延齡罈花蘭 & D \index{Acanthephippium@\textit{Acanthephippium}!pictum@\textit{pictum}}  \index{延齡罈花蘭} \\
    Orchidaceae & 蘭科 & \textit{Bulbophyllum ciliisepalum}  & 毛緣萼豆蘭 & B2ab(iii,v) \index{Bulbophyllum@\textit{Bulbophyllum}!ciliisepalum@\textit{ciliisepalum}}  \index{毛緣萼豆蘭} \\
    Orchidaceae & 蘭科 & \textit{Bulbophyllum griffithii}  & 溪頭豆蘭 & D \index{Bulbophyllum@\textit{Bulbophyllum}!griffithii@\textit{griffithii}}  \index{溪頭豆蘭} \\
    Orchidaceae & 蘭科 & \textit{Bulbophyllum kuanwuense} var. \textit{kuanwuense}  & 觀霧豆蘭 & B2ab(iii,iv,v) \index{Bulbophyllum@\textit{Bulbophyllum}!kuanwuense@\textit{kuanwuense}!var. kuanwuense@var. \textit{kuanwuense}}  \index{觀霧豆蘭} \\
    Orchidaceae & 蘭科 & \textit{Bulbophyllum pingtungensis}  & 大花豆蘭 & A2cd \index{Bulbophyllum@\textit{Bulbophyllum}!pingtungensis@\textit{pingtungensis}}  \index{大花豆蘭} \\
    Orchidaceae & 蘭科 & \textit{Bulbophyllum taiwanense}  & 臺灣捲瓣蘭 & A2cd \index{Bulbophyllum@\textit{Bulbophyllum}!taiwanense@\textit{taiwanense}}  \index{臺灣捲瓣蘭} \\
    Orchidaceae & 蘭科 & \textit{Bulbophyllum tokioi}  & 小葉豆蘭 & D1 \index{Bulbophyllum@\textit{Bulbophyllum}!tokioi@\textit{tokioi}}  \index{小葉豆蘭} \\
    Orchidaceae & 蘭科 & \textit{Cleisostoma uraiensis}  & 烏來閉口蘭 & D \index{Cleisostoma@\textit{Cleisostoma}!uraiensis@\textit{uraiensis}}  \index{烏來閉口蘭} \\
    Orchidaceae & 蘭科 & \textit{Collabium chinense}  & 柯麗白蘭 & D \index{Collabium@\textit{Collabium}!chinense@\textit{chinense}}  \index{柯麗白蘭} \\
    Orchidaceae & 蘭科 & \textit{Corybas sinii}  & 辛氏盔蘭 & B2ab(iii) \index{Corybas@\textit{Corybas}!sinii@\textit{sinii}}  \index{辛氏盔蘭} \\
    Orchidaceae & 蘭科 & \textit{Corybas taiwanesis}  & 紅盔蘭 & D \index{Corybas@\textit{Corybas}!taiwanesis@\textit{taiwanesis}}  \index{紅盔蘭} \\
    Orchidaceae & 蘭科 & \textit{Corybas taliensis}  & 杉林溪盔蘭 & D \index{Corybas@\textit{Corybas}!taliensis@\textit{taliensis}}  \index{杉林溪盔蘭} \\
    Orchidaceae & 蘭科 & \textit{Cymbidium cochleare}  & 香莎草蘭 & D \index{Cymbidium@\textit{Cymbidium}!cochleare@\textit{cochleare}}  \index{香莎草蘭} \\
    Orchidaceae & 蘭科 & \textit{Cypripedium formosanum}  & 臺灣喜普鞋蘭 & D \index{Cypripedium@\textit{Cypripedium}!formosanum@\textit{formosanum}}  \index{臺灣喜普鞋蘭} \\
    Orchidaceae & 蘭科 & \textit{Cypripedium macranthum}  & 奇萊喜普鞋蘭 & D \index{Cypripedium@\textit{Cypripedium}!macranthum@\textit{macranthum}}  \index{奇萊喜普鞋蘭} \\
    Orchidaceae & 蘭科 & \textit{Cyrtosia javanica}  & 肉果蘭 & B2ac(iii,iv); D \index{Cyrtosia@\textit{Cyrtosia}!javanica@\textit{javanica}}  \index{肉果蘭} \\
    Orchidaceae & 蘭科 & \textit{Dendrobium equitans}  & 燕子石斛 & B1ab(iii,iv) \index{Dendrobium@\textit{Dendrobium}!equitans@\textit{equitans}}  \index{燕子石斛} \\
    Orchidaceae & 蘭科 & \textit{Dendrobium furcatopedicellatum}  & 雙花石斛 & D \index{Dendrobium@\textit{Dendrobium}!furcatopedicellatum@\textit{furcatopedicellatum}}  \index{雙花石斛} \\
    Orchidaceae & 蘭科 & \textit{Disperis siamensis}  & 雙袋蘭 & D \index{Disperis@\textit{Disperis}!siamensis@\textit{siamensis}}  \index{雙袋蘭} \\
    Orchidaceae & 蘭科 & \textit{Erythrodes triantherae} var. \textit{triantherae}  & 三葯細筆蘭 & D \index{Erythrodes@\textit{Erythrodes}!triantherae@\textit{triantherae}!var. triantherae@var. \textit{triantherae}}  \index{三葯細筆蘭} \\
    Orchidaceae & 蘭科 & \textit{Gastrodia appendiculata}  & 無蕊喙赤箭 & B1ac(iv)+2ac(iv); D \index{Gastrodia@\textit{Gastrodia}!appendiculata@\textit{appendiculata}}  \index{無蕊喙赤箭} \\
    Orchidaceae & 蘭科 & \textit{Gastrodia elata}  & 高赤箭 & D \index{Gastrodia@\textit{Gastrodia}!elata@\textit{elata}}  \index{高赤箭} \\
    Orchidaceae & 蘭科 & \textit{Gastrodia javanica}  & 爪哇赤箭 & B2ab(iii) \index{Gastrodia@\textit{Gastrodia}!javanica@\textit{javanica}}  \index{爪哇赤箭} \\
    Orchidaceae & 蘭科 & \textit{Goodyera biflora}  & 大花斑葉蘭 & D \index{Goodyera@\textit{Goodyera}!biflora@\textit{biflora}}  \index{大花斑葉蘭} \\
    Orchidaceae & 蘭科 & \textit{Goodyera repens}  & 南投斑葉蘭 & D \index{Goodyera@\textit{Goodyera}!repens@\textit{repens}}  \index{南投斑葉蘭} \\
    Orchidaceae & 蘭科 & \textit{Lecanorchis thalassica}  & 紋皿柱蘭 & D \index{Lecanorchis@\textit{Lecanorchis}!thalassica@\textit{thalassica}}  \index{紋皿柱蘭} \\
    Orchidaceae & 蘭科 & \textit{Lecanorchis triloba}  & 三裂皿蘭 & D \index{Lecanorchis@\textit{Lecanorchis}!triloba@\textit{triloba}}  \index{三裂皿蘭} \\
    Orchidaceae & 蘭科 & \textit{Liparis barbata}  & 鬚唇羊耳蒜 & B2ab(iii) \index{Liparis@\textit{Liparis}!barbata@\textit{barbata}}  \index{鬚唇羊耳蒜} \\
    Orchidaceae & 蘭科 & \textit{Liparis caespitosa}  & 小花羊耳蒜 & C2a(i) \index{Liparis@\textit{Liparis}!caespitosa@\textit{caespitosa}}  \index{小花羊耳蒜} \\
    Orchidaceae & 蘭科 & \textit{Liparis campylostalix}  & 彎柱羊耳蒜 & D \index{Liparis@\textit{Liparis}!campylostalix@\textit{campylostalix}}  \index{彎柱羊耳蒜} \\
    Orchidaceae & 蘭科 & \textit{Liparis derchiensis}  & 德基羊耳蒜 & D \index{Liparis@\textit{Liparis}!derchiensis@\textit{derchiensis}}  \index{德基羊耳蒜} \\
    Orchidaceae & 蘭科 & \textit{Liparis japonica}  & 長穗羊耳蒜 & D \index{Liparis@\textit{Liparis}!japonica@\textit{japonica}}  \index{長穗羊耳蒜} \\
    Orchidaceae & 蘭科 & \textit{Liparis reckoniana}  & 雲頂羊耳蒜 & D \index{Liparis@\textit{Liparis}!reckoniana@\textit{reckoniana}}  \index{雲頂羊耳蒜} \\
    Orchidaceae & 蘭科 & \textit{Liparis somai}  & 高士佛羊耳蒜 & D \index{Liparis@\textit{Liparis}!somai@\textit{somai}}  \index{高士佛羊耳蒜} \\
    Orchidaceae & 蘭科 & \textit{Neottia kuanshanensis}  & 關山雙葉蘭 & D \index{Neottia@\textit{Neottia}!kuanshanensis@\textit{kuanshanensis}}  \index{關山雙葉蘭} \\
    Orchidaceae & 蘭科 & \textit{Neottia meifongensis}  & 梅峰雙葉蘭 & D \index{Neottia@\textit{Neottia}!meifongensis@\textit{meifongensis}}  \index{梅峰雙葉蘭} \\
    Orchidaceae & 蘭科 & \textit{Nephelaphyllum tenuiflorum}  & 雲葉蘭 & D \index{Nephelaphyllum@\textit{Nephelaphyllum}!tenuiflorum@\textit{tenuiflorum}}  \index{雲葉蘭} \\
    Orchidaceae & 蘭科 & \textit{Oberonia gigantea}  & 大莪白蘭 & C2a(i) \index{Oberonia@\textit{Oberonia}!gigantea@\textit{gigantea}}  \index{大莪白蘭} \\
    Orchidaceae & 蘭科 & \textit{Oberonia rosea}  & 裂瓣莪白蘭 & C2a(i) \index{Oberonia@\textit{Oberonia}!rosea@\textit{rosea}}  \index{裂瓣莪白蘭} \\
    Orchidaceae & 蘭科 & \textit{Oberonia segawae}  & 齒唇莪白蘭 & C2a(i) \index{Oberonia@\textit{Oberonia}!segawae@\textit{segawae}}  \index{齒唇莪白蘭} \\
    Orchidaceae & 蘭科 & \textit{Oberonia seidenfadenii}  & 密花小騎士蘭 & B2ab(iii) \index{Oberonia@\textit{Oberonia}!seidenfadenii@\textit{seidenfadenii}}  \index{密花小騎士蘭} \\
    Orchidaceae & 蘭科 & \textit{Oreorchis indica}  & 印度山蘭 & D \index{Oreorchis@\textit{Oreorchis}!indica@\textit{indica}}  \index{印度山蘭} \\
    Orchidaceae & 蘭科 & \textit{Peristylus gracilis}  & 纖細闊蕊蘭 & D \index{Peristylus@\textit{Peristylus}!gracilis@\textit{gracilis}}  \index{纖細闊蕊蘭} \\
    Orchidaceae & 蘭科 & \textit{Peristylus lacertifer} var. \textit{taipoensis}  & 短裂闊蕊蘭 & D \index{Peristylus@\textit{Peristylus}!lacertifer@\textit{lacertifer}!var. taipoensis@var. \textit{taipoensis}}  \index{短裂闊蕊蘭} \\
    Orchidaceae & 蘭科 & \textit{Phaius takeoi}  & 粗莖鶴頂蘭 & D \index{Phaius@\textit{Phaius}!takeoi@\textit{takeoi}}  \index{粗莖鶴頂蘭} \\
    Orchidaceae & 蘭科 & \textit{Platanthera sonoharae}  & 琉球蜻蛉蘭 & D \index{Platanthera@\textit{Platanthera}!sonoharae@\textit{sonoharae}}  \index{琉球蜻蛉蘭} \\
    Orchidaceae & 蘭科 & \textit{Pogonia minor}  & 小鬚唇蘭 & D \index{Pogonia@\textit{Pogonia}!minor@\textit{minor}}  \index{小鬚唇蘭} \\
    Orchidaceae & 蘭科 & \textit{Saccolabiopsis viridiflora} subsp. \textit{taiwaniana}  & 臺灣擬囊唇蘭 & B2ab(v) \index{Saccolabiopsis@\textit{Saccolabiopsis}!viridiflora@\textit{viridiflora}!subsp. taiwaniana@subsp. \textit{taiwaniana}}  \index{臺灣擬囊唇蘭} \\
    Orchidaceae & 蘭科 & \textit{Stereosandra javanica}  & 肉葯蘭 & D \index{Stereosandra@\textit{Stereosandra}!javanica@\textit{javanica}}  \index{肉葯蘭} \\
    Orchidaceae & 蘭科 & \textit{Thelasis pygmaea}  & 閉花八粉蘭 & D \index{Thelasis@\textit{Thelasis}!pygmaea@\textit{pygmaea}}  \index{閉花八粉蘭} \\
    Orchidaceae & 蘭科 & \textit{Thrixspermum annamense}  & 白毛風蘭 & D \index{Thrixspermum@\textit{Thrixspermum}!annamense@\textit{annamense}}  \index{白毛風蘭} \\
    Orchidaceae & 蘭科 & \textit{Thrixspermum eximium}  & 異色風蘭 & B2ab(v) \index{Thrixspermum@\textit{Thrixspermum}!eximium@\textit{eximium}}  \index{異色風蘭} \\
    Orchidaceae & 蘭科 & \textit{Thrixspermum merguense}  & 高士佛風蘭 & B2ab(iii) \index{Thrixspermum@\textit{Thrixspermum}!merguense@\textit{merguense}}  \index{高士佛風蘭} \\
    Orchidaceae & 蘭科 & \textit{Yoania japonica}  & 長花柄蘭 & D \index{Yoania@\textit{Yoania}!japonica@\textit{japonica}}  \index{長花柄蘭} \\
    Orobanchaceae & 列當科 & \textit{Melampyrum roseum}  & 山蘿花 & B2a\&; D \index{Melampyrum@\textit{Melampyrum}!roseum@\textit{roseum}}  \index{山蘿花} \\
    Orobanchaceae & 列當科 & \textit{Siphonostegia chinensis}  & 陰行草 & B2ab(iii) \index{Siphonostegia@\textit{Siphonostegia}!chinensis@\textit{chinensis}}  \index{陰行草} \\
    Paulowniaceae & 泡桐科 & \textit{Paulownia fortunei}  & 泡桐 & D \index{Paulownia@\textit{Paulownia}!fortunei@\textit{fortunei}}  \index{泡桐} \\
    Pentaphylacaceae & 五列木科 & \textit{Eurya rengechiensis}  & 蓮華池柃木 & B2ab(i); D \index{Eurya@\textit{Eurya}!rengechiensis@\textit{rengechiensis}}  \index{蓮華池柃木} \\
    Plantaginaceae & 車前科 & \textit{Limnophila fragrans}  & 無柄田香草 &  \index{Limnophila@\textit{Limnophila}!fragrans@\textit{fragrans}}  \index{無柄田香草} \\
    Plantaginaceae & 車前科 & \textit{Limnophila sessiliflora}  & 無柄花石龍尾 & B2ab(iii) \index{Limnophila@\textit{Limnophila}!sessiliflora@\textit{sessiliflora}}  \index{無柄花石龍尾} \\
    Plantaginaceae & 車前科 & \textit{Limnophila trichophylla}  & 石龍尾 & B2ab(iii) \index{Limnophila@\textit{Limnophila}!trichophylla@\textit{trichophylla}}  \index{石龍尾} \\
    Plantaginaceae & 車前科 & \textit{Veronicastrum formosanum}  & 臺灣腹水草 & D \index{Veronicastrum@\textit{Veronicastrum}!formosanum@\textit{formosanum}}  \index{臺灣腹水草} \\
    Plantaginaceae & 車前科 & \textit{Veronicastrum kitamurae}  & 高山腹水草 & D \index{Veronicastrum@\textit{Veronicastrum}!kitamurae@\textit{kitamurae}}  \index{高山腹水草} \\
    Poaceae & 禾本科 & \textit{Agrostis dimorpholemma}  & 多形翦穎 & B2ac(iv); C2b \index{Agrostis@\textit{Agrostis}!dimorpholemma@\textit{dimorpholemma}}  \index{多形翦穎} \\
    Poaceae & 禾本科 & \textit{Digitaria heterantha}  & 粗穗馬唐 & B2ac(ii) \index{Digitaria@\textit{Digitaria}!heterantha@\textit{heterantha}}  \index{粗穗馬唐} \\
    Poaceae & 禾本科 & \textit{Eragrostis unioloides}  & 牛虱草 & A4d; B2ac(ii,iii) \index{Eragrostis@\textit{Eragrostis}!unioloides@\textit{unioloides}}  \index{牛虱草} \\
    Poaceae & 禾本科 & \textit{Eriochloa villosa}  & 野黍 & B2ac(iii) \index{Eriochloa@\textit{Eriochloa}!villosa@\textit{villosa}}  \index{野黍} \\
    Poaceae & 禾本科 & \textit{Eulalia leschenaultiana}  & 細稈金茅 & B1ac(iv)+2ac(iv) \index{Eulalia@\textit{Eulalia}!leschenaultiana@\textit{leschenaultiana}}  \index{細稈金茅} \\
    Poaceae & 禾本科 & \textit{Festuca japonica}  & 日本羊茅 & B2ac(ii,iv); D \index{Festuca@\textit{Festuca}!japonica@\textit{japonica}}  \index{日本羊茅} \\
    Poaceae & 禾本科 & \textit{Hymenachne pseudointerrupta}  & 膜稃草 & A4c; B2ac(ii,iii,iv); C2b; D \index{Hymenachne@\textit{Hymenachne}!pseudointerrupta@\textit{pseudointerrupta}}  \index{膜稃草} \\
    Primulaceae & 報春花科 & \textit{Stimpsonia chamaedryoides}  & 施丁草 & B2ab(i,ii)c(iii,iv); D \index{Stimpsonia@\textit{Stimpsonia}!chamaedryoides@\textit{chamaedryoides}}  \index{施丁草} \\
    Ranunculaceae & 毛茛科 & \textit{Calathodes polycarpa}  & 多果雞爪草 & D \index{Calathodes@\textit{Calathodes}!polycarpa@\textit{polycarpa}}  \index{多果雞爪草} \\
    Ranunculaceae & 毛茛科 & \textit{Ranunculus morii}  & 森氏毛茛 & D \index{Ranunculus@\textit{Ranunculus}!morii@\textit{morii}}  \index{森氏毛茛} \\
    Ranunculaceae & 毛茛科 & \textit{Ranunculus nankotaizanus}  & 南湖毛茛 & D \index{Ranunculus@\textit{Ranunculus}!nankotaizanus@\textit{nankotaizanus}}  \index{南湖毛茛} \\
    Rhamnaceae & 鼠李科 & \textit{Berchemia fenchifuensis}  & 奮起湖黃鱔藤 & D \index{Berchemia@\textit{Berchemia}!fenchifuensis@\textit{fenchifuensis}}  \index{奮起湖黃鱔藤} \\
    Rhamnaceae & 鼠李科 & \textit{Colubrina asiatica}  & 亞洲濱棗 & A4a; D \index{Colubrina@\textit{Colubrina}!asiatica@\textit{asiatica}}  \index{亞洲濱棗} \\
    Rhamnaceae & 鼠李科 & \textit{Paliurus ramosissimus}  & 馬甲子 & A4 \index{Paliurus@\textit{Paliurus}!ramosissimus@\textit{ramosissimus}}  \index{馬甲子} \\
    Rhamnaceae & 鼠李科 & \textit{Rhamnus chingshuiensis} var. \textit{chingshuiensis}  & 清水鼠李 & D \index{Rhamnus@\textit{Rhamnus}!chingshuiensis@\textit{chingshuiensis}!var. chingshuiensis@var. \textit{chingshuiensis}}  \index{清水鼠李} \\
    Rosaceae & 薔薇科 & \textit{Aria alnifolia}  & 赤楊葉梨 & D \index{Aria@\textit{Aria}!alnifolia@\textit{alnifolia}}  \index{赤楊葉梨} \\
    Rosaceae & 薔薇科 & \textit{Cotoneaster bullatus}  & 泡葉栒子 & B1ab() \index{Cotoneaster@\textit{Cotoneaster}!bullatus@\textit{bullatus}}  \index{泡葉栒子} \\
    Rosaceae & 薔薇科 & \textit{Potentilla chinensis}  & 委陵菜 & A1a \index{Potentilla@\textit{Potentilla}!chinensis@\textit{chinensis}}  \index{委陵菜} \\
    Rosaceae & 薔薇科 & \textit{Potentilla nipponica}  & 日本翻白草 & B2ab(ii,iii); D \index{Potentilla@\textit{Potentilla}!nipponica@\textit{nipponica}}  \index{日本翻白草} \\
    Rosaceae & 薔薇科 & \textit{Potentilla tugitakensis}  & 雪山翻白草 & C2a(i) \index{Potentilla@\textit{Potentilla}!tugitakensis@\textit{tugitakensis}}  \index{雪山翻白草} \\
    Rosaceae & 薔薇科 & \textit{Prunus grisea}  & 蘭嶼野櫻花 & B1ab(iii)+2ab(iii) \index{Prunus@\textit{Prunus}!grisea@\textit{grisea}}  \index{蘭嶼野櫻花} \\
    Rosaceae & 薔薇科 & \textit{Prunus obtusata}  & 臺灣椆李 & B2ab(ii) \index{Prunus@\textit{Prunus}!obtusata@\textit{obtusata}}  \index{臺灣椆李} \\
    Rosaceae & 薔薇科 & \textit{Prunus phaeosticta} var. \textit{ilicifolia}  & 冬青葉桃仁 & B1ab() \index{Prunus@\textit{Prunus}!phaeosticta@\textit{phaeosticta}!var. ilicifolia@var. \textit{ilicifolia}}  \index{冬青葉桃仁} \\
    Rosaceae & 薔薇科 & \textit{Rosa laevigata}  & 金櫻子 & B1ab(iii)+2ab(iii); C2a(i); D1 \index{Rosa@\textit{Rosa}!laevigata@\textit{laevigata}}  \index{金櫻子} \\
    Rosaceae & 薔薇科 & \textit{Rubus sumatranus}  & 紅腺懸鉤子 & A4a; B1ab(iii); D \index{Rubus@\textit{Rubus}!sumatranus@\textit{sumatranus}}  \index{紅腺懸鉤子} \\
    Rubiaceae & 茜草科 & \textit{Galium nankotaizanum}  & 南湖大山豬殃殃 & D \index{Galium@\textit{Galium}!nankotaizanum@\textit{nankotaizanum}}  \index{南湖大山豬殃殃} \\
    Rubiaceae & 茜草科 & \textit{Galium tarokoense}  & 太魯閣豬殃殃 & D \index{Galium@\textit{Galium}!tarokoense@\textit{tarokoense}}  \index{太魯閣豬殃殃} \\
    Rubiaceae & 茜草科 & \textit{Ixora philippinensis}  & 小仙丹花 & D \index{Ixora@\textit{Ixora}!philippinensis@\textit{philippinensis}}  \index{小仙丹花} \\
    Rubiaceae & 茜草科 & \textit{Lasianthus chinensis}  & 白果雞屎樹 & A2acd \index{Lasianthus@\textit{Lasianthus}!chinensis@\textit{chinensis}}  \index{白果雞屎樹} \\
    Rubiaceae & 茜草科 & \textit{Lasianthus obliquinervis} var. \textit{simizui}  & 清水氏雞屎樹 & D \index{Lasianthus@\textit{Lasianthus}!obliquinervis@\textit{obliquinervis}!var. simizui@var. \textit{simizui}}  \index{清水氏雞屎樹} \\
    Rubiaceae & 茜草科 & \textit{Theligonum formosanum}  & 臺灣纖花草 & B2ab(iii) \index{Theligonum@\textit{Theligonum}!formosanum@\textit{formosanum}}  \index{臺灣纖花草} \\
    Rutaceae & 芸香科 & \textit{Zanthoxylum pistaciiflorum}  & 三葉花椒 & B2ab(v); C2a(i) \index{Zanthoxylum@\textit{Zanthoxylum}!pistaciiflorum@\textit{pistaciiflorum}}  \index{三葉花椒} \\
    Rutaceae & 芸香科 & \textit{Zanthoxylum simulans}  & 刺花椒 & B2ab(v); C2a(i) \index{Zanthoxylum@\textit{Zanthoxylum}!simulans@\textit{simulans}}  \index{刺花椒} \\
    Rutaceae & 芸香科 & \textit{Zanthoxylum wutaiense}  & 屏東花椒 & B2ab(v); C2a(i) \index{Zanthoxylum@\textit{Zanthoxylum}!wutaiense@\textit{wutaiense}}  \index{屏東花椒} \\
    Salicaceae & 楊柳科 & \textit{Salix kusanoi}  & 水社柳 & B2ab(v); D \index{Salix@\textit{Salix}!kusanoi@\textit{kusanoi}}  \index{水社柳} \\
    Salicaceae & 楊柳科 & \textit{Salix okamotoana}  & 關山嶺柳 & D \index{Salix@\textit{Salix}!okamotoana@\textit{okamotoana}}  \index{關山嶺柳} \\
    Sapotaceae & 山欖科 & \textit{Planchonella duclitan}  & 蘭嶼山欖 & B1ab(iii) \index{Planchonella@\textit{Planchonella}!duclitan@\textit{duclitan}}  \index{蘭嶼山欖} \\
    Scrophulariaceae & 玄參科 & \textit{Myoporum bontioides}  & 苦藍盤 & D \index{Myoporum@\textit{Myoporum}!bontioides@\textit{bontioides}}  \index{苦藍盤} \\
    Solanaceae & 茄科 & \textit{Physaliastrum chamaesarachoides}  & 林氏燈籠草 & B1ab(i); D \index{Physaliastrum@\textit{Physaliastrum}!chamaesarachoides@\textit{chamaesarachoides}}  \index{林氏燈籠草} \\
    Symplocaceae & 灰木科 & \textit{Symplocos koshunensis}  & 恆春灰木 & B2ab(i); D \index{Symplocos@\textit{Symplocos}!koshunensis@\textit{koshunensis}}  \index{恆春灰木} \\
    Triuridaceae & 霉草科 & \textit{Sciaphila arfakiana}  & 蘭嶼霉草 & D \index{Sciaphila@\textit{Sciaphila}!arfakiana@\textit{arfakiana}}  \index{蘭嶼霉草} \\
    Triuridaceae & 霉草科 & \textit{Sciaphila maculata}  & 斑點霉草 & D \index{Sciaphila@\textit{Sciaphila}!maculata@\textit{maculata}}  \index{斑點霉草} \\
    Triuridaceae & 霉草科 & \textit{Sciaphila ramosa}  & 多枝霉草 & D \index{Sciaphila@\textit{Sciaphila}!ramosa@\textit{ramosa}}  \index{多枝霉草} \\
    Triuridaceae & 霉草科 & \textit{Sciaphila secundiflora}  & 錫蘭霉草 & D \index{Sciaphila@\textit{Sciaphila}!secundiflora@\textit{secundiflora}}  \index{錫蘭霉草} \\
    Urticaceae & 蕁麻科 & \textit{Boehmeria longispica}  & 長穗苧麻 & B1ab(ii,iii) \index{Boehmeria@\textit{Boehmeria}!longispica@\textit{longispica}}  \index{長穗苧麻} \\
    Urticaceae & 蕁麻科 & \textit{Poikilospermum acuminata}  & 錐頭麻 & B1ab(iii) \index{Poikilospermum@\textit{Poikilospermum}!acuminata@\textit{acuminata}}  \index{錐頭麻} \\
    Vitaceae & 葡萄科 & \textit{Vitis thunbergii} var. \textit{taiwaniana}  & 小葉葡萄 & A1a; D \index{Vitis@\textit{Vitis}!thunbergii@\textit{thunbergii}!var. taiwaniana@var. \textit{taiwaniana}}  \index{小葉葡萄} \\
    Zingiberaceae & 薑科 & \textit{Zingiber oligophyllum}  & 少葉薑 & D \index{Zingiber@\textit{Zingiber}!oligophyllum@\textit{oligophyllum}}  \index{少葉薑} \\
    \bottomrule
        \end{longtable}
    %%\end{table}
        }
    