%\begin{table}[!h]
    \begin{longtable}{p{3cm}p{2cm}p{5cm}p{3cm}}
    \toprule
      科名 & 科中名 & 分類群學名 & 分類群中名  \\
    \midrule 
    \endfirsthead
    
    {{\bfseries \tablename\ \thetable{} 續前頁 }} \\
    科名 & 科中名 & 分類群學名 & 分類群中名  \\
    \midrule
    \endhead
            Alismataceae & 澤瀉科 & \textit{Caldesia grandis}  & 圓葉澤瀉\\
    Annonaceae & 番荔枝科 & \textit{Goniothalamus amuyon}  & 恆春哥納香\\
    Annonaceae & 番荔枝科 & \textit{Polyalthia liukiuensis}  & 琉球暗羅\\
    Apiaceae & 繖形科 & \textit{Sium suave}  & 細葉零餘子\\
    Apocynaceae & 夾竹桃科 & \textit{Telosma pallida}  & 夜香花\\
    Aponogetonaceae & 水蕹科 & \textit{Aponogeton taiwanensis}  & 水蕹\\
    Araceae & 天南星科 & \textit{Lemna trisulca}  & 品藻\\
    Aristolochiaceae & 馬兜鈴科 & \textit{Aristolochia yujungiana}  & 裕榮馬兜鈴\\
    Aristolochiaceae & 馬兜鈴科 & \textit{Asarum tawushanianum}  & 大武山細辛\\
    Asparagaceae & 天門冬科 & \textit{Thysanotus chinensis}  & 異蕊草\\
    Aspleniaceae & 鐵角蕨科 & \textit{Asplenium crinicaule}  & 毛軸鐵角蕨\\
    Asteraceae & 菊科 & \textit{Dendranthema horaimontana}  & 蓬萊油菊\\
    Asteraceae & 菊科 & \textit{Echinops grijsii}  & 漏盧\\
    Asteraceae & 菊科 & \textit{Gerbera anandria}  & 大丁草\\
    Asteraceae & 菊科 & \textit{Syneilesis intermedia}  & 臺灣破傘菊\\
    Balanophoraceae & 蛇菰科 & \textit{Balanophora japonica}  & 日本蛇菰\\
    Berberidaceae & 小蘗科 & \textit{Berberis chingshuiensis}  & 清水山小蘗\\
    Berberidaceae & 小蘗科 & \textit{Berberis mingetsuensis}  & 眠月小檗\\
    Berberidaceae & 小蘗科 & \textit{Berberis tarokoensis}  & 太魯閣小蘗\\
    Blechnaceae & 烏毛蕨科 & \textit{Diploblechnum fraseri}  & 假桫欏\\
    Burmanniaceae & 水玉簪科 & \textit{Thismia taiwanensis}  & 臺灣水玉杯\\
    Capparaceae & 山柑科 & \textit{Capparis lanceolaris}  & 蘭嶼山柑\\
    Celastraceae & 衛矛科 & \textit{Euonymus huangii}  & 黃氏衛矛\\
    Celastraceae & 衛矛科 & \textit{Euonymus japonicus}  & 日本衛矛\\
    Convolvulaceae & 旋花科 & \textit{Argyreia akoensis}  & 屏東朝顏\\
    Convolvulaceae & 旋花科 & \textit{Lepistemon intermedius}  & 光滑鱗蕊藤\\
    Convolvulaceae & 旋花科 & \textit{Merremia similis}  & 紅花姬旋花\\
    Cycadaceae & 蘇鐵科 & \textit{Cycas taitungensis}  & 臺東蘇鐵\\
    Cyperaceae & 莎草科 & \textit{Carex kobomugi}  & 海米\\
    Cyperaceae & 莎草科 & \textit{Cyperus albescens}  & 華湖瓜草\\
    Cyperaceae & 莎草科 & \textit{Cyperus leptocarpus}  & 銀穗湖瓜草\\
    Cyperaceae & 莎草科 & \textit{Eleocharis retroflexa}  & 貝殼葉荸薺\\
    Cyperaceae & 莎草科 & \textit{Fimbristylis autumnalis}  & 秋飄拂草\\
    Cyperaceae & 莎草科 & \textit{Fimbristylis nutans}  & 點頭飄拂草\\
    Cyperaceae & 莎草科 & \textit{Rhynchospora malasica}  & 馬來刺子莞\\
    Cystopteridaceae & 冷蕨科 & \textit{Gymnocarpium oyamense}  & 羽節蕨\\
    Dennstaedtiaceae & 碗蕨科 & \textit{Microlepia platyphylla}  & 闊葉鱗蓋蕨\\
    Dennstaedtiaceae & 碗蕨科 & \textit{Paesia radula}  & 曲軸蕨\\
    Dioscoreaceae & 薯蕷科 & \textit{Tacca leontopetaloides}  & 蒟蒻薯\\
    Dryopteridaceae & 鱗毛蕨科 & \textit{Elaphoglossum commutatum}  & 大葉舌蕨\\
    Dryopteridaceae & 鱗毛蕨科 & \textit{Polystichum attenuatum}  & 長羽芽苞耳蕨\\
    Dryopteridaceae & 鱗毛蕨科 & \textit{Polystichum capillipes}  & 小耳蕨\\
    Dryopteridaceae & 鱗毛蕨科 & \textit{Polystichum chunii}  & 陳氏耳蕨\\
    Dryopteridaceae & 鱗毛蕨科 & \textit{Polystichum grandifrons}  & 九州耳蕨\\
    Dryopteridaceae & 鱗毛蕨科 & \textit{Polystichum herbaceum}  & 草葉耳蕨\\
    Dryopteridaceae & 鱗毛蕨科 & \textit{Polystichum tenuius}  & 離脈柳葉蕨\\
    Elaeagnaceae & 胡頹子科 & \textit{Elaeagnus formosensis}  & 蓬萊胡頹子\\
    Eriocaulaceae & 穀精草科 & \textit{Eriocaulon nantoense}  & 南投穀精草\\
    Eriocaulaceae & 穀精草科 & \textit{Eriocaulon nepalense}  & 尼泊爾穀精草\\
    Eriocaulaceae & 穀精草科 & \textit{Eriocaulon taishanense}  & 泰山穀精草\\
    Fabaceae & 豆科 & \textit{Hylodesmum densum}  & 菱葉山螞蝗\\
    Fabaceae & 豆科 & \textit{Indigofera byobiensis}  & 貓鼻頭木藍\\
    Fabaceae & 豆科 & \textit{Indigofera taiwaniana}  & 臺灣木藍\\
    Fabaceae & 豆科 & \textit{Kummerowia stipulacea}  & 圓葉雞眼草\\
    Fabaceae & 豆科 & \textit{Lespedeza daurica}  & 大胡枝子\\
    Fabaceae & 豆科 & \textit{Millettia pulchra} var. \textit{microphylla}  & 小葉魚藤\\
    Fabaceae & 豆科 & \textit{Mucuna gigantea} subsp. \textit{tashiroi}  & 大血藤\\
    Fabaceae & 豆科 & \textit{Vigna adenantha}  & 腺藥豇豆\\
    Fagaceae & 殼斗科 & \textit{Lithocarpus formosanus}  & 臺灣石櫟\\
    Fagaceae & 殼斗科 & \textit{Quercus aliena}  & 大槲樹\\
    Gentianaceae & 龍膽科 & \textit{Gentiana tarokoensis}  & 太魯閣龍膽\\
    Gentianaceae & 龍膽科 & \textit{Gentiana tentyoensis}  & 厚葉龍膽\\
    Gentianaceae & 龍膽科 & \textit{Lomatogonium chilaiensis}  & 奇萊肋柱花\\
    Gentianaceae & 龍膽科 & \textit{Tripterospermum lilungshanensis}  & 里龍山肺形草\\
    Goodeniaceae & 草海桐科 & \textit{Scaevola hainanensis}  & 海南草海桐\\
    Hymenophyllaceae & 膜蕨科 & \textit{Crepidomanes bipunctatum}  & 南洋假脈蕨\\
    Hymenophyllaceae & 膜蕨科 & \textit{Crepidomanes parvifolium}  & 小葉假脈蕨\\
    Hymenophyllaceae & 膜蕨科 & \textit{Hymenophyllum palmatifidum}  & 毛緣細口團扇蕨\\
    Hymenophyllaceae & 膜蕨科 & \textit{Hymenophyllum productum}  & 南洋蕗蕨\\
    Hymenophyllaceae & 膜蕨科 & \textit{Hymenophyllum simonsianum}  & 寬片膜蕨\\
    Hymenophyllaceae & 膜蕨科 & \textit{Hymenophyllum taiwanense}  & 臺灣蕗蕨\\
    Isoëtaceae & 水韭科 & \textit{Isoetes taiwanensis}  & 臺灣水韭\\
    Lamiaceae & 唇形科 & \textit{Lamium amplexicaule}  & 寶蓋草\\
    Lamiaceae & 唇形科 & \textit{Platostoma hispidum}  & 頂頭花\\
    Lauraceae & 樟科 & \textit{Cinnamomum kotoense}  & 蘭嶼肉桂\\
    Lauraceae & 樟科 & \textit{Cryptocarya elliptifolia}  & 菲律賓厚殼桂\\
    Lauraceae & 樟科 & \textit{Dehaasia incrassata}  & 腰果楠\\
    Lauraceae & 樟科 & \textit{Endiandra coriacea}  & 三蕊楠\\
    Lauraceae & 樟科 & \textit{Litsea garciae}  & 蘭嶼木薑子\\
    Lauraceae & 樟科 & \textit{Neolitsea villosa}  & 蘭嶼新木薑子\\
    Lentibulariaceae & 狸藻科 & \textit{Utricularia australis}  & 南方狸藻\\
    Lentibulariaceae & 狸藻科 & \textit{Utricularia caerulea}  & 短梗挖耳草\\
    Liliaceae & 百合科 & \textit{Lilium speciosum} var. \textit{gloriosoides}  & 艷紅鹿子百合\\
    Lythraceae & 千屈菜科 & \textit{Rotala hippuris}  & 水杉菜\\
    Lythraceae & 千屈菜科 & \textit{Trapa japonica}  & 日本菱\\
    Marattiaceae & 觀音座蓮舅科 & \textit{Ptisana pellucida}  & 觀音座蓮舅\\
    Melastomataceae & 野牡丹科 & \textit{Bredia laisherana}  & 來社山布勒德藤\\
    Menispermaceae & 防己科 & \textit{Cissampelos pareira}  & 毛錫生藤\\
    Menyanthaceae & 睡菜科 & \textit{Nymphoides aurantiaca}  & 黃花莕菜\\
    Menyanthaceae & 睡菜科 & \textit{Nymphoides hydrophylla}  & 龍骨瓣莕菜\\
    Nymphaeaceae & 睡蓮科 & \textit{Euryale ferox}  & 芡\\
    Nymphaeaceae & 睡蓮科 & \textit{Nuphar shimadai}  & 臺灣萍蓬草\\
    Oleaceae & 木犀科 & \textit{Chionanthus coriaceus}  & 厚葉李欖\\
    Onocleaceae & 球子蕨科 & \textit{Pentarhizidium orientale}  & 東方莢果蕨\\
    Ophioglossaceae & 瓶爾小草科 & \textit{Helminthostachys zeylanica}  & 錫蘭七指蕨\\
    Orchidaceae & 蘭科 & \textit{Agrostophyllum inocephalum}  & 臺灣禾葉蘭\\
    Orchidaceae & 蘭科 & \textit{Amitostigma gracile}  & 小雛蘭\\
    Orchidaceae & 蘭科 & \textit{Appendicula lucbanensis}  & 多枝竹節蘭\\
    Orchidaceae & 蘭科 & \textit{Arundina graminifolia}  & 葦草蘭\\
    Orchidaceae & 蘭科 & \textit{Brachycorythis galeandra}  & 寬唇苞葉蘭\\
    Orchidaceae & 蘭科 & \textit{Brachycorythis peitawuensis}  & 北大武苞葉蘭\\
    Orchidaceae & 蘭科 & \textit{Bulbophyllum fimbriperianthium}  & 流蘇豆蘭\\
    Orchidaceae & 蘭科 & \textit{Bulbophyllum riyanum}  & 白花豆蘭\\
    Orchidaceae & 蘭科 & \textit{Bulbophyllum rubrolabellum}  & 紅心豆蘭\\
    Orchidaceae & 蘭科 & \textit{Calanthe alpina}  & 羽唇根節蘭\\
    Orchidaceae & 蘭科 & \textit{Cheirostylis pusilla} var. \textit{simplex}  & 沈氏指柱蘭\\
    Orchidaceae & 蘭科 & \textit{Cheirostylis tortilacinia}  & 和社指柱蘭\\
    Orchidaceae & 蘭科 & \textit{Chiloschista parishii}  & 寬囊大蜘蛛蘭\\
    Orchidaceae & 蘭科 & \textit{Corybas himalaicus}  & 喜馬拉雅盔蘭\\
    Orchidaceae & 蘭科 & \textit{Corybas puniceus}  & 艷紫盔蘭\\
    Orchidaceae & 蘭科 & \textit{Cymbidium sinense}  & 報歲蘭\\
    Orchidaceae & 蘭科 & \textit{Cypripedium segawai}  & 寶島喜普鞋蘭\\
    Orchidaceae & 蘭科 & \textit{Dendrobium crumenatum}  & 鴿石斛\\
    Orchidaceae & 蘭科 & \textit{Dendrobium linawianum}  & 櫻石斛\\
    Orchidaceae & 蘭科 & \textit{Dendrobium luzonense}  & 呂宋石斛\\
    Orchidaceae & 蘭科 & \textit{Epipogium aphyllum}  & 無葉上鬚蘭\\
    Orchidaceae & 蘭科 & \textit{Epipogium japonicum}  & 日本上鬚蘭\\
    Orchidaceae & 蘭科 & \textit{Eria herklotsii}  & 香港毛蘭\\
    Orchidaceae & 蘭科 & \textit{Eria javanica}  & 大葉绒蘭\\
    Orchidaceae & 蘭科 & \textit{Eria robusta}  & 細花絨蘭\\
    Orchidaceae & 蘭科 & \textit{Eulophia dentata}  & 紫芋蘭\\
    Orchidaceae & 蘭科 & \textit{Eulophia pelorica}  & 輻射芋蘭\\
    Orchidaceae & 蘭科 & \textit{Flickingeria tairukounia}  & 輻射暫花蘭\\
    Orchidaceae & 蘭科 & \textit{Flickingeria xantholeuca}  & 淺黃暫花蘭\\
    Orchidaceae & 蘭科 & \textit{Gastrochilus hoii}  & 何氏松蘭\\
    Orchidaceae & 蘭科 & \textit{Gastrodia flexistyla}  & 摺柱赤箭\\
    Orchidaceae & 蘭科 & \textit{Gastrodia fontinalis}  & 春赤箭\\
    Orchidaceae & 蘭科 & \textit{Gastrodia sui}  & 蘇氏赤箭\\
    Orchidaceae & 蘭科 & \textit{Hancockia uniflora}  & 漢考克蘭\\
    Orchidaceae & 蘭科 & \textit{Lecanorchis virella}  & 綠皿蘭\\
    Orchidaceae & 蘭科 & \textit{Liparis amabilis}  & 白花羊耳蒜\\
    Orchidaceae & 蘭科 & \textit{Liparis liangzuensis}  & 良如羊耳蘭\\
    Orchidaceae & 蘭科 & \textit{Liparis odorata}  & 香花羊耳蒜\\
    Orchidaceae & 蘭科 & \textit{Luisia cordata}  & 心唇金釵蘭\\
    Orchidaceae & 蘭科 & \textit{Neottia pseudonipponica}  & 假日本雙葉蘭\\
    Orchidaceae & 蘭科 & \textit{Nervilia cumberlegei}  & 古氏脈葉蘭\\
    Orchidaceae & 蘭科 & \textit{Nervilia hungii}  & 鐮唇脈葉蘭\\
    Orchidaceae & 蘭科 & \textit{Odontochilus elwesii}  & 紫葉齒唇蘭\\
    Orchidaceae & 蘭科 & \textit{Odontochilus guangdongensis}  & 南嶺疊鞘蘭\\
    Orchidaceae & 蘭科 & \textit{Odontochilus poilanei}  & 齒爪齒唇蘭\\
    Orchidaceae & 蘭科 & \textit{Pachystoma pubesens}  & 粉口蘭\\
    Orchidaceae & 蘭科 & \textit{Peristylus monticola}  & 深山闊蕊蘭\\
    Orchidaceae & 蘭科 & \textit{Phalaenopsis aphrodite}  & 白蝴蝶蘭\\
    Orchidaceae & 蘭科 & \textit{Phalaenopsis equestris}  & 桃紅蝴蝶蘭\\
    Orchidaceae & 蘭科 & \textit{Phalaenopsis equestris}  & 桃紅蝴蝶蘭\\
    Orchidaceae & 蘭科 & \textit{Phreatia caulescens}  & 垂莖芙樂蘭\\
    Orchidaceae & 蘭科 & \textit{Spathoglottis plicata}  & 紫苞舌蘭\\
    Orchidaceae & 蘭科 & \textit{Tropidia namasiae}  & 那瑪夏摺唇蘭\\
    Orchidaceae & 蘭科 & \textit{Vanda lamellata}  & 雅美萬代蘭\\
    Orchidaceae & 蘭科 & \textit{Yoania amagiensis} var. \textit{squamipes}  & 密鱗長花柄蘭\\
    Orchidaceae & 蘭科 & \textit{Zeuxine philippinensis}  & 菲律賓線柱蘭\\
    Orobanchaceae & 列當科 & \textit{Phacellanthus tubiflorus}  & 黃筒花\\
    Pinaceae & 松科 & \textit{Keteleeria davidiana} var. \textit{formosana}  & 臺灣油杉\\
    Plagiogyriaceae & 瘤足蕨科 & \textit{Plagiogyria koidzumii}  & 小泉氏瘤足蕨\\
    Plantaginaceae & 車前科 & \textit{Limnophila heterophylla}  & 異葉石龍尾\\
    Plantaginaceae & 車前科 & \textit{Veronicastrum loshanense}  & 羅山腹水草\\
    Plumbaginaceae & 藍雪科 & \textit{Limonium wrightii}  & 烏芙蓉\\
    Poaceae & 禾本科 & \textit{Aristida chinensis}  & 華三芒草\\
    Poaceae & 禾本科 & \textit{Chikusichloa mutica}  & 無芒山澗草\\
    Poaceae & 禾本科 & \textit{Dimeria ornithopoda}  & 觿茅\\
    Poaceae & 禾本科 & \textit{Eragrostis cylindrica}  & 短穗畫眉草\\
    Poaceae & 禾本科 & \textit{Eragrostis nevinii}  & 尼氏畫眉草\\
    Poaceae & 禾本科 & \textit{Eragrostis pilosissima}  & 多毛知風草\\
    Poaceae & 禾本科 & \textit{Eragrostis pilosiuscula}  & 毛葉知風草\\
    Poaceae & 禾本科 & \textit{Eulalia quadrinervis}  & 四脈金茅\\
    Poaceae & 禾本科 & \textit{Eulalia speciosa}  & 金茅\\
    Poaceae & 禾本科 & \textit{Eulaliopsis binata}  & 擬金茅\\
    Poaceae & 禾本科 & \textit{Festuca parvigluma}  & 小穎羊茅\\
    Poaceae & 禾本科 & \textit{Leptaspis formosana}  & 囊稃竹\\
    Poaceae & 禾本科 & \textit{Oryzopsis obtusa}  & 鈍頭落芒草\\
    Poaceae & 禾本科 & \textit{Panicum curviflorum} var. \textit{suishaense}  & 水社黍\\
    Poaceae & 禾本科 & \textit{Themeda caudata}  & 苞子草\\
    Poaceae & 禾本科 & \textit{Tripogon chinensis}  & 中華草沙蠶\\
    Podocarpaceae & 羅漢松科 & \textit{Podocarpus costalis}  & 蘭嶼羅漢松\\
    Polygonaceae & 蓼科 & \textit{Persicaria maackiana}  & 長戟葉蓼\\
    Polypodiaceae & 水龍骨科 & \textit{Lepisorus mucronatus}  & 尖嘴蕨\\
    Polypodiaceae & 水龍骨科 & \textit{Oreogrammitis caespitosa}  & 穴孢濱禾蕨\\
    Polypodiaceae & 水龍骨科 & \textit{Oreogrammitis marivelesensis}  & 弼昭禾葉蕨\\
    Polypodiaceae & 水龍骨科 & \textit{Oreogrammitis orientalis}  & 東亞禾葉蕨\\
    Polypodiaceae & 水龍骨科 & \textit{Prosaptia pectinata}  & 篦齒穴子蕨\\
    Polypodiaceae & 水龍骨科 & \textit{Radiogrammitis ilanensis}  & 宜蘭禾葉蕨\\
    Polypodiaceae & 水龍骨科 & \textit{Radiogrammitis moorei}  & 牟氏輻禾蕨\\
    Potamogetonaceae & 眼子菜科 & \textit{Potamogeton cristatus}  & 冠果眼子菜\\
    Primulaceae & 報春花科 & \textit{Lysimachia chingshuiensis}  & 清水山過路黃\\
    Pteridaceae & 鳳尾蕨科 & \textit{Adiantum capillus-junonis}  & 團羽鐵線蕨\\
    Pteridaceae & 鳳尾蕨科 & \textit{Adiantum meishanianum}  & 梅山口鐵線蕨\\
    Pteridaceae & 鳳尾蕨科 & \textit{Haplopteris heterophylla}  & 異葉書帶蕨\\
    Pteridaceae & 鳳尾蕨科 & \textit{Pteris angustipinna}  & 細葉鳳尾蕨\\
    Pteridaceae & 鳳尾蕨科 & \textit{Pteris dimorpha} var. \textit{metagrevilleana}  & 擬翅柄鳳尾蕨\\
    Pteridaceae & 鳳尾蕨科 & \textit{Pteris × wulaiensis}   & 烏來鳳尾蕨\\
    Pteridaceae & 鳳尾蕨科 & \textit{Vaginularia trichoidea}  & 一條線蕨\\
    Ranunculaceae & 毛茛科 & \textit{Clematis uncinata} var. \textit{okinawensis}  & 毛果鐵線蓮\\
    Rosaceae & 薔薇科 & \textit{Cotoneaster chingshuiensis}  & 清水山栒子\\
    Rosaceae & 薔薇科 & \textit{Malus hupehensis}  & 湖北海棠\\
    Rosaceae & 薔薇科 & \textit{Osteomeles anthyllidifolia} var. \textit{subrotunda}  & 小石積\\
    Rosaceae & 薔薇科 & \textit{Osteomeles schwerinae}  & 華西小石積\\
    Rosaceae & 薔薇科 & \textit{Pyrus calleryana}  & 豆梨\\
    Rosaceae & 薔薇科 & \textit{Pyrus taiwanensis}  & 臺灣野梨\\
    Rosaceae & 薔薇科 & \textit{Rubus flagelliflorus}  & 裂緣苞懸鉤子\\
    Rosaceae & 薔薇科 & \textit{Sanguisorba officinalis} var. \textit{longifolia}  & 臺灣地榆\\
    Rubiaceae & 茜草科 & \textit{Cephalanthus naucleoides}  & 風箱樹\\
    Rubiaceae & 茜草科 & \textit{Pavetta indica}  & 茜木\\
    Rutaceae & 芸香科 & \textit{Phellodendron amurense} var. \textit{wilsonii}  & 臺灣黃蘗\\
    Salviniaceae & 槐葉萍科 & \textit{Salvinia natans}  & 槐葉蘋\\
    Sapindaceae & 無患子科 & \textit{Acer buergerianum} var. \textit{formosanum}  & 臺灣三角楓\\
    Schizaeaceae & 莎草蕨科 & \textit{Schizaea dichotoma}  & 分枝莎草蕨\\
    Selaginellaceae & 卷柏科 & \textit{Selaginella nipponica}  & 日本卷柏\\
    Solanaceae & 茄科 & \textit{Solanum luzoniense}  & 呂宋茄\\
    Theaceae & 茶科 & \textit{Pyrenaria buisanensis}  & 武威山烏皮茶\\
    Thelypteridaceae & 金星蕨科 & \textit{Metathelypteris flaccida}  & 薄葉凸軸蕨\\
    Urticaceae & 蕁麻科 & \textit{Elatostema multicanaliculatum}  & 多溝樓梯草\\
    Urticaceae & 蕁麻科 & \textit{Pouzolzia taiwaniana}  & 臺灣霧水葛\\
    Woodsiaceae & 岩蕨科 & \textit{Woodsia andersonii}  & 蜘蛛岩蕨\\
    Woodsiaceae & 岩蕨科 & \textit{Woodsia okamotoi}  & 岡本氏岩蕨\\
    Xyridaceae & 蔥草科 & \textit{Xyris formosana}  & 桃園草\\
    \bottomrule
    \end{longtable}
%%\end{table}