%\begin{table}[!h]
    \begin{longtable}{p{3cm}p{2cm}p{5cm}p{3cm}}
    \toprule
      科名 & 科中名 & 分類群學名 & 分類群中名  \\
    \midrule 
    \endfirsthead
    
    {{\bfseries \tablename\ \thetable{} 續前頁 }} \\
    科名 & 科中名 & 分類群學名 & 分類群中名  \\
    \midrule
    \endhead
            Asteraceae & 菊科 & \textit{Artemisia annua}  & 黃花蒿\\
    Asteraceae & 菊科 & \textit{Tephroseris kirilowii}  & 狗舌草\\
    Cyperaceae & 莎草科 & \textit{Actinoscirpus grossus}  & 大藨草\\
    Cyperaceae & 莎草科 & \textit{Carex metallica}  & 寬穗薹\\
    Cyperaceae & 莎草科 & \textit{Carex scabrifolia}  & 鹼簣\\
    Cyperaceae & 莎草科 & \textit{Cyperus unioloides}  & 水社扁莎\\
    Cyperaceae & 莎草科 & \textit{Fimbristylis acuminata}  & 尖穗飄拂草\\
    Cyperaceae & 莎草科 & \textit{Fimbristylis tetragona}  & 四方型飄拂草\\
    Cyperaceae & 莎草科 & \textit{Rhynchospora chinensis}  & 華刺子莞\\
    Cyperaceae & 莎草科 & \textit{Schoenus falcatus}  & 赤箭莎\\
    Cyperaceae & 莎草科 & \textit{Scleria sumatrensis}  & 印尼珍珠茅\\
    Hydrocharitaceae & 水鱉科 & \textit{Najas ancistrocarpa}  & 彎果茨藻\\
    Lentibulariaceae & 狸藻科 & \textit{Utricularia uliginosa}  & 齒萼挖耳草\\
    Linderniaceae & 母草科 & \textit{Lindernia nummularifolia}  & 寬葉母草\\
    Onagraceae & 柳葉菜科 & \textit{Circaea glabrescens}  & 禿梗露珠草\\
    Orchidaceae & 蘭科 & \textit{Liparis ferruginea}  & 明潭羊耳蒜\\
    Poaceae & 禾本科 & \textit{Hygroryza aristata}  & 水禾\\
    Poaceae & 禾本科 & \textit{Oryza rufipogon}  & 野生稻\\
    Primulaceae & 報春花科 & \textit{Lysimachia candida}  & 澤珍珠菜\\
    Rhizophoraceae & 紅樹科 & \textit{Bruguiera gymnorrhiza}  & 紅茄苳\\
    Rhizophoraceae & 紅樹科 & \textit{Ceriops tagal}  & 細蕊紅樹\\
    Rutaceae & 芸香科 & \textit{Zanthoxylum armatum}  & 秦椒\\
    \bottomrule
    \end{longtable}
%%\end{table}