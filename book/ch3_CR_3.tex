\noindent\normalfont\selectfont Angiosperms 被子植物
\footnotesize\selectfont
%\begin{table}[!h]
        {\def\arraystretch{1.5}\tabcolsep=2pt
        \begin{longtable}{p{2.5cm}p{2.5cm}p{4.5cm}p{2.5cm}p{3cm}}
        \toprule
          科名 & 科中名 & 分類群學名 & 分類群中名 & 評估標準 \\
        \midrule 
        \endfirsthead

        {{\bfseries 續前頁 }} \\
        科名 & 科中名 & 分類群學名 & 分類群中名 & 評估標準 \\
        \midrule
        \endhead
                Alismataceae & 澤瀉科 & \textit{Caldesia grandis}  & 圓葉澤瀉 & B1ab(iii,v)+2ab(iii,v); C1+2a(i,ii); D \index{Caldesia@\textit{Caldesia}!grandis@\textit{grandis}}  \index{圓葉澤瀉} \\
    Annonaceae & 番荔枝科 & \textit{Goniothalamus amuyon}  & 恆春哥納香 & B2b(iv)c(iv); C2b \index{Goniothalamus@\textit{Goniothalamus}!amuyon@\textit{amuyon}}  \index{恆春哥納香} \\
    Annonaceae & 番荔枝科 & \textit{Polyalthia liukiuensis}  & 琉球暗羅 & B1ab(ii,v); C1 \index{Polyalthia@\textit{Polyalthia}!liukiuensis@\textit{liukiuensis}}  \index{琉球暗羅} \\
    Apiaceae & 繖形科 & \textit{Sium suave}  & 細葉零餘子 & D \index{Sium@\textit{Sium}!suave@\textit{suave}}  \index{細葉零餘子} \\
    Apocynaceae & 夾竹桃科 & \textit{Telosma pallida}  & 夜香花 & B2ab(ii) \index{Telosma@\textit{Telosma}!pallida@\textit{pallida}}  \index{夜香花} \\
    Aponogetonaceae & 水蕹科 & \textit{Aponogeton taiwanensis}  & 水蕹 & A2cd \index{Aponogeton@\textit{Aponogeton}!taiwanensis@\textit{taiwanensis}}  \index{水蕹} \\
    Araceae & 天南星科 & \textit{Lemna trisulca}  & 品藻 & A2ac; B2b(iii, iv)c(ii, iii) \index{Lemna@\textit{Lemna}!trisulca@\textit{trisulca}}  \index{品藻} \\
    Aristolochiaceae & 馬兜鈴科 & \textit{Aristolochia yujungiana}  & 裕榮馬兜鈴 & B1ab(iii,v); D \index{Aristolochia@\textit{Aristolochia}!yujungiana@\textit{yujungiana}}  \index{裕榮馬兜鈴} \\
    Aristolochiaceae & 馬兜鈴科 & \textit{Asarum tawushanianum}  & 大武山細辛 & B2ab(ii, iii, v) \index{Asarum@\textit{Asarum}!tawushanianum@\textit{tawushanianum}}  \index{大武山細辛} \\
    Asparagaceae & 天門冬科 & \textit{Thysanotus chinensis}  & 異蕊草 & B1 ab(i, iii, iv) \index{Thysanotus@\textit{Thysanotus}!chinensis@\textit{chinensis}}  \index{異蕊草} \\
    Asteraceae & 菊科 & \textit{Dendranthema horaimontana}  & 蓬萊油菊 & A2c \index{Dendranthema@\textit{Dendranthema}!horaimontana@\textit{horaimontana}}  \index{蓬萊油菊} \\
    Asteraceae & 菊科 & \textit{Echinops grijsii}  & 漏盧 & C2a(i, ii) \index{Echinops@\textit{Echinops}!grijsii@\textit{grijsii}}  \index{漏盧} \\
    Asteraceae & 菊科 & \textit{Gerbera anandria}  & 大丁草 & D \index{Gerbera@\textit{Gerbera}!anandria@\textit{anandria}}  \index{大丁草} \\
    Asteraceae & 菊科 & \textit{Syneilesis intermedia}  & 臺灣破傘菊 & B1+2abcde; D1+2 \index{Syneilesis@\textit{Syneilesis}!intermedia@\textit{intermedia}}  \index{臺灣破傘菊} \\
    Balanophoraceae & 蛇菰科 & \textit{Balanophora japonica}  & 日本蛇菰 & B2ac; C2a; D1 \index{Balanophora@\textit{Balanophora}!japonica@\textit{japonica}}  \index{日本蛇菰} \\
    Berberidaceae & 小檗科 & \textit{Berberis chingshuiensis}  & 清水山小檗 & B1ac(iv) \index{Berberis@\textit{Berberis}!chingshuiensis@\textit{chingshuiensis}}  \index{清水山小檗} \\
    Berberidaceae & 小檗科 & \textit{Berberis mingetsuensis}  & 眠月小檗 & B1ac(ii,iii) \index{Berberis@\textit{Berberis}!mingetsuensis@\textit{mingetsuensis}}  \index{眠月小檗} \\
    Berberidaceae & 小檗科 & \textit{Berberis tarokoensis}  & 太魯閣小檗 & B1ac(iv) \index{Berberis@\textit{Berberis}!tarokoensis@\textit{tarokoensis}}  \index{太魯閣小檗} \\
    Burmanniaceae & 水玉簪科 & \textit{Thismia taiwanensis}  & 臺灣水玉杯 & D \index{Thismia@\textit{Thismia}!taiwanensis@\textit{taiwanensis}}  \index{臺灣水玉杯} \\
    Capparaceae & 山柑科 & \textit{Capparis lanceolaris}  & 蘭嶼山柑 & B1ab(ii,v); C1 \index{Capparis@\textit{Capparis}!lanceolaris@\textit{lanceolaris}}  \index{蘭嶼山柑} \\
    Celastraceae & 衛矛科 & \textit{Euonymus huangii}  & 黃氏衛矛 & B1b(iii)c(iii) \index{Euonymus@\textit{Euonymus}!huangii@\textit{huangii}}  \index{黃氏衛矛} \\
    Celastraceae & 衛矛科 & \textit{Euonymus japonicus}  & 日本衛矛 & B2b(iv)c(iv); C2b \index{Euonymus@\textit{Euonymus}!japonicus@\textit{japonicus}}  \index{日本衛矛} \\
    Convolvulaceae & 旋花科 & \textit{Argyreia akoensis}  & 屏東朝顏 & B1a+2a; D1; D \index{Argyreia@\textit{Argyreia}!akoensis@\textit{akoensis}}  \index{屏東朝顏} \\
    Convolvulaceae & 旋花科 & \textit{Lepistemon intermedius}  & 光滑鮮蕊藤 & D \index{Lepistemon@\textit{Lepistemon}!intermedius@\textit{intermedius}}  \index{光滑鮮蕊藤} \\
    Convolvulaceae & 旋花科 & \textit{Merremia similis}  & 紅花姬旋花 & D \index{Merremia@\textit{Merremia}!similis@\textit{similis}}  \index{紅花姬旋花} \\
    Cyperaceae & 莎草科 & \textit{Carex kobomugi}  & 海米 & B1ab(iii,iv,v)+2ab(iii,iv,v); C1; D1 \index{Carex@\textit{Carex}!kobomugi@\textit{kobomugi}}  \index{海米} \\
    Cyperaceae & 莎草科 & \textit{Cyperus albescens}  & 華湖瓜草 & D1 \index{Cyperus@\textit{Cyperus}!albescens@\textit{albescens}}  \index{華湖瓜草} \\
    Cyperaceae & 莎草科 & \textit{Cyperus leptocarpus}  & 銀穗湖瓜草 & B1ab(i,ii,iii,iv,v)+2ab(i,ii,iii,iv,v) \index{Cyperus@\textit{Cyperus}!leptocarpus@\textit{leptocarpus}}  \index{銀穗湖瓜草} \\
    Cyperaceae & 莎草科 & \textit{Eleocharis retroflexa}  & 貝殼葉荸薺 & B1ab(iii) \index{Eleocharis@\textit{Eleocharis}!retroflexa@\textit{retroflexa}}  \index{貝殼葉荸薺} \\
    Cyperaceae & 莎草科 & \textit{Fimbristylis autumnalis}  & 秋飄拂草 & C1 \index{Fimbristylis@\textit{Fimbristylis}!autumnalis@\textit{autumnalis}}  \index{秋飄拂草} \\
    Cyperaceae & 莎草科 & \textit{Fimbristylis nutans}  & 點頭飄拂草 & D1 \index{Fimbristylis@\textit{Fimbristylis}!nutans@\textit{nutans}}  \index{點頭飄拂草} \\
    Cyperaceae & 莎草科 & \textit{Rhynchospora malasica}  & 馬來刺子莞 & A2d; D \index{Rhynchospora@\textit{Rhynchospora}!malasica@\textit{malasica}}  \index{馬來刺子莞} \\
    Dioscoreaceae & 薯蕷科 & \textit{Tacca leontopetaloides}  & 蒟蒻薯 & B1ab(iii,v) \index{Tacca@\textit{Tacca}!leontopetaloides@\textit{leontopetaloides}}  \index{蒟蒻薯} \\
    Elaeagnaceae & 胡頹子科 & \textit{Elaeagnus formosensis}  & 蓬萊胡頹子 & B1b(i,ii,iii,v)\&; D \index{Elaeagnus@\textit{Elaeagnus}!formosensis@\textit{formosensis}}  \index{蓬萊胡頹子} \\
    Eriocaulaceae & 穀精草科 & \textit{Eriocaulon nantoense}  & 南投穀精草 & B2ac(ii, iii) \index{Eriocaulon@\textit{Eriocaulon}!nantoense@\textit{nantoense}}  \index{南投穀精草} \\
    Eriocaulaceae & 穀精草科 & \textit{Eriocaulon nepalense}  & 尼泊爾穀精草 & B2ac(ii, iii) \index{Eriocaulon@\textit{Eriocaulon}!nepalense@\textit{nepalense}}  \index{尼泊爾穀精草} \\
    Eriocaulaceae & 穀精草科 & \textit{Eriocaulon taishanense}  & 泰山穀精草 & B2ac(ii, iv); D \index{Eriocaulon@\textit{Eriocaulon}!taishanense@\textit{taishanense}}  \index{泰山穀精草} \\
    Fabaceae & 豆科 & \textit{Hylodesmum densum}  & 菱葉山螞蝗 & B2ab(iii) \index{Hylodesmum@\textit{Hylodesmum}!densum@\textit{densum}}  \index{菱葉山螞蝗} \\
    Fabaceae & 豆科 & \textit{Indigofera byobiensis}  & 貓鼻頭木藍 & D1 \index{Indigofera@\textit{Indigofera}!byobiensis@\textit{byobiensis}}  \index{貓鼻頭木藍} \\
    Fabaceae & 豆科 & \textit{Indigofera taiwaniana}  & 臺灣木藍 & D \index{Indigofera@\textit{Indigofera}!taiwaniana@\textit{taiwaniana}}  \index{臺灣木藍} \\
    Fabaceae & 豆科 & \textit{Kummerowia stipulacea}  & 圓葉雞眼草 & B2ab(i, iii) \index{Kummerowia@\textit{Kummerowia}!stipulacea@\textit{stipulacea}}  \index{圓葉雞眼草} \\
    Fabaceae & 豆科 & \textit{Lespedeza daurica}  & 大胡枝子 & B2ab(i, iii) \index{Lespedeza@\textit{Lespedeza}!daurica@\textit{daurica}}  \index{大胡枝子} \\
    Fabaceae & 豆科 & \textit{Millettia pulchra} var. \textit{microphylla}  & 小葉魚藤 & D1 \index{Millettia@\textit{Millettia}!pulchra@\textit{pulchra}!var. microphylla@var. \textit{microphylla}}  \index{小葉魚藤} \\
    Fabaceae & 豆科 & \textit{Mucuna gigantea} subsp. \textit{tashiroi}  & 大血藤 & D1 \index{Mucuna@\textit{Mucuna}!gigantea@\textit{gigantea}!subsp. tashiroi@subsp. \textit{tashiroi}}  \index{大血藤} \\
    Fabaceae & 豆科 & \textit{Vigna adenantha}  & 腺葯豇豆 & B2ab(iii) \index{Vigna@\textit{Vigna}!adenantha@\textit{adenantha}}  \index{腺葯豇豆} \\
    Fagaceae & 殼斗科 & \textit{Lithocarpus formosanus}  & 臺灣石櫟 & D \index{Lithocarpus@\textit{Lithocarpus}!formosanus@\textit{formosanus}}  \index{臺灣石櫟} \\
    Fagaceae & 殼斗科 & \textit{Quercus aliena}  & 大槲樹 & B1ab(iii,v)+2ab(iii,v); C1 \index{Quercus@\textit{Quercus}!aliena@\textit{aliena}}  \index{大槲樹} \\
    Gentianaceae & 龍膽科 & \textit{Gentiana tarokoensis}  & 太魯閣龍膽 & B2ac(iii) \index{Gentiana@\textit{Gentiana}!tarokoensis@\textit{tarokoensis}}  \index{太魯閣龍膽} \\
    Gentianaceae & 龍膽科 & \textit{Gentiana tentyoensis}  & 厚葉龍膽 & B3ac(iii) \index{Gentiana@\textit{Gentiana}!tentyoensis@\textit{tentyoensis}}  \index{厚葉龍膽} \\
    Gentianaceae & 龍膽科 & \textit{Lomatogonium chilaiensis}  & 奇萊肋柱花 & B2ac(iii) \index{Lomatogonium@\textit{Lomatogonium}!chilaiensis@\textit{chilaiensis}}  \index{奇萊肋柱花} \\
    Gentianaceae & 龍膽科 & \textit{Tripterospermum lilungshanensis}  & 里龍山肺形草 & B2ac(iv) \index{Tripterospermum@\textit{Tripterospermum}!lilungshanensis@\textit{lilungshanensis}}  \index{里龍山肺形草} \\
    Goodeniaceae & 草海桐科 & \textit{Scaevola hainanensis}  & 海南草海桐 & B2ab(ii) \index{Scaevola@\textit{Scaevola}!hainanensis@\textit{hainanensis}}  \index{海南草海桐} \\
    Lamiaceae & 唇形科 & \textit{Lamium amplexicaule}  & 寶蓋草 & B2ab(v) \index{Lamium@\textit{Lamium}!amplexicaule@\textit{amplexicaule}}  \index{寶蓋草} \\
    Lamiaceae & 唇形科 & \textit{Platostoma hispidum}  & 頂頭花 & D \index{Platostoma@\textit{Platostoma}!hispidum@\textit{hispidum}}  \index{頂頭花} \\
    Lauraceae & 樟科 & \textit{Cinnamomum kotoense}  & 蘭嶼肉桂 & B1ab(iii,iv)+2ab(iii,iv); C2a(ii) \index{Cinnamomum@\textit{Cinnamomum}!kotoense@\textit{kotoense}}  \index{蘭嶼肉桂} \\
    Lauraceae & 樟科 & \textit{Cryptocarya elliptifolia}  & 菲律賓厚殼桂 & B1ab(iii)+2ab(iii); C2a(i) \index{Cryptocarya@\textit{Cryptocarya}!elliptifolia@\textit{elliptifolia}}  \index{菲律賓厚殼桂} \\
    Lauraceae & 樟科 & \textit{Dehaasia incrassata}  & 腰果楠 & D \index{Dehaasia@\textit{Dehaasia}!incrassata@\textit{incrassata}}  \index{腰果楠} \\
    Lauraceae & 樟科 & \textit{Endiandra coriacea}  & 三蕊楠 & C2a(i) \index{Endiandra@\textit{Endiandra}!coriacea@\textit{coriacea}}  \index{三蕊楠} \\
    Lauraceae & 樟科 & \textit{Litsea garciae}  & 蘭嶼木薑子 & B2ab(iii,v) \index{Litsea@\textit{Litsea}!garciae@\textit{garciae}}  \index{蘭嶼木薑子} \\
    Lauraceae & 樟科 & \textit{Neolitsea villosa}  & 蘭嶼新木薑子 & B1ab(iii,v) \index{Neolitsea@\textit{Neolitsea}!villosa@\textit{villosa}}  \index{蘭嶼新木薑子} \\
    Lentibulariaceae & 狸藻科 & \textit{Utricularia australis}  & 南方狸藻 & B1b(v)c(i); D \index{Utricularia@\textit{Utricularia}!australis@\textit{australis}}  \index{南方狸藻} \\
    Lentibulariaceae & 狸藻科 & \textit{Utricularia caerulea}  & 短梗挖耳草 & B2ac(ii); C2b; D \index{Utricularia@\textit{Utricularia}!caerulea@\textit{caerulea}}  \index{短梗挖耳草} \\
    Liliaceae & 百合科 & \textit{Lilium speciosum} var. \textit{gloriosoides}  & 艷紅鹿子百合 & C1 \index{Lilium@\textit{Lilium}!speciosum@\textit{speciosum}!var. gloriosoides@var. \textit{gloriosoides}}  \index{艷紅鹿子百合} \\
    Lythraceae & 千屈菜科 & \textit{Rotala hippuris}  & 水杉菜 & B2 \index{Rotala@\textit{Rotala}!hippuris@\textit{hippuris}}  \index{水杉菜} \\
    Lythraceae & 千屈菜科 & \textit{Trapa japonica}  & 日本菱 & B1ab(iii)+2ab(iii) \index{Trapa@\textit{Trapa}!japonica@\textit{japonica}}  \index{日本菱} \\
    Melastomataceae & 野牡丹科 & \textit{Bredia laisherana}  & 來社山布勒德藤 & B2ab(iii) \index{Bredia@\textit{Bredia}!laisherana@\textit{laisherana}}  \index{來社山布勒德藤} \\
    Menispermaceae & 防己科 & \textit{Cissampelos pareira}  & 毛錫生藤 & B1a+2a; D2 \index{Cissampelos@\textit{Cissampelos}!pareira@\textit{pareira}}  \index{毛錫生藤} \\
    Menyanthaceae & 睡菜科 & \textit{Nymphoides aurantiaca}  & 黃花莕菜 & D \index{Nymphoides@\textit{Nymphoides}!aurantiaca@\textit{aurantiaca}}  \index{黃花莕菜} \\
    Menyanthaceae & 睡菜科 & \textit{Nymphoides hydrophylla}  & 龍骨瓣莕菜 & B2ab(iii) \index{Nymphoides@\textit{Nymphoides}!hydrophylla@\textit{hydrophylla}}  \index{龍骨瓣莕菜} \\
    Musaceae & 芭蕉科 & \textit{Musa itinerans} var. \textit{chiumei}  & 泰雅芭蕉 & D \index{Musa@\textit{Musa}!itinerans@\textit{itinerans}!var. chiumei@var. \textit{chiumei}}  \index{泰雅芭蕉} \\
    Musaceae & 芭蕉科 & \textit{Musa itinerans} var. \textit{kavalanensis}  & 葛瑪蘭芭蕉 & D \index{Musa@\textit{Musa}!itinerans@\textit{itinerans}!var. kavalanensis@var. \textit{kavalanensis}}  \index{葛瑪蘭芭蕉} \\
    Nymphaeaceae & 睡蓮科 & \textit{Euryale ferox}  & 芡 & D \index{Euryale@\textit{Euryale}!ferox@\textit{ferox}}  \index{芡} \\
    Nymphaeaceae & 睡蓮科 & \textit{Nuphar shimadai}  & 臺灣萍蓬草 & D \index{Nuphar@\textit{Nuphar}!shimadai@\textit{shimadai}}  \index{臺灣萍蓬草} \\
    Oleaceae & 木犀科 & \textit{Chionanthus coriaceus}  & 厚葉李欖 & D \index{Chionanthus@\textit{Chionanthus}!coriaceus@\textit{coriaceus}}  \index{厚葉李欖} \\
    Orchidaceae & 蘭科 & \textit{Agrostophyllum inocephalum}  & 臺灣禾葉蘭 & C2a(i) \index{Agrostophyllum@\textit{Agrostophyllum}!inocephalum@\textit{inocephalum}}  \index{臺灣禾葉蘭} \\
    Orchidaceae & 蘭科 & \textit{Amitostigma gracile}  & 小雛蘭 & D \index{Amitostigma@\textit{Amitostigma}!gracile@\textit{gracile}}  \index{小雛蘭} \\
    Orchidaceae & 蘭科 & \textit{Appendicula lucbanensis}  & 多枝竹節蘭 & B2ab(iii) \index{Appendicula@\textit{Appendicula}!lucbanensis@\textit{lucbanensis}}  \index{多枝竹節蘭} \\
    Orchidaceae & 蘭科 & \textit{Arundina graminifolia}  & 葦草蘭 & Aacd \index{Arundina@\textit{Arundina}!graminifolia@\textit{graminifolia}}  \index{葦草蘭} \\
    Orchidaceae & 蘭科 & \textit{Brachycorythis galeandra}  & 寬唇苞葉蘭 & D \index{Brachycorythis@\textit{Brachycorythis}!galeandra@\textit{galeandra}}  \index{寬唇苞葉蘭} \\
    Orchidaceae & 蘭科 & \textit{Brachycorythis peitawuensis}  & 北大武苞葉蘭 & D1 \index{Brachycorythis@\textit{Brachycorythis}!peitawuensis@\textit{peitawuensis}}  \index{北大武苞葉蘭} \\
    Orchidaceae & 蘭科 & \textit{Bulbophyllum fimbriperianthium}  & 流蘇豆蘭 & D \index{Bulbophyllum@\textit{Bulbophyllum}!fimbriperianthium@\textit{fimbriperianthium}}  \index{流蘇豆蘭} \\
    Orchidaceae & 蘭科 & \textit{Bulbophyllum riyanum}  & 白花豆蘭 & B2ab(ii,iii) \index{Bulbophyllum@\textit{Bulbophyllum}!riyanum@\textit{riyanum}}  \index{白花豆蘭} \\
    Orchidaceae & 蘭科 & \textit{Bulbophyllum rubrolabellum}  & 紅心豆蘭 & D1 \index{Bulbophyllum@\textit{Bulbophyllum}!rubrolabellum@\textit{rubrolabellum}}  \index{紅心豆蘭} \\
    Orchidaceae & 蘭科 & \textit{Calanthe alpina}  & 羽唇根節蘭 & C2a(i) \index{Calanthe@\textit{Calanthe}!alpina@\textit{alpina}}  \index{羽唇根節蘭} \\
    Orchidaceae & 蘭科 & \textit{Cheirostylis pusilla} var. \textit{simplex}  & 沈氏指柱蘭 & D \index{Cheirostylis@\textit{Cheirostylis}!pusilla@\textit{pusilla}!var. simplex@var. \textit{simplex}}  \index{沈氏指柱蘭} \\
    Orchidaceae & 蘭科 & \textit{Cheirostylis tortilacinia}  & 和社指柱蘭 & D \index{Cheirostylis@\textit{Cheirostylis}!tortilacinia@\textit{tortilacinia}}  \index{和社指柱蘭} \\
    Orchidaceae & 蘭科 & \textit{Chiloschista parishii}  & 寬囊大蜘蛛蘭 & B2ab(iii) \index{Chiloschista@\textit{Chiloschista}!parishii@\textit{parishii}}  \index{寬囊大蜘蛛蘭} \\
    Orchidaceae & 蘭科 & \textit{Corybas himalaicus}  & 喜馬拉雅盔蘭 & D \index{Corybas@\textit{Corybas}!himalaicus@\textit{himalaicus}}  \index{喜馬拉雅盔蘭} \\
    Orchidaceae & 蘭科 & \textit{Corybas puniceus}  & 艷紫盔蘭 & C2a(i) \index{Corybas@\textit{Corybas}!puniceus@\textit{puniceus}}  \index{艷紫盔蘭} \\
    Orchidaceae & 蘭科 & \textit{Cymbidium sinense}  & 報歲蘭 & B2ab(i,iv); C2a(i) \index{Cymbidium@\textit{Cymbidium}!sinense@\textit{sinense}}  \index{報歲蘭} \\
    Orchidaceae & 蘭科 & \textit{Cypripedium segawai}  & 寶島喜普鞋蘭 & B1ab(iii,iv); D \index{Cypripedium@\textit{Cypripedium}!segawai@\textit{segawai}}  \index{寶島喜普鞋蘭} \\
    Orchidaceae & 蘭科 & \textit{Dendrobium crumenatum}  & 鴿石斛 & B1ab(iii,iv); D \index{Dendrobium@\textit{Dendrobium}!crumenatum@\textit{crumenatum}}  \index{鴿石斛} \\
    Orchidaceae & 蘭科 & \textit{Dendrobium linawianum}  & 櫻石斛 & A2acd \index{Dendrobium@\textit{Dendrobium}!linawianum@\textit{linawianum}}  \index{櫻石斛} \\
    Orchidaceae & 蘭科 & \textit{Dendrobium luzonense}  & 呂宋石斛 & C2a(i) \index{Dendrobium@\textit{Dendrobium}!luzonense@\textit{luzonense}}  \index{呂宋石斛} \\
    Orchidaceae & 蘭科 & \textit{Epipogium aphyllum}  & 無葉上鬚蘭 & D \index{Epipogium@\textit{Epipogium}!aphyllum@\textit{aphyllum}}  \index{無葉上鬚蘭} \\
    Orchidaceae & 蘭科 & \textit{Epipogium japonicum}  & 日本上鬚蘭 & D \index{Epipogium@\textit{Epipogium}!japonicum@\textit{japonicum}}  \index{日本上鬚蘭} \\
    Orchidaceae & 蘭科 & \textit{Eria herklotsii}  & 香港毛蘭 & D \index{Eria@\textit{Eria}!herklotsii@\textit{herklotsii}}  \index{香港毛蘭} \\
    Orchidaceae & 蘭科 & \textit{Eria javanica}  & 大葉絨蘭 & B1ab(iii,iv); D \index{Eria@\textit{Eria}!javanica@\textit{javanica}}  \index{大葉絨蘭} \\
    Orchidaceae & 蘭科 & \textit{Eria robusta}  & 細花絨蘭 & D \index{Eria@\textit{Eria}!robusta@\textit{robusta}}  \index{細花絨蘭} \\
    Orchidaceae & 蘭科 & \textit{Eulophia dentata}  & 紫芋蘭 & B2ab(iii,iv) \index{Eulophia@\textit{Eulophia}!dentata@\textit{dentata}}  \index{紫芋蘭} \\
    Orchidaceae & 蘭科 & \textit{Eulophia pelorica}  & 輻射芋蘭 & B2ab(v) \index{Eulophia@\textit{Eulophia}!pelorica@\textit{pelorica}}  \index{輻射芋蘭} \\
    Orchidaceae & 蘭科 & \textit{Flickingeria tairukounia}  & 輻射暫花蘭 & C2a(i) \index{Flickingeria@\textit{Flickingeria}!tairukounia@\textit{tairukounia}}  \index{輻射暫花蘭} \\
    Orchidaceae & 蘭科 & \textit{Flickingeria xantholeuca}  & 淺黃暫花蘭 & B2ab(iii) \index{Flickingeria@\textit{Flickingeria}!xantholeuca@\textit{xantholeuca}}  \index{淺黃暫花蘭} \\
    Orchidaceae & 蘭科 & \textit{Gastrochilus hoii}  & 何氏松蘭 & D \index{Gastrochilus@\textit{Gastrochilus}!hoii@\textit{hoii}}  \index{何氏松蘭} \\
    Orchidaceae & 蘭科 & \textit{Gastrodia flexistyla}  & 摺柱赤箭 & D \index{Gastrodia@\textit{Gastrodia}!flexistyla@\textit{flexistyla}}  \index{摺柱赤箭} \\
    Orchidaceae & 蘭科 & \textit{Gastrodia fontinalis} var. \textit{fontinalis}  & 春赤箭 & B2ab(iii) \index{Gastrodia@\textit{Gastrodia}!fontinalis@\textit{fontinalis}!var. fontinalis@var. \textit{fontinalis}}  \index{春赤箭} \\
    Orchidaceae & 蘭科 & \textit{Gastrodia sui}  & 蘇氏赤箭 & D \index{Gastrodia@\textit{Gastrodia}!sui@\textit{sui}}  \index{蘇氏赤箭} \\
    Orchidaceae & 蘭科 & \textit{Hancockia uniflora}  & 漢考克蘭 & D1 \index{Hancockia@\textit{Hancockia}!uniflora@\textit{uniflora}}  \index{漢考克蘭} \\
    Orchidaceae & 蘭科 & \textit{Lecanorchis virella}  & 綠皿蘭 & D \index{Lecanorchis@\textit{Lecanorchis}!virella@\textit{virella}}  \index{綠皿蘭} \\
    Orchidaceae & 蘭科 & \textit{Liparis amabilis}  & 白花羊耳蒜 & C2a(i) \index{Liparis@\textit{Liparis}!amabilis@\textit{amabilis}}  \index{白花羊耳蒜} \\
    Orchidaceae & 蘭科 & \textit{Liparis liangzuensis}  & 良如羊耳蘭 & D \index{Liparis@\textit{Liparis}!liangzuensis@\textit{liangzuensis}}  \index{良如羊耳蘭} \\
    Orchidaceae & 蘭科 & \textit{Liparis odorata}  & 香花羊耳蒜 & B2ab(iv) \index{Liparis@\textit{Liparis}!odorata@\textit{odorata}}  \index{香花羊耳蒜} \\
    Orchidaceae & 蘭科 & \textit{Luisia cordata}  & 心唇金釵蘭 & C2a(i) \index{Luisia@\textit{Luisia}!cordata@\textit{cordata}}  \index{心唇金釵蘭} \\
    Orchidaceae & 蘭科 & \textit{Neottia pseudonipponica}  & 假日本雙葉蘭 & D \index{Neottia@\textit{Neottia}!pseudonipponica@\textit{pseudonipponica}}  \index{假日本雙葉蘭} \\
    Orchidaceae & 蘭科 & \textit{Nervilia cumberlegei}  & 古氏脈葉蘭 & D \index{Nervilia@\textit{Nervilia}!cumberlegei@\textit{cumberlegei}}  \index{古氏脈葉蘭} \\
    Orchidaceae & 蘭科 & \textit{Nervilia hungii}  & 鐮唇脈葉蘭 & B2ab(iii) \index{Nervilia@\textit{Nervilia}!hungii@\textit{hungii}}  \index{鐮唇脈葉蘭} \\
    Orchidaceae & 蘭科 & \textit{Odontochilus elwesii}  & 紫葉齒唇蘭 & D \index{Odontochilus@\textit{Odontochilus}!elwesii@\textit{elwesii}}  \index{紫葉齒唇蘭} \\
    Orchidaceae & 蘭科 & \textit{Odontochilus guangdongensis}  & 南嶺疊鞘蘭 & D \index{Odontochilus@\textit{Odontochilus}!guangdongensis@\textit{guangdongensis}}  \index{南嶺疊鞘蘭} \\
    Orchidaceae & 蘭科 & \textit{Odontochilus poilanei}  & 齒爪齒唇蘭 & D \index{Odontochilus@\textit{Odontochilus}!poilanei@\textit{poilanei}}  \index{齒爪齒唇蘭} \\
    Orchidaceae & 蘭科 & \textit{Pachystoma pubesens}  & 粉口蘭 & C2a(i) \index{Pachystoma@\textit{Pachystoma}!pubesens@\textit{pubesens}}  \index{粉口蘭} \\
    Orchidaceae & 蘭科 & \textit{Peristylus monticola}  & 深山闊蕊蘭 & D \index{Peristylus@\textit{Peristylus}!monticola@\textit{monticola}}  \index{深山闊蕊蘭} \\
    Orchidaceae & 蘭科 & \textit{Phalaenopsis aphrodite}  & 白蝴蝶蘭 & B2a; C1+2a(i); D1 \index{Phalaenopsis@\textit{Phalaenopsis}!aphrodite@\textit{aphrodite}}  \index{白蝴蝶蘭} \\
    Orchidaceae & 蘭科 & \textit{Phalaenopsis equestris}  & 桃紅蝴蝶蘭 & C2a(i,ii); D \index{Phalaenopsis@\textit{Phalaenopsis}!equestris@\textit{equestris}}  \index{桃紅蝴蝶蘭} \\
    Orchidaceae & 蘭科 & \textit{Phreatia caulescens}  & 垂莖芙樂蘭 & D \index{Phreatia@\textit{Phreatia}!caulescens@\textit{caulescens}}  \index{垂莖芙樂蘭} \\
    Orchidaceae & 蘭科 & \textit{Spathoglottis plicata}  & 紫苞舌蘭 & Aacd; B2a \index{Spathoglottis@\textit{Spathoglottis}!plicata@\textit{plicata}}  \index{紫苞舌蘭} \\
    Orchidaceae & 蘭科 & \textit{Tropidia namasiae}  & 那瑪夏摺唇蘭 & D \index{Tropidia@\textit{Tropidia}!namasiae@\textit{namasiae}}  \index{那瑪夏摺唇蘭} \\
    Orchidaceae & 蘭科 & \textit{Vanda lamellata}  & 雅美萬代蘭 & A1acd \index{Vanda@\textit{Vanda}!lamellata@\textit{lamellata}}  \index{雅美萬代蘭} \\
    Orchidaceae & 蘭科 & \textit{Yoania amagiensis} var. \textit{squamipes}  & 密鱗長花柄蘭 & C2a(i) \index{Yoania@\textit{Yoania}!amagiensis@\textit{amagiensis}!var. squamipes@var. \textit{squamipes}}  \index{密鱗長花柄蘭} \\
    Orchidaceae & 蘭科 & \textit{Zeuxine philippinensis}  & 菲律賓線柱蘭 & C2a(i) \index{Zeuxine@\textit{Zeuxine}!philippinensis@\textit{philippinensis}}  \index{菲律賓線柱蘭} \\
    Orobanchaceae & 列當科 & \textit{Phacellanthus tubiflorus}  & 黃筒花 & B2; D \index{Phacellanthus@\textit{Phacellanthus}!tubiflorus@\textit{tubiflorus}}  \index{黃筒花} \\
    Plantaginaceae & 車前科 & \textit{Veronicastrum loshanense}  & 羅山腹水草 & D \index{Veronicastrum@\textit{Veronicastrum}!loshanense@\textit{loshanense}}  \index{羅山腹水草} \\
    Plumbaginaceae & 藍雪科 & \textit{Limonium wrightii}  & 烏芙蓉 & A1acd+2acd+3cd+4acd; B1ab(i,ii,iii,iv)+2ab(i,ii,iii,iv); C1 \index{Limonium@\textit{Limonium}!wrightii@\textit{wrightii}}  \index{烏芙蓉} \\
    Poaceae & 禾本科 & \textit{Aristida chinensis}  & 華三芒草 & A4e; C2b; D \index{Aristida@\textit{Aristida}!chinensis@\textit{chinensis}}  \index{華三芒草} \\
    Poaceae & 禾本科 & \textit{Chikusichloa mutica}  & 無芒山澗草 & A4c; B1ab(iii)+2ab(iii) \index{Chikusichloa@\textit{Chikusichloa}!mutica@\textit{mutica}}  \index{無芒山澗草} \\
    Poaceae & 禾本科 & \textit{Dimeria ornithopoda}  & 觿茅 & B2ab(iii) \index{Dimeria@\textit{Dimeria}!ornithopoda@\textit{ornithopoda}}  \index{觿茅} \\
    Poaceae & 禾本科 & \textit{Eragrostis cylindrica}  & 短穗畫眉草 & B2ac(iv); D \index{Eragrostis@\textit{Eragrostis}!cylindrica@\textit{cylindrica}}  \index{短穗畫眉草} \\
    Poaceae & 禾本科 & \textit{Eragrostis nevinii}  & 尼氏畫眉草 & A4de; B2ac(iv); D \index{Eragrostis@\textit{Eragrostis}!nevinii@\textit{nevinii}}  \index{尼氏畫眉草} \\
    Poaceae & 禾本科 & \textit{Eragrostis pilosissima}  & 多毛知風草 & B2ac(ii,iv); D \index{Eragrostis@\textit{Eragrostis}!pilosissima@\textit{pilosissima}}  \index{多毛知風草} \\
    Poaceae & 禾本科 & \textit{Eragrostis pilosiuscula}  & 毛葉知風草 & A4e; C2b; D \index{Eragrostis@\textit{Eragrostis}!pilosiuscula@\textit{pilosiuscula}}  \index{毛葉知風草} \\
    Poaceae & 禾本科 & \textit{Eulalia quadrinervis}  & 四脈金茅 & B1ac(iv)+2ac(iv) \index{Eulalia@\textit{Eulalia}!quadrinervis@\textit{quadrinervis}}  \index{四脈金茅} \\
    Poaceae & 禾本科 & \textit{Eulalia speciosa}  & 金茅 & B1ac(iv)+2ac(iv) \index{Eulalia@\textit{Eulalia}!speciosa@\textit{speciosa}}  \index{金茅} \\
    Poaceae & 禾本科 & \textit{Eulaliopsis binata}  & 擬金茅 & B2ac(iv) \index{Eulaliopsis@\textit{Eulaliopsis}!binata@\textit{binata}}  \index{擬金茅} \\
    Poaceae & 禾本科 & \textit{Festuca parvigluma}  & 小穎羊茅 & D \index{Festuca@\textit{Festuca}!parvigluma@\textit{parvigluma}}  \index{小穎羊茅} \\
    Poaceae & 禾本科 & \textit{Leptaspis formosana}  & 囊稃竹 & D \index{Leptaspis@\textit{Leptaspis}!formosana@\textit{formosana}}  \index{囊稃竹} \\
    Poaceae & 禾本科 & \textit{Oryzopsis obtusa}  & 鈍頭落芒草 & B2ac(ii,iii) \index{Oryzopsis@\textit{Oryzopsis}!obtusa@\textit{obtusa}}  \index{鈍頭落芒草} \\
    Poaceae & 禾本科 & \textit{Panicum curviflorum} var. \textit{suishaense}  & 水社黍 & A4e; B1ac(iv)+2ac(iv) \index{Panicum@\textit{Panicum}!curviflorum@\textit{curviflorum}!var. suishaense@var. \textit{suishaense}}  \index{水社黍} \\
    Poaceae & 禾本科 & \textit{Themeda caudata}  & 苞子草 & D1 \index{Themeda@\textit{Themeda}!caudata@\textit{caudata}}  \index{苞子草} \\
    Poaceae & 禾本科 & \textit{Tripogon chinensis}  & 中華草沙蠶 & B2ab(ii,iii,v); D1 \index{Tripogon@\textit{Tripogon}!chinensis@\textit{chinensis}}  \index{中華草沙蠶} \\
    Polygonaceae & 蓼科 & \textit{Persicaria maackiana}  & 長戟葉蓼 & A2d \index{Persicaria@\textit{Persicaria}!maackiana@\textit{maackiana}}  \index{長戟葉蓼} \\
    Potamogetonaceae & 眼子菜科 & \textit{Potamogeton cristatus}  & 冠果眼子菜 & B1b(i,iii,iv)c(i,iii,iv) \index{Potamogeton@\textit{Potamogeton}!cristatus@\textit{cristatus}}  \index{冠果眼子菜} \\
    Primulaceae & 報春花科 & \textit{Lysimachia chingshuiensis}  & 清水山過路黃 & B2ab(ii) \index{Lysimachia@\textit{Lysimachia}!chingshuiensis@\textit{chingshuiensis}}  \index{清水山過路黃} \\
    Ranunculaceae & 毛茛科 & \textit{Clematis uncinata} var. \textit{okinawensis}  & 毛果鐵線蓮 & D \index{Clematis@\textit{Clematis}!uncinata@\textit{uncinata}!var. okinawensis@var. \textit{okinawensis}}  \index{毛果鐵線蓮} \\
    Rosaceae & 薔薇科 & \textit{Cotoneaster chingshuiensis}  & 清水山栒子 & D \index{Cotoneaster@\textit{Cotoneaster}!chingshuiensis@\textit{chingshuiensis}}  \index{清水山栒子} \\
    Rosaceae & 薔薇科 & \textit{Malus hupehensis}  & 湖北海棠 & B1ab(v) \index{Malus@\textit{Malus}!hupehensis@\textit{hupehensis}}  \index{湖北海棠} \\
    Rosaceae & 薔薇科 & \textit{Osteomeles anthyllidifolia} var. \textit{subrotunda}  & 小石積 & A4c \index{Osteomeles@\textit{Osteomeles}!anthyllidifolia@\textit{anthyllidifolia}!var. subrotunda@var. \textit{subrotunda}}  \index{小石積} \\
    Rosaceae & 薔薇科 & \textit{Osteomeles schwerinae}  & 華西小石積 & C2a(ii) \index{Osteomeles@\textit{Osteomeles}!schwerinae@\textit{schwerinae}}  \index{華西小石積} \\
    Rosaceae & 薔薇科 & \textit{Pyrus calleryana}  & 豆梨 & C2b \index{Pyrus@\textit{Pyrus}!calleryana@\textit{calleryana}}  \index{豆梨} \\
    Rosaceae & 薔薇科 & \textit{Pyrus taiwanensis}  & 臺灣野梨 & D \index{Pyrus@\textit{Pyrus}!taiwanensis@\textit{taiwanensis}}  \index{臺灣野梨} \\
    Rosaceae & 薔薇科 & \textit{Rubus flagelliflorus}  & 裂緣苞懸鉤子 & B1ab(i) \index{Rubus@\textit{Rubus}!flagelliflorus@\textit{flagelliflorus}}  \index{裂緣苞懸鉤子} \\
    Rosaceae & 薔薇科 & \textit{Sanguisorba officinalis} var. \textit{longifolia}  & 臺灣地榆 & B1ab(i) \index{Sanguisorba@\textit{Sanguisorba}!officinalis@\textit{officinalis}!var. longifolia@var. \textit{longifolia}}  \index{臺灣地榆} \\
    Rubiaceae & 茜草科 & \textit{Cephalanthus naucleoides}  & 風箱樹 & D \index{Cephalanthus@\textit{Cephalanthus}!naucleoides@\textit{naucleoides}}  \index{風箱樹} \\
    Rubiaceae & 茜草科 & \textit{Pavetta indica}  & 茜木 & D \index{Pavetta@\textit{Pavetta}!indica@\textit{indica}}  \index{茜木} \\
    Rutaceae & 芸香科 & \textit{Phellodendron amurense} var. \textit{wilsonii}  & 臺灣黃蘗 & A1a; C2b \index{Phellodendron@\textit{Phellodendron}!amurense@\textit{amurense}!var. wilsonii@var. \textit{wilsonii}}  \index{臺灣黃蘗} \\
    Sapindaceae & 無患子科 & \textit{Acer buergerianum} var. \textit{formosanum}  & 臺灣三角楓 & B2b(iv)c(iv); C2b \index{Acer@\textit{Acer}!buergerianum@\textit{buergerianum}!var. formosanum@var. \textit{formosanum}}  \index{臺灣三角楓} \\
    Solanaceae & 茄科 & \textit{Solanum luzoniense}  & 呂宋茄 & B2b(iv)c(iv); C2b \index{Solanum@\textit{Solanum}!luzoniense@\textit{luzoniense}}  \index{呂宋茄} \\
    Theaceae & 茶科 & \textit{Pyrenaria buisanensis}  & 武威山烏皮茶 & A4; B2a; D \index{Pyrenaria@\textit{Pyrenaria}!buisanensis@\textit{buisanensis}}  \index{武威山烏皮茶} \\
    Urticaceae & 蕁麻科 & \textit{Elatostema multicanaliculatum}  & 多溝樓梯草 & B1ab(ii,iii) \index{Elatostema@\textit{Elatostema}!multicanaliculatum@\textit{multicanaliculatum}}  \index{多溝樓梯草} \\
    Urticaceae & 蕁麻科 & \textit{Pouzolzia taiwaniana}  & 臺灣霧水葛 & B2ac \index{Pouzolzia@\textit{Pouzolzia}!taiwaniana@\textit{taiwaniana}}  \index{臺灣霧水葛} \\
    Xyridaceae & 蔥草科 & \textit{Xyris formosana}  & 桃園草 & B2ab(ii,iii,iv,v) \index{Xyris@\textit{Xyris}!formosana@\textit{formosana}}  \index{桃園草} \\
    \bottomrule
        \end{longtable}
    %%\end{table}
        }
    