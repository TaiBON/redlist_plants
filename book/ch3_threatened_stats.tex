臺灣的野生維管束植物共 5240 分類群,
其中 736 分類群不適用(Not Applicable)區域評估篩選條件,
4504 分類群進入評估流程。評估的結果顯示臺灣有 26 種野生維管束植物已經滅絕,
其中 4 種屬於野外絕滅(Extinct in the Wild),22 種屬於區域滅絕(Regionally Extinct)。
國家受威脅(National Threatened)野生維管束植物共有 1059 分類群,
其中屬於極危(Critically Endangered)類別有 217 分類群,
瀕危(Endangered)類別有 290 分類群,易危(Vulnerable)類別有 516 分類群。
另有 480 分類群歸於接近受脅(Near Threatened)的類別,326 分類群歸於資料缺乏(Data Deficient)的類別,
其餘 2639 分類群則屬於暫無危機(Least Concern)的類別。國家受威脅及接近受脅的野生維管束植物種數分別占評估種數的 23.5 \%及 10.7 \%。
所有納入評估候選的維管束植物的名單及其個別的保育類別亦另列一表,供作參考。
