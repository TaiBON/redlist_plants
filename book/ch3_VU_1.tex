\noindent\normalfont\selectfont Monilophytes 蕨類植物
\footnotesize\selectfont
%\begin{table}[!h]
        {\def\arraystretch{1.5}\tabcolsep=2pt
        \begin{longtable}{p{2.5cm}p{2.5cm}p{4.5cm}p{2.5cm}p{3cm}}
        \toprule
          科名 & 科中名 & 分類群學名 & 分類群中名 & 評估標準 \\
        \midrule 
        \endfirsthead

        {{\bfseries 續前頁 }} \\
        科名 & 科中名 & 分類群學名 & 分類群中名 & 評估標準 \\
        \midrule
        \endhead
                Aspleniaceae & 鐵角蕨科 & \textit{Asplenium boreale}  & 北方倒掛鐵角蕨 & B1ac(i,ii) \index{Asplenium@\textit{Asplenium}!boreale@\textit{boreale}}  \index{北方倒掛鐵角蕨} \\
    Aspleniaceae & 鐵角蕨科 & \textit{Asplenium ruta-muraria}  & 銀杏葉鐵角蕨 & C2a(i) \index{Asplenium@\textit{Asplenium}!ruta-muraria@\textit{ruta-muraria}}  \index{銀杏葉鐵角蕨} \\
    Aspleniaceae & 鐵角蕨科 & \textit{Asplenium steerei}  & 劍羽鐵角蕨 & B1ac(ii,iii,iv) \index{Asplenium@\textit{Asplenium}!steerei@\textit{steerei}}  \index{劍羽鐵角蕨} \\
    Athyriaceae & 蹄蓋蕨科 & \textit{Athyrium auriculatum}  & 耳垂蹄蓋蕨 & D1 \index{Athyrium@\textit{Athyrium}!auriculatum@\textit{auriculatum}}  \index{耳垂蹄蓋蕨} \\
    Athyriaceae & 蹄蓋蕨科 & \textit{Athyrium minimum}  & 七星山蹄蓋蕨 & D2 \index{Athyrium@\textit{Athyrium}!minimum@\textit{minimum}}  \index{七星山蹄蓋蕨} \\
    Athyriaceae & 蹄蓋蕨科 & \textit{Athyrium palustre}  & 沼生蹄蓋蕨 & D2 \index{Athyrium@\textit{Athyrium}!palustre@\textit{palustre}}  \index{沼生蹄蓋蕨} \\
    Athyriaceae & 蹄蓋蕨科 & \textit{Athyrium puncticaule}  & 密腺蹄蓋蕨 & D2 \index{Athyrium@\textit{Athyrium}!puncticaule@\textit{puncticaule}}  \index{密腺蹄蓋蕨} \\
    Athyriaceae & 蹄蓋蕨科 & \textit{Deparia unifurcata}  & 東亞假鱗毛蕨 & C2a(i) \index{Deparia@\textit{Deparia}!unifurcata@\textit{unifurcata}}  \index{東亞假鱗毛蕨} \\
    Athyriaceae & 蹄蓋蕨科 & \textit{Diplazium crassiusculum}  & 厚葉雙蓋蕨 & D1+2 \index{Diplazium@\textit{Diplazium}!crassiusculum@\textit{crassiusculum}}  \index{厚葉雙蓋蕨} \\
    Athyriaceae & 蹄蓋蕨科 & \textit{Diplazium megaphyllum}  & 大葉雙蓋蕨 & B1ac(ii,iii,iv) \index{Diplazium@\textit{Diplazium}!megaphyllum@\textit{megaphyllum}}  \index{大葉雙蓋蕨} \\
    Blechnaceae & 烏毛蕨科 & \textit{Brainea insignis}  & 蘇鐵蕨 & B2ab(iii) \index{Brainea@\textit{Brainea}!insignis@\textit{insignis}}  \index{蘇鐵蕨} \\
    Blechnaceae & 烏毛蕨科 & \textit{Cleistoblechnum eburneum}  & 天長烏毛蕨 & D1 \index{Cleistoblechnum@\textit{Cleistoblechnum}!eburneum@\textit{eburneum}}  \index{天長烏毛蕨} \\
    Cyatheaceae & 桫欏科 & \textit{Alsophila fenicis}  & 蘭嶼筆筒樹 & C2a(ii) \index{Alsophila@\textit{Alsophila}!fenicis@\textit{fenicis}}  \index{蘭嶼筆筒樹} \\
    Dennstaedtiaceae & 碗蕨科 & \textit{Microlepia rhomboidea}  & 斜方鱗蓋蕨 & B1ac(ii,iii,iv) \index{Microlepia@\textit{Microlepia}!rhomboidea@\textit{rhomboidea}}  \index{斜方鱗蓋蕨} \\
    Dryopteridaceae & 鱗毛蕨科 & \textit{Bolbitis scalpturata}  & 紅柄實蕨 & B2ab(iii) \index{Bolbitis@\textit{Bolbitis}!scalpturata@\textit{scalpturata}}  \index{紅柄實蕨} \\
    Dryopteridaceae & 鱗毛蕨科 & \textit{Cyrtomium fortunei}  & 貫眾蕨 & B1ab(ii,iii,iv,v)c(ii,iii,iv)+2ab(ii,iii,iv,v)c(ii,iii,iv) \index{Cyrtomium@\textit{Cyrtomium}!fortunei@\textit{fortunei}}  \index{貫眾蕨} \\
    Dryopteridaceae & 鱗毛蕨科 & \textit{Cyrtomium taiwanense}  & 臺灣貫眾蕨 & B1ac(ii,iii,iv)+2ac(ii,iii,iv) \index{Cyrtomium@\textit{Cyrtomium}!taiwanense@\textit{taiwanense}}  \index{臺灣貫眾蕨} \\
    Dryopteridaceae & 鱗毛蕨科 & \textit{Dryopteris decipiens}  & 迷人鱗毛蕨 & B1ab(iii)+2ab(iii); D \index{Dryopteris@\textit{Dryopteris}!decipiens@\textit{decipiens}}  \index{迷人鱗毛蕨} \\
    Dryopteridaceae & 鱗毛蕨科 & \textit{Dryopteris lacera}  & 二型鱗毛蕨 & D1+2 \index{Dryopteris@\textit{Dryopteris}!lacera@\textit{lacera}}  \index{二型鱗毛蕨} \\
    Dryopteridaceae & 鱗毛蕨科 & \textit{Dryopteris maximowicziana}  & 白鱗肋毛蕨 & D1 \index{Dryopteris@\textit{Dryopteris}!maximowicziana@\textit{maximowicziana}}  \index{白鱗肋毛蕨} \\
    Dryopteridaceae & 鱗毛蕨科 & \textit{Dryopteris namegatae}  & 黑鱗遠軸鱗毛蕨 & D1 \index{Dryopteris@\textit{Dryopteris}!namegatae@\textit{namegatae}}  \index{黑鱗遠軸鱗毛蕨} \\
    Dryopteridaceae & 鱗毛蕨科 & \textit{Dryopteris pseudocaenopteris}  & 紅線蕨 & C2a(i) \index{Dryopteris@\textit{Dryopteris}!pseudocaenopteris@\textit{pseudocaenopteris}}  \index{紅線蕨} \\
    Dryopteridaceae & 鱗毛蕨科 & \textit{Dryopteris ryo-itoana}  & 寬羽鱗毛蕨 & D1+2 \index{Dryopteris@\textit{Dryopteris}!ryo-itoana@\textit{ryo-itoana}}  \index{寬羽鱗毛蕨} \\
    Dryopteridaceae & 鱗毛蕨科 & \textit{Dryopteris wuzhaohongii}  & 大孢魚鱗蕨 & B1ac(iii,iv) \index{Dryopteris@\textit{Dryopteris}!wuzhaohongii@\textit{wuzhaohongii}}  \index{大孢魚鱗蕨} \\
    Dryopteridaceae & 鱗毛蕨科 & \textit{Elaphoglossum callifolium}  & 銳頭舌蕨 & D1 \index{Elaphoglossum@\textit{Elaphoglossum}!callifolium@\textit{callifolium}}  \index{銳頭舌蕨} \\
    Dryopteridaceae & 鱗毛蕨科 & \textit{Elaphoglossum luzonicum}  & 臺灣舌蕨 & D2 \index{Elaphoglossum@\textit{Elaphoglossum}!luzonicum@\textit{luzonicum}}  \index{臺灣舌蕨} \\
    Dryopteridaceae & 鱗毛蕨科 & \textit{Polystichum fraxinellum}  & 網脈耳蕨 & B1ac(i,ii) \index{Polystichum@\textit{Polystichum}!fraxinellum@\textit{fraxinellum}}  \index{網脈耳蕨} \\
    Dryopteridaceae & 鱗毛蕨科 & \textit{Polystichum glaciale}  & 玉龍蕨 & B1ac(ii,iv,v) \index{Polystichum@\textit{Polystichum}!glaciale@\textit{glaciale}}  \index{玉龍蕨} \\
    Dryopteridaceae & 鱗毛蕨科 & \textit{Polystichum neo-lobatum}  & 硬葉耳蕨 & B1ac(ii,iv,v) \index{Polystichum@\textit{Polystichum}!neo-lobatum@\textit{neo-lobatum}}  \index{硬葉耳蕨} \\
    Dryopteridaceae & 鱗毛蕨科 & \textit{Polystichum obliquum}  & 知本耳蕨 & B1ab(iii)+2ab(iii) \index{Polystichum@\textit{Polystichum}!obliquum@\textit{obliquum}}  \index{知本耳蕨} \\
    Dryopteridaceae & 鱗毛蕨科 & \textit{Polystichum prescottianum}  & 南湖耳蕨 & B1ac(ii,iv,v) \index{Polystichum@\textit{Polystichum}!prescottianum@\textit{prescottianum}}  \index{南湖耳蕨} \\
    Dryopteridaceae & 鱗毛蕨科 & \textit{Polystichum taizhongense}  & 臺中耳蕨 & B1ac(i,ii,iv,v) \index{Polystichum@\textit{Polystichum}!taizhongense@\textit{taizhongense}}  \index{臺中耳蕨} \\
    Hymenophyllaceae & 膜蕨科 & \textit{Didymoglossum bimarginatum}  & 叉脈單葉假脈蕨 & D2 \index{Didymoglossum@\textit{Didymoglossum}!bimarginatum@\textit{bimarginatum}}  \index{叉脈單葉假脈蕨} \\
    Hymenophyllaceae & 膜蕨科 & \textit{Hymenophyllum blandum}  & 爪哇厚壁蕨 & D2 \index{Hymenophyllum@\textit{Hymenophyllum}!blandum@\textit{blandum}}  \index{爪哇厚壁蕨} \\
    Hymenophyllaceae & 膜蕨科 & \textit{Hymenophyllum pallidum}  & 毛葉蕨 & D1 \index{Hymenophyllum@\textit{Hymenophyllum}!pallidum@\textit{pallidum}}  \index{毛葉蕨} \\
    Hymenophyllaceae & 膜蕨科 & \textit{Hymenophyllum pilosissimum}  & 星毛膜蕨 & D2 \index{Hymenophyllum@\textit{Hymenophyllum}!pilosissimum@\textit{pilosissimum}}  \index{星毛膜蕨} \\
    Lindsaeaceae & 陵齒蕨科 & \textit{Lindsaea cultrata}  & 網脈陵齒蕨 & D1 \index{Lindsaea@\textit{Lindsaea}!cultrata@\textit{cultrata}}  \index{網脈陵齒蕨} \\
    Lindsaeaceae & 陵齒蕨科 & \textit{Lindsaea lucida}  & 方柄陵齒蕨 & C2a(ii) \index{Lindsaea@\textit{Lindsaea}!lucida@\textit{lucida}}  \index{方柄陵齒蕨} \\
    Lindsaeaceae & 陵齒蕨科 & \textit{Tapeinidium pinnatum} var. \textit{biserratum}  & 二羽達邊蕨 & D1 \index{Tapeinidium@\textit{Tapeinidium}!pinnatum@\textit{pinnatum}!var. biserratum@var. \textit{biserratum}}  \index{二羽達邊蕨} \\
    Lygodiaceae & 海金沙科 & \textit{Lygodium microphyllum}  & 小葉海金沙 & D1 \index{Lygodium@\textit{Lygodium}!microphyllum@\textit{microphyllum}}  \index{小葉海金沙} \\
    Marattiaceae & 觀音座蓮舅科 & \textit{Angiopteris evecta}  & 蘭嶼觀音座蓮 & D1 \index{Angiopteris@\textit{Angiopteris}!evecta@\textit{evecta}}  \index{蘭嶼觀音座蓮} \\
    Ophioglossaceae & 瓶爾小草科 & \textit{Botrychium lanuginosum}  & 阿里山蕨萁 & C2a(i); D1 \index{Botrychium@\textit{Botrychium}!lanuginosum@\textit{lanuginosum}}  \index{阿里山蕨萁} \\
    Osmundaceae & 紫萁科 & \textit{Claytosmunda claytoniana}  & 絨假紫萁 & D1 \index{Claytosmunda@\textit{Claytosmunda}!claytoniana@\textit{claytoniana}}  \index{絨假紫萁} \\
    Osmundaceae & 紫萁科 & \textit{Osmundastrum cinnamomeum}  & 分株假紫萁 & D1+2 \index{Osmundastrum@\textit{Osmundastrum}!cinnamomeum@\textit{cinnamomeum}}  \index{分株假紫萁} \\
    Plagiogyriaceae & 瘤足蕨科 & \textit{Plagiogyria japonica}  & 華東瘤足蕨 & D1 \index{Plagiogyria@\textit{Plagiogyria}!japonica@\textit{japonica}}  \index{華東瘤足蕨} \\
    Polypodiaceae & 水龍骨科 & \textit{Calymmodon ordinatus}  & 姬荷包蕨 & D1 \index{Calymmodon@\textit{Calymmodon}!ordinatus@\textit{ordinatus}}  \index{姬荷包蕨} \\
    Polypodiaceae & 水龍骨科 & \textit{Oreogrammitis adspersa}  & 無毛禾葉蕨 & D2 \index{Oreogrammitis@\textit{Oreogrammitis}!adspersa@\textit{adspersa}}  \index{無毛禾葉蕨} \\
    Polypodiaceae & 水龍骨科 & \textit{Phymatosorus membranifolius}  & 薄葉擬茀蕨 & C2b \index{Phymatosorus@\textit{Phymatosorus}!membranifolius@\textit{membranifolius}}  \index{薄葉擬茀蕨} \\
    Polypodiaceae & 水龍骨科 & \textit{Polypodiodes fieldingiana}  & 栗柄水龍骨 & D1 \index{Polypodiodes@\textit{Polypodiodes}!fieldingiana@\textit{fieldingiana}}  \index{栗柄水龍骨} \\
    Polypodiaceae & 水龍骨科 & \textit{Pyrrosia angustissima}  & 捲葉蕨 & B2ab(iii); D2 \index{Pyrrosia@\textit{Pyrrosia}!angustissima@\textit{angustissima}}  \index{捲葉蕨} \\
    Polypodiaceae & 水龍骨科 & \textit{Radiogrammitis setigera}  & 剛毛輻禾蕨 & C2a(ii) \index{Radiogrammitis@\textit{Radiogrammitis}!setigera@\textit{setigera}}  \index{剛毛輻禾蕨} \\
    Polypodiaceae & 水龍骨科 & \textit{Selliguea taeniata}  & 掌葉茀蕨 & D1+2 \index{Selliguea@\textit{Selliguea}!taeniata@\textit{taeniata}}  \index{掌葉茀蕨} \\
    Polypodiaceae & 水龍骨科 & \textit{Themelium tenuisectum}  & 細葉蒿蕨 & D1+2 \index{Themelium@\textit{Themelium}!tenuisectum@\textit{tenuisectum}}  \index{細葉蒿蕨} \\
    Pteridaceae & 鳳尾蕨科 & \textit{Acrostichum aureum}  & 鹵蕨 & D1+2 \index{Acrostichum@\textit{Acrostichum}!aureum@\textit{aureum}}  \index{鹵蕨} \\
    Pteridaceae & 鳳尾蕨科 & \textit{Adiantum formosanum}  & 深山鐵線蕨 & D1 \index{Adiantum@\textit{Adiantum}!formosanum@\textit{formosanum}}  \index{深山鐵線蕨} \\
    Pteridaceae & 鳳尾蕨科 & \textit{Adiantum myriosorum}  & 灰背鐵線蕨 & B2ab(iii); D2 \index{Adiantum@\textit{Adiantum}!myriosorum@\textit{myriosorum}}  \index{灰背鐵線蕨} \\
    Pteridaceae & 鳳尾蕨科 & \textit{Adiantum roborowskii} var. \textit{taiwanianum}  & 臺灣高山鐵線蕨 & C2a(i) \index{Adiantum@\textit{Adiantum}!roborowskii@\textit{roborowskii}!var. taiwanianum@var. \textit{taiwanianum}}  \index{臺灣高山鐵線蕨} \\
    Pteridaceae & 鳳尾蕨科 & \textit{Adiantum soboliferum}  & 翅柄鐵線蕨 & C2a(i) \index{Adiantum@\textit{Adiantum}!soboliferum@\textit{soboliferum}}  \index{翅柄鐵線蕨} \\
    Pteridaceae & 鳳尾蕨科 & \textit{Antrophyum castaneum}  & 阿里山車前蕨 & B2b(ii)c(ii); C2a(i) \index{Antrophyum@\textit{Antrophyum}!castaneum@\textit{castaneum}}  \index{阿里山車前蕨} \\
    Pteridaceae & 鳳尾蕨科 & \textit{Antrophyum henryi}  & 亨利氏車前蕨 & B2ab(iii); C2a(i); D1 \index{Antrophyum@\textit{Antrophyum}!henryi@\textit{henryi}}  \index{亨利氏車前蕨} \\
    Pteridaceae & 鳳尾蕨科 & \textit{Antrophyum parvulum}  & 無柄車前蕨 & D1 \index{Antrophyum@\textit{Antrophyum}!parvulum@\textit{parvulum}}  \index{無柄車前蕨} \\
    Pteridaceae & 鳳尾蕨科 & \textit{Haplopteris mediosora}  & 細葉書帶蕨 & D1 \index{Haplopteris@\textit{Haplopteris}!mediosora@\textit{mediosora}}  \index{細葉書帶蕨} \\
    Pteridaceae & 鳳尾蕨科 & \textit{Parahemionitis arifolia}  & 澤瀉蕨 & C2a(i) \index{Parahemionitis@\textit{Parahemionitis}!arifolia@\textit{arifolia}}  \index{澤瀉蕨} \\
    Pteridaceae & 鳳尾蕨科 & \textit{Pteris grevilleana} var. \textit{ornata}  & 白斑翅柄鳳尾蕨 & D2 \index{Pteris@\textit{Pteris}!grevilleana@\textit{grevilleana}!var. ornata@var. \textit{ornata}}  \index{白斑翅柄鳳尾蕨} \\
    Pteridaceae & 鳳尾蕨科 & \textit{Pteris kawabatae}  & 無柄鳳尾蕨 & D2 \index{Pteris@\textit{Pteris}!kawabatae@\textit{kawabatae}}  \index{無柄鳳尾蕨} \\
    Pteridaceae & 鳳尾蕨科 & \textit{Pteris latipinna}  & 寬羽鳳尾蕨 & D2 \index{Pteris@\textit{Pteris}!latipinna@\textit{latipinna}}  \index{寬羽鳳尾蕨} \\
    Schizaeaceae & 莎草蕨科 & \textit{Actinostachys digitata}  & 莎草蕨 & VD2 \index{Actinostachys@\textit{Actinostachys}!digitata@\textit{digitata}}  \index{莎草蕨} \\
    Tectariaceae & 三叉蕨科 & \textit{Pteridrys cnemidaria}  & 長柄牙蕨 & D1 \index{Pteridrys@\textit{Pteridrys}!cnemidaria@\textit{cnemidaria}}  \index{長柄牙蕨} \\
    Tectariaceae & 三叉蕨科 & \textit{Tectaria sulitii}  & 多羽三叉蕨 & D1 \index{Tectaria@\textit{Tectaria}!sulitii@\textit{sulitii}}  \index{多羽三叉蕨} \\
    Thelypteridaceae & 金星蕨科 & \textit{Phegopteris connectilis}  & 長柄卵果蕨 & B1b(ii,iii,iv,v)c(ii,iii,iv) \index{Phegopteris@\textit{Phegopteris}!connectilis@\textit{connectilis}}  \index{長柄卵果蕨} \\
    Thelypteridaceae & 金星蕨科 & \textit{Pronephrium longipetiolatum}  & 長柄新月蕨 & B1ac(ii,iv)+2ac(ii,iv) \index{Pronephrium@\textit{Pronephrium}!longipetiolatum@\textit{longipetiolatum}}  \index{長柄新月蕨} \\
    Thelypteridaceae & 金星蕨科 & \textit{Pseudophegopteris levingei}  & 高山紫柄蕨 & B1b(ii,iii,iv,v)c(ii,iii,iv) \index{Pseudophegopteris@\textit{Pseudophegopteris}!levingei@\textit{levingei}}  \index{高山紫柄蕨} \\
    \bottomrule
        \end{longtable}
    %%\end{table}
        }
    