\noindent 本報告依據國際自然保育聯盟(International Union for Conservation of Nature, IUCN)
建議類別與標準對所有臺灣野生維管束植物進行國家紅皮書名錄評估。具有野生紀錄的維管束植物共 5219 分類群,
其中 744 分類群不適用(Not Applicable)區域評估篩選條件, 4475 分類群進入評估流程。
結果臺灣有 26 種野生維管束植物已經滅絕,其中 4 種屬於野外絕滅(Extinct in the Wild),
22 種屬於區域滅絕(Regionally Extinct)。國家受威脅(National Threatened)野生維管束植物共有 1044 分類群,
其中屬於極危(Critically Endangered)類別有 216 分類群,
瀕危(Endangered)類別有 287 分類群,易危(Vulnerable)類別有 511 分類群。
另有 480 分類群歸於接近受脅(Near Threatened)的類別,
321 分類群歸於資料缺乏(Data Deficient)的類別,其餘 2630 分類群則屬於暫無危機(Least Concern)的類別。
國家受威脅和接近受脅的野生維管束植物種數分別占評估種數的 23.3 \% 及 10.7 \%。 
