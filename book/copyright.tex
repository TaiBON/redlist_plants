\clearpage
\pagestyle{plain}
\thispagestyle{empty}
\noindent \Large 2017 臺灣維管束植物紅皮書名錄 \\
\Large The Red List of Vascular Plants of Taiwan, 2017 \\
\hrule
\hfill \\
\hfill \\
\normalsize
\linespread{1.2}\selectfont
\begin{table}[H]
  \begin{tabular}{ll}
      \makebox[5em][s]{發\hspace{\fill}行\hspace{\fill}人} &  楊嘉棟、林華慶、劉和義 \\
      \makebox[5em][s]{作\hspace{\fill}者}                 &  臺灣植物紅皮書編輯委員會 \\
      \makebox[5em][s]{編輯委員會}                         
                                                           & 2008--2010: 王震哲(召集人,師範大學)、邱文良(林業試驗所)、 \\
                                                           & 張和明(特有生物研究保育中 心)、許再文(特有生物研究保育中心)、 \\
                                                           & 郭長生(成功大學)、彭鏡毅(中央研究院)、楊國禎 (靜宜大學)、 \\
                                                           & 劉和義(中山大學)、謝長富(臺灣大學) \\
                                                           & \\
                                                           & 2017: 王志強(屏東科技大學)、江友中(中山大學)、林政道(嘉義大學)、\\
                                                           & 張和明(特有生物研究保育中心)、許再文(特有生物研究保育中心)、\\ 
                                                           & 曾彥學(中興大學)、劉以誠(嘉義大學)、劉和義(召集人,中山大學)、\\ 
                                                           & 謝宗欣(台南大學)、鍾國芳(中央研究院) \\
      \makebox[5em][s]{執\hspace{\fill}行\hspace{\fill}編\hspace{\fill}輯}  &  張和明、劉和義、許再文、林政道 \\
      \makebox[5em][s]{助\hspace{\fill}理\hspace{\fill}編\hspace{\fill}輯}  &  楊松翰、劉瓊芳、廖國藩 \\
      \makebox[5em][s]{出\hspace{\fill}版}      &  行政院農業委員會特有生物研究保育中心\\ 
                                                 &  行政院農業委員會林務局 \\
                                                 &  臺灣植物分類學會 \\
      \makebox[5em][s]{地\hspace{\fill}址}      &  55244 臺灣南投縣集集鎮民生東路1 號 \\
      \makebox[5em][s]{出\hspace{\fill}版\hspace{\fill}年\hspace{\fill}月}  &  2017 年 12 月 \\
      \makebox[5em][s]{I\hspace{\fill}S\hspace{\fill}B\hspace{\fill}N}      &  978-986-05-5021-4 \\
      \makebox[5em][s]{G\hspace{\fill}P\hspace{\fill}N}       & \\
      \makebox[5em][s]{定\hspace{\fill}價}      &  新台幣350元 \\
           &  \\
  \end{tabular}
\end{table}

\hfill \\
\begin{figure}[H]
    \includegraphics[width=10em]{images/ccby40.png}
\end{figure}
\noindent 本報告依據 \href{https://creativecommons.org/licenses/by/4.0}{Creative Commons Attribution 4.0 International License} 授權,
使用者可進行重製散布、重混、調整,以及依原著作建立新著作(包括商業與非商業性利用),惟使用時必須引用按照指定方式註明來源。\\
\hfill \\
引用方式:臺灣植物紅皮書編輯委員會(2017) 2017 臺灣維管束植物紅皮書名錄。行政院農業委員會特有生物研究保育中心、行政院農業委員會林務局、臺灣植物分類學會。南投。 \\
\noindent Citation: Editorial Committee of the Red List of Taiwan Plants (2017) The Red List of Vascular Plants of Taiwan. 
Endemic Species Research Institute, Forestry Bureau, Council of Agriculture, Executive Yuan and Society of Taiwan Plant Systematics.
Nantou, Taiwan.
