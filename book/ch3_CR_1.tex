\noindent\normalfont\selectfont Monilophytes 蕨類植物
\footnotesize\selectfont
%\begin{table}[!h]
        {\def\arraystretch{1.5}\tabcolsep=2pt
        \begin{longtable}{p{2.5cm}p{2.5cm}p{4.5cm}p{2.5cm}p{3cm}}
        \toprule
          科名 & 科中名 & 分類群學名 & 分類群中名 & 評估標準 \\
        \midrule 
        \endfirsthead

        {{\bfseries 續前頁 }} \\
        科名 & 科中名 & 分類群學名 & 分類群中名 & 評估標準 \\
        \midrule
        \endhead
                Aspleniaceae & 鐵角蕨科 & \textit{Asplenium crinicaule}  & 毛軸鐵角蕨 & B1ab(iii,v)c(iv)+2ab(iii,v)c(iv); C1+2a(ii)b; D \index{Asplenium@\textit{Asplenium}!crinicaule@\textit{crinicaule}}  \index{毛軸鐵角蕨} \\
    Blechnaceae & 烏毛蕨科 & \textit{Diploblechnum fraseri}  & 假桫欏 & C1+2a(ii) \index{Diploblechnum@\textit{Diploblechnum}!fraseri@\textit{fraseri}}  \index{假桫欏} \\
    Cystopteridaceae & 冷蕨科 & \textit{Gymnocarpium oyamense}  & 羽節蕨 & C2a(i) \index{Gymnocarpium@\textit{Gymnocarpium}!oyamense@\textit{oyamense}}  \index{羽節蕨} \\
    Dennstaedtiaceae & 碗蕨科 & \textit{Microlepia platyphylla}  & 闊葉鱗蓋蕨 & B1ab(ii,iii,iv)c(ii,iii,iv); C1+2a(ii)b; D \index{Microlepia@\textit{Microlepia}!platyphylla@\textit{platyphylla}}  \index{闊葉鱗蓋蕨} \\
    Dennstaedtiaceae & 碗蕨科 & \textit{Paesia radula}  & 曲軸蕨 & D \index{Paesia@\textit{Paesia}!radula@\textit{radula}}  \index{曲軸蕨} \\
    Dryopteridaceae & 鱗毛蕨科 & \textit{Elaphoglossum commutatum}  & 大葉舌蕨 & D \index{Elaphoglossum@\textit{Elaphoglossum}!commutatum@\textit{commutatum}}  \index{大葉舌蕨} \\
    Dryopteridaceae & 鱗毛蕨科 & \textit{Polystichum attenuatum}  & 長羽芽苞耳蕨 & B1ab(iii,v)c(ii,iv)+2ab(iii,v)c(ii,iv); C1+2a(ii)b; D \index{Polystichum@\textit{Polystichum}!attenuatum@\textit{attenuatum}}  \index{長羽芽苞耳蕨} \\
    Dryopteridaceae & 鱗毛蕨科 & \textit{Polystichum capillipes}  & 小耳蕨 & C2a(i) \index{Polystichum@\textit{Polystichum}!capillipes@\textit{capillipes}}  \index{小耳蕨} \\
    Dryopteridaceae & 鱗毛蕨科 & \textit{Polystichum chunii}  & 陳氏耳蕨 & B1ab(iii,v)c(ii,iv)+2ab(iii,v)c(ii,iv); C1+2a(ii)b; D \index{Polystichum@\textit{Polystichum}!chunii@\textit{chunii}}  \index{陳氏耳蕨} \\
    Dryopteridaceae & 鱗毛蕨科 & \textit{Polystichum grandifrons}  & 九州耳蕨 & B1ab(iii,v)c(ii,iv)+2ab(iii,v)c(ii,iv); C1+2a(ii)b; D \index{Polystichum@\textit{Polystichum}!grandifrons@\textit{grandifrons}}  \index{九州耳蕨} \\
    Dryopteridaceae & 鱗毛蕨科 & \textit{Polystichum herbaceum}  & 草葉耳蕨 & B1ab(iii,v)c(ii,iv)+2ab(iii,v)c(ii,iv); C1+2a(ii)b; D \index{Polystichum@\textit{Polystichum}!herbaceum@\textit{herbaceum}}  \index{草葉耳蕨} \\
    Dryopteridaceae & 鱗毛蕨科 & \textit{Polystichum tenuius}  & 離脈柳葉蕨 & A4cd; B1b(i,ii, iii, iv, v)c(ii,iii,iv)+2b(i,ii, iii,iv,v)c(ii,iii,iv) \index{Polystichum@\textit{Polystichum}!tenuius@\textit{tenuius}}  \index{離脈柳葉蕨} \\
    Hymenophyllaceae & 膜蕨科 & \textit{Crepidomanes bipunctatum}  & 南洋假脈蕨 & B1ab(iii)+2ab(iii); C1 \index{Crepidomanes@\textit{Crepidomanes}!bipunctatum@\textit{bipunctatum}}  \index{南洋假脈蕨} \\
    Hymenophyllaceae & 膜蕨科 & \textit{Crepidomanes parvifolium}  & 小葉假脈蕨 & D \index{Crepidomanes@\textit{Crepidomanes}!parvifolium@\textit{parvifolium}}  \index{小葉假脈蕨} \\
    Hymenophyllaceae & 膜蕨科 & \textit{Hymenophyllum palmatifidum}  & 毛緣細口團扇蕨 & D \index{Hymenophyllum@\textit{Hymenophyllum}!palmatifidum@\textit{palmatifidum}}  \index{毛緣細口團扇蕨} \\
    Hymenophyllaceae & 膜蕨科 & \textit{Hymenophyllum productum}  & 南洋蕗蕨 & B2ab(iii) \index{Hymenophyllum@\textit{Hymenophyllum}!productum@\textit{productum}}  \index{南洋蕗蕨} \\
    Hymenophyllaceae & 膜蕨科 & \textit{Hymenophyllum simonsianum}  & 寬片膜蕨 & D \index{Hymenophyllum@\textit{Hymenophyllum}!simonsianum@\textit{simonsianum}}  \index{寬片膜蕨} \\
    Hymenophyllaceae & 膜蕨科 & \textit{Hymenophyllum taiwanense}  & 臺灣蕗蕨 & D \index{Hymenophyllum@\textit{Hymenophyllum}!taiwanense@\textit{taiwanense}}  \index{臺灣蕗蕨} \\
    Marattiaceae & 觀音座蓮舅科 & \textit{Ptisana pellucida}  & 觀音座蓮舅 & C2a(ii) \index{Ptisana@\textit{Ptisana}!pellucida@\textit{pellucida}}  \index{觀音座蓮舅} \\
    Onocleaceae & 球子蕨科 & \textit{Pentarhizidium orientale}  & 東方莢果蕨 & C2a(i) \index{Pentarhizidium@\textit{Pentarhizidium}!orientale@\textit{orientale}}  \index{東方莢果蕨} \\
    Ophioglossaceae & 瓶爾小草科 & \textit{Helminthostachys zeylanica}  & 錫蘭七指蕨 & B2ab(iii) \index{Helminthostachys@\textit{Helminthostachys}!zeylanica@\textit{zeylanica}}  \index{錫蘭七指蕨} \\
    Plagiogyriaceae & 瘤足蕨科 & \textit{Plagiogyria koidzumii}  & 小泉氏瘤足蕨 & C2a(i) \index{Plagiogyria@\textit{Plagiogyria}!koidzumii@\textit{koidzumii}}  \index{小泉氏瘤足蕨} \\
    Polypodiaceae & 水龍骨科 & \textit{Lepisorus mucronatus}  & 尖嘴蕨 & C2a(i) \index{Lepisorus@\textit{Lepisorus}!mucronatus@\textit{mucronatus}}  \index{尖嘴蕨} \\
    Polypodiaceae & 水龍骨科 & \textit{Oreogrammitis caespitosa}  & 穴孢濱禾蕨 & D \index{Oreogrammitis@\textit{Oreogrammitis}!caespitosa@\textit{caespitosa}}  \index{穴孢濱禾蕨} \\
    Polypodiaceae & 水龍骨科 & \textit{Oreogrammitis marivelesensis}  & 弼昭禾葉蕨 & B1ab(iii)+2ab(iii); C2a(i,ii); D \index{Oreogrammitis@\textit{Oreogrammitis}!marivelesensis@\textit{marivelesensis}}  \index{弼昭禾葉蕨} \\
    Polypodiaceae & 水龍骨科 & \textit{Oreogrammitis orientalis}  & 東亞禾葉蕨 & D \index{Oreogrammitis@\textit{Oreogrammitis}!orientalis@\textit{orientalis}}  \index{東亞禾葉蕨} \\
    Polypodiaceae & 水龍骨科 & \textit{Prosaptia pectinata}  & 篦齒穴子蕨 & C2a(i) \index{Prosaptia@\textit{Prosaptia}!pectinata@\textit{pectinata}}  \index{篦齒穴子蕨} \\
    Polypodiaceae & 水龍骨科 & \textit{Radiogrammitis ilanensis}  & 宜蘭禾葉蕨 & C2a(i) \index{Radiogrammitis@\textit{Radiogrammitis}!ilanensis@\textit{ilanensis}}  \index{宜蘭禾葉蕨} \\
    Polypodiaceae & 水龍骨科 & \textit{Radiogrammitis moorei}  & 牟氏輻禾蕨 & B2ab(v); C2a(ii) \index{Radiogrammitis@\textit{Radiogrammitis}!moorei@\textit{moorei}}  \index{牟氏輻禾蕨} \\
    Polypodiaceae & 水龍骨科 & \textit{Xiphopterella devolii}  & 劍羽蕨 & B2ab(iii) \index{Xiphopterella@\textit{Xiphopterella}!devolii@\textit{devolii}}  \index{劍羽蕨} \\
    Pteridaceae & 鳳尾蕨科 & \textit{Adiantum capillus-junonis}  & 團羽鐵線蕨 & C2b \index{Adiantum@\textit{Adiantum}!capillus-junonis@\textit{capillus-junonis}}  \index{團羽鐵線蕨} \\
    Pteridaceae & 鳳尾蕨科 & \textit{Haplopteris heterophylla}  & 異葉書帶蕨 & B1bc; C2a(iii)b \index{Haplopteris@\textit{Haplopteris}!heterophylla@\textit{heterophylla}}  \index{異葉書帶蕨} \\
    Pteridaceae & 鳳尾蕨科 & \textit{Pteris angustipinna}  & 細葉鳳尾蕨 & C2a(i) \index{Pteris@\textit{Pteris}!angustipinna@\textit{angustipinna}}  \index{細葉鳳尾蕨} \\
    Pteridaceae & 鳳尾蕨科 & \textit{Pteris dimorpha} var. \textit{metagrevilleana}  & 擬翅柄鳳尾蕨 & B2ac(ii); C2b; D \index{Pteris@\textit{Pteris}!dimorpha@\textit{dimorpha}!var. metagrevilleana@var. \textit{metagrevilleana}}  \index{擬翅柄鳳尾蕨} \\
    Pteridaceae & 鳳尾蕨科 & \textit{Pteris × wulaiensis}   & 烏來鳳尾蕨 & C2a(i); D \index{Pteris@\textit{Pteris}!×@\textit{×}!wulaiensis @wulaiensis \textit{}}  \index{烏來鳳尾蕨} \\
    Pteridaceae & 鳳尾蕨科 & \textit{Vaginularia trichoidea}  & 一條線蕨 & B1b(ii,iii)c(ii,iv)+2b(ii,iii)c(ii,iv); C2b; D \index{Vaginularia@\textit{Vaginularia}!trichoidea@\textit{trichoidea}}  \index{一條線蕨} \\
    Salviniaceae & 槐葉萍科 & \textit{Salvinia natans}  & 槐葉蘋 & A2c \index{Salvinia@\textit{Salvinia}!natans@\textit{natans}}  \index{槐葉蘋} \\
    Schizaeaceae & 莎草蕨科 & \textit{Schizaea dichotoma}  & 分枝莎草蕨 & B2ab(i,iv); C2a(i) \index{Schizaea@\textit{Schizaea}!dichotoma@\textit{dichotoma}}  \index{分枝莎草蕨} \\
    Thelypteridaceae & 金星蕨科 & \textit{Metathelypteris flaccida}  & 薄葉凸軸蕨 & B1ac(ii,iii,iv)+2ac(ii,iii,iv) \index{Metathelypteris@\textit{Metathelypteris}!flaccida@\textit{flaccida}}  \index{薄葉凸軸蕨} \\
    Woodsiaceae & 岩蕨科 & \textit{Woodsia andersonii}  & 蜘蛛岩蕨 & B1ab(v)+2ab(v); D \index{Woodsia@\textit{Woodsia}!andersonii@\textit{andersonii}}  \index{蜘蛛岩蕨} \\
    Woodsiaceae & 岩蕨科 & \textit{Woodsia okamotoi}  & 岡本氏岩蕨 & C2a(i) \index{Woodsia@\textit{Woodsia}!okamotoi@\textit{okamotoi}}  \index{岡本氏岩蕨} \\
    \bottomrule
        \end{longtable}
    %%\end{table}
        }
    