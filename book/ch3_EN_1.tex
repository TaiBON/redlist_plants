\noindent\normalfont\selectfont Monilophytes 蕨類植物
\footnotesize\selectfont
%\begin{table}[!h]
        \begin{longtable}{p{3cm}p{5cm}p{3cm}p{4cm}}
        \toprule
          科名 (科中名) & 分類群學名 & 分類群中名 & 評估標準 \\
        \midrule 
        \endfirsthead

        {{\bfseries 續前頁 }} \\
        科名 (科中名) & 分類群學名 & 分類群中名 & 評估標準 \\
        \midrule
        \endhead
                Aspleniaceae (鐵角蕨科) & \textit{Asplenium pifongiae}  & 碧鳳鐵角蕨 & B2ac(i,ii,iv) \index{Asplenium@\textit{Asplenium}!pifongiae@\textit{pifongiae}}  \index{碧鳳鐵角蕨} \\
    Athyriaceae (蹄蓋蕨科) & \textit{Athyrium minimum}  & 七星山蹄蓋蕨 & C2a(ii) \index{Athyrium@\textit{Athyrium}!minimum@\textit{minimum}}  \index{七星山蹄蓋蕨} \\
    Athyriaceae (蹄蓋蕨科) & \textit{Diplazium chinense}  & 中華雙蓋蕨 & B1ac(iv); D \index{Diplazium@\textit{Diplazium}!chinense@\textit{chinense}}  \index{中華雙蓋蕨} \\
    Athyriaceae (蹄蓋蕨科) & \textit{Diplazium incomptum}  & 翅柄雙蓋蕨 & B2ab(v); D \index{Diplazium@\textit{Diplazium}!incomptum@\textit{incomptum}}  \index{翅柄雙蓋蕨} \\
    Davalliaceae (骨碎補科) & \textit{Davallia pectinata}  & 馬來陰石蕨 & D \index{Davallia@\textit{Davallia}!pectinata@\textit{pectinata}}  \index{馬來陰石蕨} \\
    Dennstaedtiaceae (碗蕨科) & \textit{Hypolepis pallida}  & 灰姬蕨 & B1ac(ii,iii,iv)+2ac(ii,iii,iv) \index{Hypolepis@\textit{Hypolepis}!pallida@\textit{pallida}}  \index{灰姬蕨} \\
    Dryopteridaceae (鱗毛蕨科) & \textit{Arachniodes chinensis}  & 中華複葉耳蕨 & B1ac(iii,iv)+2ac(iii,iv) \index{Arachniodes@\textit{Arachniodes}!chinensis@\textit{chinensis}}  \index{中華複葉耳蕨} \\
    Dryopteridaceae (鱗毛蕨科) & \textit{Cyrtomium macrophyllum} var. \textit{macrophyllum}  & 大葉貫眾蕨 & D \index{Cyrtomium@\textit{Cyrtomium}!macrophyllum@\textit{macrophyllum}!var. macrophyllum@var. \textit{macrophyllum}}  \index{大葉貫眾蕨} \\
    Dryopteridaceae (鱗毛蕨科) & \textit{Cyrtomium macrophyllum} var. \textit{simadae}  & 尖葉貫眾蕨 & B1ac(ii,iii,iv)+2ac(ii,iii,iv) \index{Cyrtomium@\textit{Cyrtomium}!macrophyllum@\textit{macrophyllum}!var. simadae@var. \textit{simadae}}  \index{尖葉貫眾蕨} \\
    Dryopteridaceae (鱗毛蕨科) & \textit{Dryopteris kinkiensis}  & 近畿鱗毛蕨 & D \index{Dryopteris@\textit{Dryopteris}!kinkiensis@\textit{kinkiensis}}  \index{近畿鱗毛蕨} \\
    Dryopteridaceae (鱗毛蕨科) & \textit{Dryopteris kwanzanensis}  & 擬倒鱗鱗毛蕨 & D \index{Dryopteris@\textit{Dryopteris}!kwanzanensis@\textit{kwanzanensis}}  \index{擬倒鱗鱗毛蕨} \\
    Dryopteridaceae (鱗毛蕨科) & \textit{Dryopteris toyamae}  & 外山氏鱗毛蕨 & C2a(i); D \index{Dryopteris@\textit{Dryopteris}!toyamae@\textit{toyamae}}  \index{外山氏鱗毛蕨} \\
    Dryopteridaceae (鱗毛蕨科) & \textit{Dryopteris yoroii}  & 上先型鱗毛蕨 & B2ac(iv) \index{Dryopteris@\textit{Dryopteris}!yoroii@\textit{yoroii}}  \index{上先型鱗毛蕨} \\
    Dryopteridaceae (鱗毛蕨科) & \textit{Polystichum xiphophyllum}  & 關山耳蕨 & D \index{Polystichum@\textit{Polystichum}!xiphophyllum@\textit{xiphophyllum}}  \index{關山耳蕨} \\
    Hymenophyllaceae (膜蕨科) & \textit{Crepidomanes bilabiatum}  & 圓唇假脈蕨 & B2ab(iii) \index{Crepidomanes@\textit{Crepidomanes}!bilabiatum@\textit{bilabiatum}}  \index{圓唇假脈蕨} \\
    Hymenophyllaceae (膜蕨科) & \textit{Hymenophyllum digitatum}  & 指裂細口團扇蕨 & D \index{Hymenophyllum@\textit{Hymenophyllum}!digitatum@\textit{digitatum}}  \index{指裂細口團扇蕨} \\
    Marattiaceae (觀音座蓮舅科) & \textit{Angiopteris somae}  & 臺灣原始觀音座蓮 & B1ab(ii) \index{Angiopteris@\textit{Angiopteris}!somae@\textit{somae}}  \index{臺灣原始觀音座蓮} \\
    Ophioglossaceae (瓶爾小草科) & \textit{Botrychium ternatum}  & 大陰地蕨 & C2a(i) \index{Botrychium@\textit{Botrychium}!ternatum@\textit{ternatum}}  \index{大陰地蕨} \\
    Ophioglossaceae (瓶爾小草科) & \textit{Botrychium virginianum}  & 蕨萁 & D \index{Botrychium@\textit{Botrychium}!virginianum@\textit{virginianum}}  \index{蕨萁} \\
    Polypodiaceae (水龍骨科) & \textit{Chrysogrammitis glandulosa}  & 擬虎尾蒿蕨 & D \index{Chrysogrammitis@\textit{Chrysogrammitis}!glandulosa@\textit{glandulosa}}  \index{擬虎尾蒿蕨} \\
    Polypodiaceae (水龍骨科) & \textit{Dasygrammitis mollicoma}  & 南洋蒿蕨 & D \index{Dasygrammitis@\textit{Dasygrammitis}!mollicoma@\textit{mollicoma}}  \index{南洋蒿蕨} \\
    Polypodiaceae (水龍骨科) & \textit{Loxogramme biformis}  & 二形劍蕨 & D \index{Loxogramme@\textit{Loxogramme}!biformis@\textit{biformis}}  \index{二形劍蕨} \\
    Polypodiaceae (水龍骨科) & \textit{Microsorum steerei}  & 廣葉星蕨 & D \index{Microsorum@\textit{Microsorum}!steerei@\textit{steerei}}  \index{廣葉星蕨} \\
    Polypodiaceae (水龍骨科) & \textit{Oreogrammitis congener} var. \textit{polytricha}  & 多毛禾葉蕨 & D \index{Oreogrammitis@\textit{Oreogrammitis}!congener@\textit{congener}!var. polytricha@var. \textit{polytricha}}  \index{多毛禾葉蕨} \\
    Polypodiaceae (水龍骨科) & \textit{Oreogrammitis nuda}  & 長孢禾葉蕨 & B2ab(iii) \index{Oreogrammitis@\textit{Oreogrammitis}!nuda@\textit{nuda}}  \index{長孢禾葉蕨} \\
    Polypodiaceae (水龍骨科) & \textit{Phymatosorus longissimus}  & 水社擬茀蕨 & D \index{Phymatosorus@\textit{Phymatosorus}!longissimus@\textit{longissimus}}  \index{水社擬茀蕨} \\
    Polypodiaceae (水龍骨科) & \textit{Prosaptia nutans}  & 俯垂穴子蕨 & B2ab(iii) \index{Prosaptia@\textit{Prosaptia}!nutans@\textit{nutans}}  \index{俯垂穴子蕨} \\
    Polypodiaceae (水龍骨科) & \textit{Pyrrosia assimilis}  & 相似石韋 & ? \index{Pyrrosia@\textit{Pyrrosia}!assimilis@\textit{assimilis}}  \index{相似石韋} \\
    Pteridaceae (鳳尾蕨科) & \textit{Adiantum monochlamys}  & 石長生 & B2ab(iii) \index{Adiantum@\textit{Adiantum}!monochlamys@\textit{monochlamys}}  \index{石長生} \\
    Pteridaceae (鳳尾蕨科) & \textit{Paragymnopteris vestita}  & 金毛裸蕨 & C2a(i) \index{Paragymnopteris@\textit{Paragymnopteris}!vestita@\textit{vestita}}  \index{金毛裸蕨} \\
    Pteridaceae (鳳尾蕨科) & \textit{Pteris pellucida}  & 爪哇鳳尾蕨 & D \index{Pteris@\textit{Pteris}!pellucida@\textit{pellucida}}  \index{爪哇鳳尾蕨} \\
    Thelypteridaceae (金星蕨科) & \textit{Macrothelypteris polypodioides}  & 桫欏大金星蕨 & B1b(ii,iii,iv,v)c(ii,iii,iv) \index{Macrothelypteris@\textit{Macrothelypteris}!polypodioides@\textit{polypodioides}}  \index{桫欏大金星蕨} \\
    Thelypteridaceae (金星蕨科) & \textit{Stegnogramma dictyoclinoides}  & 溪邊蕨 & B2ab(v) \index{Stegnogramma@\textit{Stegnogramma}!dictyoclinoides@\textit{dictyoclinoides}}  \index{溪邊蕨} \\
    Thelypteridaceae (金星蕨科) & \textit{Stegnogramma pozoi}  & 非洲茯蕨 & D \index{Stegnogramma@\textit{Stegnogramma}!pozoi@\textit{pozoi}}  \index{非洲茯蕨} \\
    \bottomrule
        \end{longtable}
    %%\end{table}